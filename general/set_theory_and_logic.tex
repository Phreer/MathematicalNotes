\chapter{Set Theory and Logic}
%%%%%%%%%%%%%%%%%%%%%%%%%%%%%%%%%%%%%%%%%%%%%%%%%%%%%%%
%%  Section: Set Theory
%%%%%%%%%%%%%%%%%%%%%%%%%%%%%%%%%%%%%%%%%%%%%%%%%%%%%%%
\section{Set Theory}
Set theory is the foundation of modern Mathematics. 
Here, we discuss about axiomatic set theory. 

Intuitively, sets are collections of objects. 
However, we have no way to give a further definition for collection and 
object. 
The only thing we can deal with sets is to give a axiomatic definition of 
sets. 
Most proofs in this section which are just direct applications of the axioms 
presented below is omitted. 

First, we should specify hwo can be sets generated. 

\begin{axiom}[Sets are objects]
If $A$ is a set, then $A$ is an object. 
In particular, given two sets $A$ and $B$, it is meaningful to ask whether 
$A$ is also an element of $B$. 
\end{axiom}

\begin{axiom}[Equality of sets]
Two sets $A$ and $B$ are equal, denoted by $A = B$ if every element of $A$ 
is an element of $B$ and vice versa. 
\end{axiom}

\begin{axiom}[Empty set]
There exists a set $\varnothing$ which contains no elements, \ie, for every 
object $x$ we have $x \notin \varnothing$. 
\end{axiom}

\begin{lemma}
If $A$ is a nonempty set, then there exists an object $x$ such that $x \in 
A$. 
\end{lemma}

\begin{axiom}[Singleton sets]
If $a$ is an object, then there exists a set $\{a\}$ whose only element is 
$a$. 
\end{axiom}

\begin{axiom}[Pairwise union]
Given any two sets $A$ and $B$, there exists a set $A \cup B$, called the 
union of $A$ and $B$, whose elements of all the elements which belong to $A$ 
or $B$ or both, namely, for any object 
\begin{equation*}
    x \in A \cup B \iff x \in A \text{ or } x \in B.
\end{equation*}
\end{axiom}

\begin{prop}
\begin{enumerate}
    \item If $a, b$ are objects, then $\{a, b\} = \{a\} \cup \{b\}$. 
    \item The union operation is commutative, \ie, $A \cup B = B \cup A$ 
    for any sets $A, B$. 
    \item $A \cup A = A \cup \varnothing = A$.
\end{enumerate}
\end{prop}

\begin{defn}[Subsets]
Let $A$, $B$ be sets. 
We say that $A$ is a subset of $B$, denoted $A \subseteq B$, if for any object 
$x$, $x \in B$ implies $x \in B$. 
\end{defn}

\begin{prop}[Sets are partially ordered by set inclusion]
Let $A, B, C$ be sets. 
If $A \subseteq B$ and $B \subseteq C$, then $A \subseteq C$. 
If $A \subsetneq B$ and $B \subsetneq C$, then $A \subsetneq C$. 
\end{prop}

\begin{axiom}[Specification]
Let $A$ be a set, and for each $x \in A$, let $P(x)$ be a property 
pertaining to $x$ (\ie, $P(x)$ is either a true statement or a false 
statement). 
Then there exists a set $\{x \in A: P(x)\}$ whose elements are precisely 
those elements in $A$ for which $P(x)$ is true, \ie 
\begin{equation*}
    y \in \{x \in A: P(x)\} \iff y \in A \text{ and } P(y) \text{ is true}.
\end{equation*}
\end{axiom}

\begin{defn}[Intersection]
The intersection $A \cap B$ of two sets $A, B$ is defined to be the set 
\begin{equation*}
    A \cap B = \{x \in A: x \in B\}. 
\end{equation*}
\end{defn}

\begin{defn}[Disjoint set]
Two sets $A$ and $B$ are said to be \emph{disjoint} if $A \cap B = 
\varnothing$. 
\end{defn}

\begin{defn}[Difference set]
Let $A, B$ be sets. 
We define the set $A \backslash B$ (or $A - B$) to be the set
\begin{equation*}
    A \backslash B = \{x \in A: x \notin B\}. 
\end{equation*}
\end{defn}

\begin{prop}[Sets form a boolean algebra]
Let $A, B, C$ be sets and let $X$ be a set containing $A, B, C$ as subsets. 
Then we have 
\begin{enumerate*}
    \item (Minimal element) $A \cup \varnothing = A$ and $A \cap \varnothing 
    = \varnothing$; 
    \item (Maximal element) $A \cup X = X$ and $A \cap X = A$; 
    \item (Identity) $A \cup A = A$ and $A \cap A = A$; 
    \item (Commutativity) $A \cup B = B \cup B$ and $A \cap B = B \cap A$; 
    \item (Associativity) $(A \cup B) \cup C = A \cup (B \cup C)$ and 
    $(A \cap B) \cap C = A \cap (B \cap C)$; 
    \item (Distributivity) $A \cap (B \cup C) = (A \cap B) \cup (A \cap C)$ 
    and $A \cup (B \cap C) = (A \cup B) \cap (A \cup C)$; 
    \item (Partition) $A \cup (X \backslash A) = X$ and 
    $A \cap (X \backslash) = \varnothing$; 
    \item (De Morgan laws) $X \backslash (A \cup B) = (X \backslash A) 
    \cap (X \backslash B)$ and $X \backslash (A \cap B) = (X \backslash A) 
    \cup (X \backslash B)$. 
\end{enumerate*}
\end{prop}

\begin{axiom}[Replacement]
\label{axiom:set_theory:replacement}
Let $A$ be a set. 
For any object $x \in A$ and any object $y$, suppose that we have a 
statement $P(x, y)$ pertaining to $x$ and $y$ such that for each $x \in A$ 
there exists at most one $y$ for which $P(x, y)$ is true. 
Then there exists a set $\{y: P(x, y), x \in A\}$ such that for any 
object $z$, 
\begin{equation*}
    z \in \{y: P(x, y), x \in A\} \iff 
    P(x, z) \text{ is true for some } x \in A. 
\end{equation*}
\end{axiom}

For instance, let $A = \{2, 4, 5\}$ and $P(x, y)$ be the statement $y = x++$. 
Then $\{y: P(x, y), x \in A\} = \{3, 5, 6\}$. 
We often abbreviate a set of the form 
\begin{equation*}
    \{y: y = f(x) \text{ for some } x \in A\}
\end{equation*}
as $\{f(x): x \in A\}$. 

\begin{axiom}[Existence of infinity]
There exists a set $\bN$, whose elements are called natural numbers, as 
well as an object $0 \in \bN$, and an object $n++$ assigned to every natural 
number $n \in \bN$, such that the Peano axioms hold. 
\end{axiom}

To avoid Russell's Paradox, we need the following axiom that restrain the 
extent of sets we considered. 
\begin{axiom}[Regularity]
If $A$ is a nonempty set, then there is at least one element $x$ of $A$ 
which is either not a set, or is disjoint from $A$. 
\end{axiom}

This axiom filters out ``too large'' sets. 
This axiom is less intuitive and fortunately is never need in the study of 
analysis, in which the sets we consider are of very low hierarchy. 

In analysis, we need a further axiom for our set theory. 
\begin{axiom}[Union]
\label{axiom:set_theory:union}
Let $A$ be a set, all of whose elements are themselves sets. 
Then there exists a set $\bigcup A$ whose elements are precisely those 
objects which are elements of the elements of $A$, \ie, 
\begin{equation*}
    x \in \bigcup A \iff x \in S \text{ for some } S \in A. 
\end{equation*}
\end{axiom}

If one has a set $I$ to which we often refer as an \emph{index set} and for 
every element $\alpha \in I$, we have a set $A_\alpha$, then we can form 
the union set $\bigcup_{\alpha \in I} A_\alpha$ by defining 
\begin{equation*}
    \bigcup_{\alpha \in I} A_\alpha 
    = \bigcup \left\{ A_\alpha: \alpha \in I \right\}, 
\end{equation*}
which is a set thanks to the Axiom of Replacement 
\ref{axiom:set_theory:replacement}. 
Besides, the set $\left\{ A_\alpha: \alpha \in I \right\}$ is called a 
\emph{family of set}. 
However, given a family of sets $\{ A_\alpha: \alpha \in I \}$ such that 
$I$ is nonempty, notice that the existence intersection of  defined as 
\begin{equation*}
    \bigcap_{\alpha \in I} A_\alpha 
    = \left\{ x \in A_\beta: 
        x \in A_\alpha \text{ for all } \alpha \in I \right\}
\end{equation*}
where $\beta$ is an element of $I$ (We can do this since $I$ is nonempty), 
is guaranteed by the Axiom of Specification. 
One can show that result of $\left\{ x \in A_\beta: x \in A_\alpha 
\text{ for all } \alpha \in I \right\}$ does not depend on the choice of 
$\beta$, whence $\bigcap_{\alpha \in I} A_\alpha$ is well-defined. 

\index{Zermelo-Fraenkel axioms}
\index{Zermelo-Fraenkel-Choice (ZFC) axioms of set theory}
The collection of axioms of set theory we have given are known as the 
\emph{Zermelo-Fraenkel axioms of set theory}. 
There is one further axiom we need, the famous axiom of choice, giving rise 
to the \emph{Zermelo-Fraenkel-Choice (ZFC) axioms of set theory}. 

Although a large portion of the foundation of analysis can be developped 
without it, the Axiom of Choice is a very powerful and even essential tool 
for further development. 

Before giving the Axiom of Choice, we present the definition of Cartesian 
product:
\begin{defn}[Cartesian product]
Let $I$ be a set, and for each $\alpha \in I$, let $X_\alpha$ be a set. 
We then define the Cartesian product $\prod_{\alpha \in I} X_\alpha$ to be 
the set 
\begin{equation*}
    \prod_{\alpha \in I} X_\alpha 
    = \left\{ (x_\alpha)_{\alpha \in I} \in 
        \left( \bigcup_{\beta \in I}X_\beta \right)^I: 
        x_\alpha \in X_\alpha \quad \text{for all } \alpha \in I 
        \right\}
\end{equation*}
\end{defn}

\begin{axiom}[Choice]
Let $I$ be a set, and for each $\alpha \in I$, let $X_\alpha$ be a non-empty 
set. 
Then $\prod_{\alpha \in I} X$ is also a non-empty set. 
\end{axiom}

There are some other forms of the Axiom of Choice. 
\begin{prop}
Let $X$ and $Y$ be sets, and let $P(x, y)$ be a property pertaining to 
objects $x \in X, y \in Y$ such that for every $x \in X$ there exists at 
least one $y \in X$ such that $P(x, y)$ is true. 
Then there exists a function $f: X: \to Y$ such that $P(x, f(x))$ holds 
true for all $x \in X$. 
\end{prop}
\begin{proof}
We will show that this proposition is equivalent to the Axiom of Choice. 
Let $Y_x = \{y \in Y: P(x, y) \text{ is true}\}$. 
Then by the Axiom of Choice, the set $\prod_{x \in X} Y_x$ is non-empty. 
Therefore, there exists a function 
\begin{equation*}
    (f: X \to Y) \in \left( \bigcup_{x \in X} Y_x \right)^X
\end{equation*}
such that $P(x, f(x))$ is true.  

On the other hand, suppose the proposition holds true. 
Let $I$ be a set such that for every $\alpha \in I$ there is a corresponding 
set $X_\alpha$, and let $P(\alpha, x)$ be a property that holds true if and 
only if $x \in X_\alpha$ for $\alpha \in I$ and $x \in \bigcup_
{\alpha \in I} X_\alpha$. 
Then there exists a function $f: I \to \bigcup_{\alpha \in I} X_\alpha$ such 
that $P(\alpha, f(\alpha))$ holds true for all $\alpha \in I$. 
It is clear that $f \in \prod_{\alpha \in I} X_\alpha$, which completes the 
proof. 
\end{proof}

%%%%%%%%%%%%%%%%%%%%%%%%%%%%%%%%%%%%%%%%%%%%%%%%%%%%%%%
%%  Section: Functions
%%%%%%%%%%%%%%%%%%%%%%%%%%%%%%%%%%%%%%%%%%%%%%%%%%%%%%%
\section{Functions}
\begin{defn}[Function]
Let $X, Y$ be sets, and let $P(x, y)$ be a property pertaining to an object 
$x \in X$ and an object $y \in Y$, such that for every $x \in X$, there is 
exactly one $y \in Y$ for which $P(x, y)$ is true (this is sometimes known as 
the vertical line test). 
Then we define the function $f : X \to Y$ defined by $P$ on the domain $X$ 
and range $Y$ to be the object which, given any input $x \in X$, assigns an 
output $f(x) \in Y$, defined to be the unique object $f(x)$ for which 
$P(x, f(x))$ is true. 
Thus, for any $x ∈ X$ and $y ∈ Y$,
\begin{equation*}
    y = f(x) \iff P(x, y) \text{ is true}.
\end{equation*}

The set $X$ is said to be the domain of function $f: X \to Y$. 
We refer the range of a function $f: X \to Y$ the subset $f(X) \coloneqq 
\{f(x): x \in X\}$ of $Y$. 
\end{defn}


\begin{rmk}
Sometimes, functions are referred to as maps, mappings and transformations. 
And on some context, the terminlogy ``function'' means functions that whose 
range is number filed. 
\end{rmk} 

\begin{defn}[Equality of functions]
Two functions $f: X \to Y$, $g: X \to Y$ with the same domain and range are 
said to be equal, denoted as $f = g$ if $f(x) = g(x)$ for all $x \in X$. 
\end{defn}

\begin{defn}[Composition]
Let $f: X \to Y$ and $g: Y \to Z$ be two functions. 
We then define the \emph{composition} $g \circ f$ of $g$ and $f$ to be the 
function defined explicitly by the formula 
\begin{equation*}
    (g \circ f)(x) \coloneqq g(f(x)). 
\end{equation*}
\end{defn}

\begin{prop}[Composition is associative]
Let $f: X \to W$, $g: Y \to Z$ and $h: X to Y$ be functions. 
Then $f \circ (g \circ h) = (f \circ g) \circ h$. 
\end{prop}

\begin{defn}[One-to-one function, Onto function, Bijective function]
A function $f: X \to Y$ is \emph{one-to-one} (or \text{injective}) if 
different elements map to different elements: 
\begin{equation*}
    x \neq x' \implies f(x) \neq f(x'). 
\end{equation*}

A function $f: X \to Y$ is \emph{onto} (or \emph{surjective}) if $f(X) = Y$, 
\ie, for every element $y \in Y$, there exists $x \in X$ such that 
$f(x) = y$.

A function $f: X \to Y$ is said to be \emph{bijective} if it is both one-
to-one and onto. 
\end{defn}

\begin{rmk}
If a function $f: X \to Y$ is bijective, then we sometimes call $f$ a 
\emph{perfect function} or \emph{one-to-one correspondence}. 
In this case, denote the action of $f$ using the notation $x \leftrightarrow 
f(x)$ instead of $x \mapsto f(x)$. 
\end{rmk}

If $f: X \to Y$ is a bijective function, then for each $y \in Y$, there 
exists exactly one element $x \in X$ such that $f(x) = y$. 
In this way, denoting $x$ by $f^{-1}(y)$, we obtains a function $f^{-1}$ 
from $Y$ to $X$ by $y \mapsto f^{-1}(y)$. 
We call $f^{-1}$ the inverse of $f$. 

To consider the set of all functions from $X$ to $Y$, we need the following 
axiom of set theory. 
\begin{axiom}[Power set]
Let $X$, $Y$ be two sets. 
Then there exists a set, denoted $Y^X$, that consists of all the functions 
from $X$ to $Y$. 
\end{axiom}

\begin{prop}
Let $X$ be a set. 
Then $\{Y: Y \subseteq X\}$ is a set. 
\end{prop}
\begin{proof}
By the axiom of power set, there exists a set $\{0, 1\}^X$ consisting of 
all functions from $X \to \{0, 1\}$. 
For each subset $Y$ of $X$, we can construct a function $f_Y: X \to 
\{0, 1\}$ such that $f(x) = 1$ if $x \in Y$, else $f(x) = 0$, whence 
\begin{equation*}
    \left\{ f(\{1\}): f \in \{0, 1\} ^X \right\}
    \supseteq \{ Y: Y \subset X\}. 
\end{equation*}
Conversely, for each $f \in \{0, 1\}^X$, $f(\{1\})$ is exactly one subset 
of $X$. 
Hence, 
\begin{equation*}
    \left\{ f(\{1\}): f \in \{0, 1\} ^X \right\}
    \subseteq \{ Y: Y \subset X\}. 
\end{equation*}
Then we can draw a conclusion that 
\begin{equation*}
    \left\{ f(\{1\}): f \in \{0, 1\} ^X \right\}
    = \{ Y: Y \subset X\}
\end{equation*}
does exist.
\end{proof}
If $X$ is a set, we denote the set $\{Y: Y \subset X\}$ by $2^X$, which is 
conventionally called the power set of $X$. 

%%%%%%%%%%%%%%%%%%%%%%%%%%%%%%%%%%%%%%%%%%%%%%%%%%%%%%%
%%  Section: Ordered Sets
%%%%%%%%%%%%%%%%%%%%%%%%%%%%%%%%%%%%%%%%%%%%%%%%%%%%%%%
\section{Ordered Sets}
\begin{defn}[Partially ordered set]
\index{partially ordered set}
\index{poset}
A \emph{partially ordered set} (or \emph{poset}) is a pair $(X, \le_X)$ 
where a set $X$ and $\le_X$ a relation on $X$ (That is, for any two objects 
$x, y \in X$, the statement $x \le y$ is either a true or a false one) that 
obeys the following three properties: 
\begin{enumerate}
    \item (Reflexivity) For any $x \in X$, we have $x \le_X x$. 
    \item (Anti-symmetry) If $x, y in X$ are such that $x \le_X y$ and 
    $y \le_X x$, then $x = y$. 
    \item (Transitivity) If $x, y, z \in X$ are such that $x \le_X y$ and 
    $y \le_X z$, then $x \le_X z$. 
\end{enumerate}
\end{defn}

We refer to $\le_X$ as the \emph{ordering relation} and sometimes write 
$\le$ for short when there is no ambiguity. 
By $x <_X y$ for $x, y \in X$, we mean $x \le_X y$ and $x \neq y$. 
Note that for a partially ordered set $X$, it may occur that two elements 
$x, y \in X$ are such that neither $x \le_X y$ nor $y \le_X x$ is true. 


\begin{defn}[Totally ordered set]
    A subset $Y$ in a partially ordered set $(X, \le_X)$ is said to be a 
    \emph{totally ordered set} if for any $y, y' \in Y$, we either have 
    $y \le_X y'$ or $y' \le_X y$ or both. 
\end{defn}

\begin{example}
Let $X$ be a set. 
Then $(2^X, \subseteq)$ is a partially ordered set but is not a totally 
ordered set. 
\end{example}

\begin{defn}[Maximal and minimal element]
\index{minimal element}
\index{maximal element}
Let $X$ be a partially ordered set and let $Y$ be a subset of $X$. 
We say that $y \in Y$ is a \emph{minimal element} of $Y$ if there is no 
element $y' \in Y$ such that $y ' < y$, \ie, 
\begin{equation*}
    y' \le y \implies y' = y \quad \text{for any } y' \in Y.
\end{equation*}
Likewise, $y$ is said to be a \emph{maximal element} of $Y$ if there is 
no element $y' \in Y$ such that $y < y'$. 
\end{defn}

\begin{defn}[Well-ordered set]
\index{well-ordered set}
Let $X$ be a partially ordered set and let $Y$ be a totally ordered subset 
of $X$. 
We say that $Y$ is \emph{well-ordered} if every non-empty subset of $Y$ has 
a minimal element $\min(Y) \in Y$. 
\end{defn}

\begin{example}
The natural numbers $\bN$ together with the canonical order is a totally 
ordered and well-ordered set, while the integers $\bZ$, the rational numbers 
$\bQ$ and the rational numbers $\bR$ are not well-ordered. 
\end{example}

One can show that finite totally ordered sets are well-ordered and every 
subset of a well-ordered set is again well-ordered. 

\begin{thm}[Principle of strong induction]
Let $X$ be a well-ordered set with an ordering relation $\le$, and let 
$P(x)$ be a property pertaining to element $x \in X$ (\ie, for each $x \in 
X)$, $P(x)$ is either a true or false statement). 
If for every $x \in X$, we have the following implication: 
\begin{equation*}
    P(m) \text{ is true for all } x' < x \implies P(x) \text{ is true}, 
\end{equation*}
then $P(x)$ is true for all $x \in X$. 
\end{thm}
\begin{proof}
Suppose the hypothesis is satisfied, and consider the set 
\begin{equation*}
    Y = \{x \in X: P(x') \text{ is false for some } x' \le x\}.
\end{equation*}
Then since $X$ is well-ordered, $Y$ has a minimal element $x_0$. 
Thus for every $x \in X$ with $x < x_0$, $P(x')$ holds true for every 
$x' \in X$ such that $x' \le x$. 
This means that for every $x < x_0$, $P(x)$ holds true, which implies 
$P(x')$ holds true for all $x' \le x_0$, which is a contradiction to the 
fact $x_0 \in Y$. 
\end{proof}

Until now the Axiom of Choice hasn't played a role in our text. 
\begin{defn}[Upper bound and strict upper bound]
\index{upper bound}
\index{strict upper bound}
Let $X$ be a partially ordered set and let $Y$ be a totally ordered subset 
of $X$. 
If $x \in X$, we say $x$ is a \emph{upper bound} of $Y$ if $y \le x$ for all 
$y \in Y$. 
If in addition $x \neq Y$, then we say that $x$ is a \emph{strict upper 
bound} of $Y$. 
Equivalently, $x$ is a strict upper bound of $Y$ if and only if $y < x$ 
for every $y \in Y$
\end{defn}

With Axiom of Choice, we can prove the following lemma: 
\begin{lemma}
Let $X$ be a partially ordered set with ordering relation $\le$, and let 
$x_0$ be an element of $X$. 
Then there exists a well-ordered subset $Y$ of $X$ which has $x_0$ as its 
minimal element and which has no strict upper bound. 
\end{lemma}
\begin{proof}

\end{proof}

Now we are ready to demostrate the well-known Zorn's Lemma: 
\begin{lemma}[Zorn's Lemma]\label{lemma:zorn_lemma}
\index{lemma!Zorn's \~{}}
Let $X$ be a non-empty partially ordered set, with the property that every 
totally ordered subset $Y$ of $X$ has an upper bound. 
Then $X$ contains at least one maximal element. 
\end{lemma}
%%%%%%%%%%%%%%%%%%%%%%%%%%%%%%%%%%%%%%%%%%%%%%%%%%%%%%%
%%  Section: Cardinality
%%%%%%%%%%%%%%%%%%%%%%%%%%%%%%%%%%%%%%%%%%%%%%%%%%%%%%%
\section{Cardinality}
For sets $\{1, 2, 3\}$ and $\{4, 5, 6\}$, we have a intuition that these two 
sets are the the same size. 
But now we have not defined what does ``have the same size'' mean. 
Besides, the natural number constructed on Peano axioms behaves much more 
like ordinals (first, second, third, \dots) but not cardinality (one, two, 
three, \dots). 
In this section, we are dealing with this problem. 
\begin{defn}[Equal cardinality]
Two sets $X$ and $Y$ are said to have equal cardinality if there is a 
bijective function from $X$ to $Y$. 
\end{defn}

\begin{example}
The set $\bN$ of all natural numbers and the set $Y = \{2n: n \in \bN \}$ 
of all non-negative even numbers have equal cardinality since $f: \bN \to Y, 
n \mapsto 2n$ is a bijective function. 
\end{example}

From the example above, we find that two sets having equal cardinality does 
not preclude one of the sets from containing the other. 
In fact, this property can be viewed as the definition of infinite sets, 
which we will see below. 

\begin{prop}
Having equal cardinality is a equivalence relation. 
\end{prop}

\begin{defn}
Let $n$ be a natural number. 
A set $X$ is said to have $n$ elements or have cardinality $n$ if $X$ have 
equal cardinality with the set $\{1, 2, \ldots, n\}$. 
\end{defn}

\begin{prop}[Uniqueness of cardinality]
\label{prop:set_theory:uniqueness_cardinality}
Let $n, m$ be distinct natural numbers. 
If $X$ is a set having cardinality $n$, then $X$ cannot have cardinality $m$. 
\end{prop}

Before we proceed with the proof, we need the following lemma. 
\begin{lemma}
Let $X$ be a set having cardinality $n > 0$. 
Then $X$ is non-empty if $x$ is an element of $X$, then $X \backslash \{x\}$ 
has cardinality $n - 1$. 
\end{lemma}
\begin{proof}[Proof of Proposition \ref{prop:set_theory:uniqueness_cardinality}]
This can be done by inducting on $n$ using lemma above. 
\end{proof}

\begin{defn}[Finite sets]
A set $X$ is \emph{finite} if it has cardinality $n \in \bN$, and we write 
$\#(X)$ for the cardinality of $X$. 
Otherwise, $X$ is said to be infinite. 
\end{defn}

\begin{prop}
The set $\bN$ of all natural numbers is infinite. 
\end{prop}
\begin{proof}
Suppose for the sake of contradiction that $\bN$ has cardinality $n \in \bN$. 
Then there exists a bijective function from $\{1, 2, \ldots, n\}$ to $\bN$. 
However, $f(\{1, 2, \ldots, n\})$ is bounded while $\bN$ is unbounded, which 
is absurd. 
\end{proof}

%%%%%%%%%%%%%%%%%%%%%%%%%%%%%%%%%%%%%%%%%%%%%%%%%%%%%%%
%%  Section: Intergers
%%%%%%%%%%%%%%%%%%%%%%%%%%%%%%%%%%%%%%%%%%%%%%%%%%%%%%%

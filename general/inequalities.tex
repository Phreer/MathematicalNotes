\chapter{Inequality}
\section{Inequality in Integer Field}
\begin{thm}
\label{thm:pow_diff_ineq}
Let $a, b$ be two positive real numbers, $a < b$ and $n$ a 
positive integer, then
\begin{equation}
b^n - a^n < n(b - a)b^{n - 1}.
\end{equation}
\end{thm}

This formula can be easily obtained from Theorem \ref{thm:pow_diff_eq}. 

%%%%%%%%%%%%%%%%%%%%%%%%%%%%%%%%%%%%%%%%%%%%%%%%%%%%%%%
%%  Section: Cauchy Schwarz's Inequality
%%%%%%%%%%%%%%%%%%%%%%%%%%%%%%%%%%%%%%%%%%%%%%%%%%%%%%%
\section{Cauchy Schwarz's Inequality}
\begin{thm}[Cauchy Schwarz's Inequality]
\index{inequality!Cauchy Schwarz's \~{}}
Let $u = \left(u_1, u_2, \ldots, u_n \right), v = \left(v_1, v_2, \ldots, 
v_n \right)$ be elements of $\bR^n$, then 
\begin{equation}
    \label{equ:inequality:cauchy_schwarz_inequality_rn}
    \left(\sum _{i=1}^{n}u_{i}v_{i}\right)^{2} 
    \leq \left(\sum _{i=1}^{n}u_{i}^{2}\right) 
         \left(\sum _{i=1}^{n}v_{i}^{2}\right).
\end{equation}
\end{thm}
\begin{proof}
Consider the following quadratic polynomial in $x$: 
\begin{equation*}
    0\leq (u_{1}x+v_{1})^{2}+\cdots +(u_{n}x+v_{n})^{2} 
    =\left(\sum_{i=1}^n u_{i}^{2}\right)x^{2} 
        + 2\left(\sum_{i=1}^n u_{i}v_{i}\right)x + \sum v_{i}^{2}.
\end{equation*}
Since it is nonnegative, it has at most one real root, whence its 
discriminant satisfying 
\begin{equation*}
    \Delta = 4 \left( \sum_{i=1}^{n} u_i v_i \right)^2 
        - 4 \left( \sum_{i=1}^{n} u_i^2 \right) \left( \sum_{i=1}^{n} v_i^2 \right) 
    \le 0, 
\end{equation*}
which is exactly the inequality we want. 
\end{proof}

\section{Hölder's Inequality}
\begin{lemma}[Young's Inequality]
Let $p, q$ be positive real numbers such that $\frac{1}{p} + \frac{1}{q}
= 1$. 
Then for all $a, b \in \bC$, 
\begin{equation}
    \abs{ab} \le \frac{\abs{a}^p}{p} + \frac{\abs{b}^q}{q}. 
\end{equation}
\end{lemma}
\begin{proof}
Define $\phi(t) = \frac{t^p}{p} + \frac{\abs{b}^q}{q} - \abs{b} t, t \ge 0$. 
It suffices to show that $\phi$ is non negative. 
Observing that 
\begin{equation*}
    \phi'(t) = t^{p-1} - \abs{b}
\end{equation*}
is negative when $t < \abs{b}^\frac{1}{p-1}$ and positive when 
$t > \abs{b}^\frac{1}{p-1}$, thus $\phi$ achieves its minium at 
$\abs{b}^\frac{1}{p-1}$, which is 
\begin{equation*}
    \frac{1}{p} \abs{b}^\frac{p}{p - 1} 
        - \abs{b}^{1+ \frac{1}{p - 1}} 
        + \frac{1}{q} \abs{b}^q 
    = \frac{1}{p} \abs{b} ^q + \frac{1}{q} \abs{b}^q - \abs{b}^q 
    = 0, 
\end{equation*}
which is what we want.
\end{proof}

\begin{thm}[Hölder's inequality]
Let $(X, \Sigma, \mu)$ be a measure space and let $p, q \in [1, \infty]$ 
with $\frac{1}{p} + \frac{1}{q} = 1$. 
Then for all measurable real- or complex-valued function $f$ and $g$ on $X$, 
\begin{equation*}
    \label{equ:inequalities:holder_inequality}
    \norm{fg} \le \norm{f}_p \norm{g}_q, 
\end{equation*}
where norm $\norm{\wdot}_p$ is defined in Chapter \ref{chp:lp_spaces} and 
Chapter \ref{chp:banach_spaces}.  
\end{thm}
\begin{proof}
If $\norm{f}_p = 0$, then $f$ vanishes almost anywhere thus $fg$ does too. 
Hence Hölder's inequality holds true. 
The same is true for $\norm{g}_q = 0$. 
Suppose that $\norm{f}_p \neq 0$ and $\norm{g}_q \neq 0$ in the following. 

\textbf{Case 1. $p \in (1, \infty)$. }
Consider 
\begin{equation*}
    a = \frac{\abs{f(t)}}{\norm{f}_p}, \quad 
    b = \frac{\abs{g(t)}}{\norm{g}_q}. 
\end{equation*}
By Young's inequality, we have 
\begin{equation*}
    \frac{\abs{f(t)} \abs{g(t)}}{\norm{f}_p \norm{g}_q} 
    \le \frac{1}{p} \frac{\abs{f(t)}^p}{\norm{f}_p^p} 
        + \frac{1}{q} \frac{\abs{g(t)}^q}{\norm{g}_q^q}. 
\end{equation*}
Integrating both sides of the above inequality yields 
\begin{equation*}
    \frac{1}{\norm{f}_p \norm{g}_q} \int_{X} \abs{f g} \diff t 
    \le \frac{1}{p \norm{f}_p^p} \int_{X} \abs{f}^p \diff p 
        + \frac{1}{q \norm{g}_q^q} \int_{X} \abs{g}^q \diff q, 
\end{equation*}
which is 
\begin{equation*}
    \frac{\norm{fg}_1}{\norm{f}_p \norm{g}_q} \le 1. 
\end{equation*}

\textbf{Case 2. $p = 1$ and $q = \infty$. }
Inequality (\ref{equ:inequalities:holder_inequality}) follows directly from 
the monotonicity of Lebesgue integration. 
\end{proof}

\section{Minkowski's Inequality}
\begin{thm}
\label{thm:inequality:minkowski_inequality}
\index{inequality!Minkowski's ~{}}
\index{Minkowski's inequality}
Let $(X, \Sigma, \mu)$ be a measure space and let $p \in [1, \infty]$. 
Then for all measurable real- or complex-valued function $f$ and $g$ on $X$, 
\begin{equation*}
    \norm{f + g}_p \le \norm{f}_p + \norm{g}_p. 
\end{equation*}
\end{thm}
\begin{proof}
The result is clearly true if $p = 1$. 

Assume that $p \in (1, \infty)$, then let $q > 0$ such that $\frac{1}{p} + 
\frac{1}{q} = 1$. 
Observing that 
\begin{equation*}
    \begin{aligned}
        \norm{f + g}_p^p &= \int_X \abs{f + g} \abs{f + g}^{p - 1} \diff t \\ 
        &\le \int_X \left( \abs{f} + \abs{g} \right) \abs{f + g}^{p - 1} \diff t \\
        &= \int_X \abs{f} \abs{f + g}^{p - 1} \diff t 
            + \int_X \abs{g} \abs{f + g}^{p - 1} \diff t \\ 
        &\le \norm{f \abs{f + g}^{p-1}}_1 + \norm{g \abs{f + g}^{p-1}}_1 \\ 
        \text{(Hölder's inequality)} &\le \norm{f}_p \norm{\abs{f + g}^{p-1}}_q 
            + \norm{g}_p \norm{\abs{f + g}^{p-1}}_q 
    \end{aligned}
\end{equation*}
Since $(p - 1)q = p$, it follows that 
\begin{equation*}
    \norm{f + g}_p^p \le \left( \norm{f}_p + \norm{g}_p \right) 
        \norm{f + g}_p^\frac{p}{q}, 
\end{equation*}
whence 
\begin{equation*}
    \norm{f + g}_p^{p - \frac{p}{q}} 
    = \norm{f _ g}_p \le \norm{f}_p + \norm{g}_p. 
\end{equation*}

\end{proof}
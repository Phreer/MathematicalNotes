\chapter{Topological Spaces and Continuous Functions}
\label{chp:topological_spaces}
\section{Topological Spaces}
\begin{defn}
\index{topology}
\index{topological space}
\index{space!topological \~{}}
\index{open set}
A \emph{topology} on a set $X$ is a collection $\mathcal{T}$ of subsets of 
$X$ such that 
\begin{enumerate}
    \item $\varnothing$ and $X$ are in $\mathcal{T}$; 
    \item the union of the elements of any subcollection of $\mathcal{T}$ is 
    contained in $\mathcal{T}$; 
    \item the intersection of elements of any finite subcollection of 
    $\mathcal{T}$ is contained in $\mathcal{T}$. 
\end{enumerate} 
A set $X$ on which a topology $\mathcal{T}$ is specified is called a 
topological space, denoted by $(X, \mathcal{T})$. 
An element of $\mathcal{T}$ is called an open set of topological space $(X, 
\mathcal{T})$. 
\end{defn}

Sometimes, we denote topological space $(X, \mathcal{T})$ by $X$ for 
convenience if the topology is not emphasized. 

Open set in topological space plays an essential role in the study of 
topology. 
\index{neighborhood}
By a neighborhood of $x$ in a topological space $X$ we mean a open set 
containing $x$. 


%%%%%%%%%%%%%%%%%%%%%%%%%%%%%%%%%%%%%%%%%%%%%%%%%%%%%%%
%%  Construction of Topological Spaces
%%%%%%%%%%%%%%%%%%%%%%%%%%%%%%%%%%%%%%%%%%%%%%%%%%%%%%%
\section{Construction of Topological Spaces}
Sometimes we are given a collection $\mathcal{B}$ of subsets of a set $X$ 
and we want to construct a topological space that contains $\mathcal{B}$. 
Here we restrict collection $\mathcal{B}$ with some requirements. 
Or often it is too tremendous for a topology for us to specify that we hope 
to characterize a topology by some ``basic elements''. 
\begin{defn}[Basis for a topology]
Let $X$ be a set.A collection $\mathcal{B}$ of subsets of $X$ (called basis 
elements) is called a basis for a topology if 
\begin{enumerate}
    \item for each $x \in X$, there is at least one basis element $B$ such 
    that $x \in B$; 
    \item if $x \in B_1 \cap B_2$ for some $B_1, B_2 \in \mathcal{B}$, then 
    there exists a basis elements $B_3$ containing $x$ such that $B_3 \subseteq 
    B_1 \cap B_2$. 
\end{enumerate}
If $\mathcal{B}$ obeys these two conditions, then we define a topology 
$\mathcal{T}$ generated by $\mathcal{B}$ as follows: A subset $U$ of $X$ is 
an element of $\mathcal{T}$ if for any $x \in U$, there is a basis element 
$B$ such that $x \in B$ and $B \subseteq U$. 
Obviously, $\mathcal{B} \subseteq \mathcal{T}$. 
\end{defn}

A typical and useful topological space is the metric space, induced by 
distance function. 
More concepts in topological spaces can be successively applied to metric 
spaces, see Chapter \ref{chp:metric_spaces}. 

\begin{thm}
Let $(X, \cT)$ be a topological space. 
If $\cB$ is a basis of $\cT$, then $\cT$ is the class of unions of any 
elements of $X$, \ie, 
\begin{equation*}
    \cT = \left\{ 
        \bigcup_{B \in \cB'} B: \cB' \subset \cB
    \right\}. 
\end{equation*}
\end{thm}
\begin{prop}
For any subset $\cB' \subseteq \cB$, every element of $\cB'$ belongs to 
$\cT$, and therefore the intersection of all elements of $\cB'$ belongs to 
$\cT$. 
Conversely, suppose $U \in \cT$. 
Then for any $x \in U$, there exists a $B_x \in \cB$ such that $x \in B_x 
\subseteq U$. 
Consequently, $U = \bigcup_{x \in U} B_x$. 
\end{prop}

The preceding theorem tells us that given a basis, one can construct a 
topology from it by taking all intersections of subsets. 
\begin{defn}[Subbasis for a topology]
\index{subbasis for a topology}
A \emph{subbasis} $\cS$ of a topological space $X$ is a class of subsets in 
$X$ if the union of all elements of $\cS$ is $X$. 
The topology $\cT$ generated by $\cS$ is the class of all unions of any 
finite intersections of the elements of $\cS$, \ie, 
\begin{equation*}
    \cT = \left( 
        \bigcup_{B \in \cB'} B: \cB' \subset \cB
     \right), 
\end{equation*}
where $\cB$ is the basis generated by $\cS$ in the following way: 
\begin{equation}
    \label{equ:topological_spaces:basis_generated_by_subbasis}
    \cB = \{ \bigcap_{k=1}^n 
    S_k: n \in \bNs; S_k \in \cS, \text{ for all } k = 1, 2, \ldots, n \}.
\end{equation}
\end{defn}

One can verify without much effort that $\cB$ defined in Equation 
(\ref{equ:topological_spaces:basis_generated_by_subbasis}) is exactly a 
basis. 
%%%%%%%%%%%%%%%%%%%%%%%%%%%%%%%%%%%%%%%%%%%%%%%%%%%%%%%
%%  Closed Sets and Limit Points
%%%%%%%%%%%%%%%%%%%%%%%%%%%%%%%%%%%%%%%%%%%%%%%%%%%%%%%
\section{Closed Sets and Limit Points}
\begin{defn}
A subset $A$ of a topological space $X$ is said to be closed if $X^C$ is 
open. 
\end{defn}

\begin{thm}
Let $X$ be a topological space. Then 
\begin{enumerate}
    \item $\varnothing$ and $X$ are closed; 
    \item arbitrary intersections of closed sets are closed; 
    \item finite unions of closed sets are closed. 
\end{enumerate}
\end{thm}
\begin{proof}
These are direct results from definition of closed set and DeMorgan's law. 
\end{proof}

\begin{defn}
\index{interior}
\index{interior!\~{} point}
\index{closure}
Let $X$ be a topological space and $A$ is a subset of $X$. 
The interior of $A$, denoted by $\mathring{A}$, is defined as the 
intersection of all open sets contained in $A$. 
The elements of $\mathring{A}$ are called interior points of $A$. 
The closure of $A$, denoted by $\overline{A}$, is defined as the union of all 
closed sets containing $A$. 
\end{defn}

A set in a topological space is open if and only if all the points of $A$ 
are interior points, \ie, $A = \mathring{A}$. 
Clearly, the interior of a set $A$ is the maximal open set contained in $A$ 
and the closure of $A$ is the mimimal closed set containing $A$. Thus the 
following property holds: 
\begin{equation*}
    \mathring{A} \subseteq A \subseteq \overline{A}. 
\end{equation*}
If $A$ is closed, then $A = \overline{A}$; if $A$ is open, then 
$A = \mathring{A}$. 
The interior of $A$ can also be denoted as $\interior A$ and the closure 
$\closure A$. 

Here are two ways to describe the closure of a set. 
\begin{thm}
\label{thm:topological_spaces:closure}
Let $A$ be a subset of a topological space $X$. Then 
\begin{enumerate}
    \item \label{thm:topological_spaces:closure:1}
    $\closure(A)^C = \interior(A^C)$. 
    \item \label{thm:topological_spaces:closure:2}
    $x \in \overline{A}$ if and only if every neighborhood of $x$ 
    intersects $A$. 
    \item \label{thm:topological_spaces:closure:3}
    Supposing the topology of $X$ is given by a basis, then $x \in 
    \overline{A}$ if and only if every basis element $B$ containing $x$ 
    intersects $A$. 
\end{enumerate} 
\end{thm}
\begin{proof}
\ref{thm:topological_spaces:closure:1} 
$x \in \overline{A}$ 
$\Leftrightarrow$ 
for any closed set $C$ containing $A$, we have $x \in C$ 
$\Leftrightarrow$ 
for any open set $O$ such that $O \cap A = \varnothing$, we have $x \notin O$ 
$\Leftrightarrow$ 
for any open set $O \subseteq A^C$, we have $x \notin O$ 
$\Leftrightarrow$
$x \notin \interior(A^C)$. 

Here is an alternative proof. 
$x \in \closure(A)^C$ 
$\Leftrightarrow$ 
there exists a closed set $C$ such that $A \subseteq C$ and $x \notin C$ 
$\Leftrightarrow$ 
there exists an open set $O$ such that $O \cap A = \varnothing$ and $x \in O$ 
$\Leftrightarrow$ 
there exists an open set $O \subseteq \interior(A^C)$ such that 
$x \in O$ 
$\Leftrightarrow$ 
$x \in \interior(A^C)$.

\ref{thm:topological_spaces:closure:2} 
It would be convenient to prove the following equivalent problem statement:
$x \notin \overline{A} \Leftrightarrow$ there exists a neighborhood $U$ of $x$ 
such that $U \cap A = \varnothing$. 

A direct proof is as follows. 
$x \in \overline{A}$
$\Leftrightarrow$ 
for any closed set $C \supset A$, $x \in C$ 
$\Leftrightarrow$ 
for any closed set $C$ such that $x \notin C$, $A \cap C^C \neq \varnothing$ 
$\Leftrightarrow$ 
for any neighborhood $U$ of $x$, $A \cap U \neq \varnothing$. 
\end{proof}

With closed sets, we can now talk about limit points. 
\begin{defn}
\index{limit point}
\index{cluster point}
\index{point!limit \~{}}
\index{point!cluster \~{}}
\index{point!\~{} of accumulation}
Let $A$ be a subset of a topological space $bX$. 
We say $x \in X$ is a limit point (or cluster point, or point of 
accumulation) of $A$ if every neighborhood of $x$ intersets $A$ in some 
point other than $x$ itself. 
We denote the set of all limit points of $A$ by $A'$. 
In other words, $x$ is a limit point of $A$ if $x$ is a element of the 
closure of $A\\{x}$. 
\end{defn}

Note that a limit point of $A$ needs not to be an element of $A$ and the 
criterion of limit point requires a point other than $x$. 

\begin{prop}
Let $X$ be a topological space. If $A$ and $B$ are subsets of $X$, then 
\begin{equation*}
    \overline{A \cup B} = \overline{A} \cup \overline{B}. 
\end{equation*}
\end{prop}
\begin{proof}
If $x \in \overline{A \cup B}$, then any neighborhood $V$ of $x$ intersects 
$A$ or $B$. 
Assume that $x \notin \overline{A}$ and $x \notin \overline{B}$. 
Then there exist neighborhoods $V_1$ and $V_2$ such that $V_1 \cap A = 
\varnothing$  and $V_2 \cap B = \varnothing$. 
Therefore, the neighborhood $V_1 \cap V_2$ does not intersect $A \cup B$, 
is a contradiction. 
Thus $\overline{A \cup B} \subseteq \overline{A} \cup \overline{B}$. 

On the other hand, since $A \subseteq A \cup B$, we have $\overline{A} \subseteq 
\overline{A \cup B}$. 
Likewise, $\overline{B} \subseteq \overline{A \cup B}$. 
Hence, $\overline{A} \cup \overline{B} \subseteq \overline{A \cup B}$. 
\end{proof}

\begin{thm}
Let $A$ be a subset of a topological space $X$. Then 
\begin{equation*}
    \overline{A} = A \cup A'. 
\end{equation*}
\begin{proof}
Supposing $x \in \overline{A}$ and $x \notin A$. 
According to Theorem \ref{thm:topological_spaces:closure}, every 
neighborhood of $x$ intersects $A$. 
Thus $x \in A'$. 

Conversely, if $x \in A'$, then the intersection of any neighborhood of $x$ 
and $A$ contains a point other than $x$ itself. 
Thus, $x \in \overline{A}$. 
\end{proof}
\end{thm}

From the Theorem above, we have a very important characterization of closed 
sets, which is often used as an alternative definition of closed set. 
\begin{cor}
A subset of a topological space is closed if and only if it contains all its 
limit points. 
\end{cor}
\begin{proof}
$A$ is closed $\Leftrightarrow$ $A = \overline{A}$ 
$\Leftrightarrow$ $A' \subseteq A$. 
\end{proof}

\subsection{Hausdorff Spaces}
\begin{defn}
\index{converge}
Let $\{x_1, x_2, \ldots\}$ be a sequence in a topological space. 
If for any neighborhood of $x \in X$, there is a positive integer $N$ such 
that for any $n > N$, $x_n \in U$, then $\{x_1, x_2, \ldots\}$ is said to 
converge to $x$. 
In this case, we say $\{x_1, x_2, \ldots\}$ is convergent and $x$ is a limit 
of $\{x_1, x_2, \ldots\}$. 
\end{defn}
In the case of metric space, there is a well-known conclusion: 
if a sequence is convergent, then it converges to a unique point. 
However, for general topological spaces, this is not true. 
For instance, we consider the tree points set $X = \{a, b, c\}$ equipped 
with the topology $\mathcal{T} = \left\{\{a, b, c\}, \{a, b\}, \{b, c\}, 
\{b\}, \varnothing \right\}$. 
Then the sequence $x_n = b$ converges to $a$, $b$ and $c$!
What's more, in this space, not all of the sets of a single point are 
closed. 

This kind of topological spaces is abnormal and less important. 
Therefore, Hausdorff gave a condition to make limit unique, which is 
known as Hausdorff's axiom. 
\begin{defn}
\index{Hausdorff space}
\index{space!Hausdorff \~{}}
Let $X$ be a topological space. 
$X$ is said to be a Hausdorff space if for any two distinct points $x_1$ and 
$x_2$, there exist neighborhoods $U_1$ and $U_2$ of $x_1$ and $x_2$, 
respectively, that are disjoint. 
\end{defn}

\begin{thm}
Every finite set in a Hausdorff space $X$ is closed. 
\end{thm}
\begin{proof}
Let $A = \{x_1, x_2, \ldots, x_n\}$ be a finite set in $X$ and $x \notin A$. 
Then for all $m = 1, 2, \ldots, n$, there exists neighborhood $U_m$ of $x$ 
such that $x_m \notin U_m$. 
Thus the union $\bigcap_{m=1}^n U_m$ is a subset of $A^C$, whence 
$A^C$ is open and $A$ is closed. 
\end{proof}

The converse, \ie, a topological space whose finite subsets are closed is a 
Hausdorff space, is not true. 
In fact, this condition every finite subset is closed is named $T_1$ axiom 
and topological spaces satisfying $T_1$ axiom are named $T_1$ spaces. 
But we have little interst in $T_1$ spaces here, since only Hausdorff gives 
us the following property: 
\begin{thm}
If $\{x_1, x_2, \ldots\}$ be a sequence in a topological space $X$, then 
$\{x_1, x_2, \ldots\}$ converges to a most one point. 
\end{thm}
\begin{proof}
Supposing $\{x_1, x_2, \ldots\}$ converges to two distinct points $x$ and 
$x'$, then there exist two neighborhoods $U_1$ and $U_2$ of $x$ and $x'$, 
respectively, that are disjoint. 
This means that there exist two positive intergers $N_1$ and $N_2$ such that 
for any $n > N_1$, $x_n \in U_1$ and for any $n > N_2$, $x_n \in U_2$, which 
is impossible. 
\end{proof}

For Hausdorff spaces, we write $\lim_{n \to \infty} x_n = x$ if the sequence 
$\{x_1, x_2, \ldots\}$ converges to $x$. 

%%%%%%%%%%%%%%%%%%%%%%%%%%%%%%%%%%%%%%%%%%%%%%%%%%%%%%%
%%  Continuous Functions
%%%%%%%%%%%%%%%%%%%%%%%%%%%%%%%%%%%%%%%%%%%%%%%%%%%%%%%
\section{Continuous Functions}
\begin{defn}
\label{def:topology:continuous_function}
\index{continuous function}
\index{function!continuous \~{}}
Let $X$ and $Y$ be topological spaces. A function $f: X \to Y$ is said to be 
continuous if for each open subset $V$ of $Y$, the set $f^{-1}(V)$ is an 
open subset of $X$. 
\end{defn}

\begin{thm}
Let $X$ and $Y$ be topological spaces. If $f: X \to Y$, then the following 
statements are equivalent 
\begin{enumerate}
    \item $f$ is continuous. 
    \item for every subset $A$ of $X$, $f(\overline{A}) \subseteq 
    \overline{f(A)}$. 
    \item for every closed subset $B$ of $Y$, $f^{-1}(B)$ is a closed subset 
    in $X$. 
    \item for every $x \in X$ and neighborhood $V$ of $f(x)$, there exists 
    a neighborhood $U$ of $x$ such that $f(U) \subseteq V$. 
\end{enumerate}
\end{thm}

\begin{defn}
\index{homeomorphism}
Let $X$ and $Y$ be topological spaces. A one to one function $f$ that maps 
$X$ onto $Y$ is called a homeomorphism if both $f$ and $f^{-1}$ are 
continuous. 
\end{defn}

%%%%%%%%%%%%%%%%%%%%%%%%%%%%%%%%%%%%%%%%%%%%%%%%%%%%%%%
%%  Section: Metric topology
%%%%%%%%%%%%%%%%%%%%%%%%%%%%%%%%%%%%%%%%%%%%%%%%%%%%%%%
\section{Metric Topology}
Metric spaces is a very important kind of spaces in the study of analysis. 
Facts that matter in analysis are presented in Chapter 
\ref{chp:metric_spaces}. 
Here we mainly focus on the topological properties of metric spaces. 

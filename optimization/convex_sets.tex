% chapter2.tex
% notes of the Convex Optimization, chapter 2
\chapter{Convex Sets}
\section{Cones}
\begin{defn}
A set $C$ is a cone if for every $x \in C$ and $\theta \geq 0$, $\theta 
x \in C$ always holds.
\end{defn}
Cone is also named \emph{nonnegative homogeneous}. It is easy to prove 
that a cone is closed.
\section{Dual cones}
\subsection{Dual cones}
\begin{defn}
  Let $K$ be a cone. The set is called dual cone of $K$ if
  \begin{equation*}
    K^*=\{y|y^{\text{T}}x \geq 0 \text{ for all } x \in K\}.
  \end{equation*}
\end{defn}
\subsection{Examples}
\subsubsection{Dual of a norm cone}
The dual cone of $K=\{(x,t) \in \bR^{n+1}|\norm{x} \le t\}$ is

\begin{equation*}
  K^*=\{(u,v) \in \bR^{n+1}|\norm{u} \le v\},
\end{equation*}
where the dual norm is given by $\norm{u}_*=\sup\{u^Tx|\norm{x} \le 1\}$.

\subsubsection{Nonnegative orthant}
The cone $\bR_+^n$ is self-dual.

\subsubsection{Positive semi-definitive matrix}
The set of positive semi-definitive matrix $\mathbf{S}_+^n$ is also 
self-dual.

\subsection{Property of dual cone}
\begin{itemize}
  \item $K^*$ is a cone and convex.
  \item $K$ is closed and convex.
  \item $K_1 \subseteq K_2 \text{ implies } K_2^* \subseteq K_1^* $

\end{itemize}

\section{Generalized inequalities}
\subsection{Proper cones}
\begin{defn}
A cone $K \in \bR^n$ is a proper cone if satisfies the following 
conditions:
  \begin{itemize}
    \item $K$ is convex.
    \item $K$ is closed.
    \item $K$ is solid, i.e., it has nonempty interior.
    \item $K$ is pointed, i.e., it contains no line.
  \end{itemize}
\end{defn}
Then we can define generalized inequalities on $K$, which is a partial 
ordering on $mathbb{R}^n$. We write $y$

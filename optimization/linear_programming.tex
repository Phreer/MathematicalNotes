\chapter{Linear Programming}
\section{Standard Form of Linear Progamming}
The optimization problem like 
\begin{equation}
    \label{equ: std_lp}
    \begin{aligned}
        \text{minimize} \quad & f(x) = c^Tx \\
        \text{subject to} \quad & Ax=b \\
        & x \succeq 0.
    \end{aligned}
\end{equation}
where $x, c \in \br ^n$, $A \in \br^{m \times n}$ and $b \in \br^n$, is 
called the standard form of linear programming. We may always assume that 
$m < n$. Otherwise, redundant rows of $A$ can be eliminated or one/no 
solution obeys $Ax = b$, which can be solved by numerical algebra methods, 
such as LR or QR factorization. 

Without loss of generality, we suppose that $c \ne 0$ and $\rank(A) = m$.
Obviously, the feasible set of \ref{equ: std_lp} $F=\{x \in \br^n : 
Ax = b, x \succeq 0\}$ is a closed convex set. 

\section{Simplex Method}
Simplex method is a prevailing approach to solve linear programming problem. 

Since $A$ has the attribute that $\rank(A) = m$, we can split $A$ to 
\begin{equation*}
    A = \begin{pmatrix}
        B, N
    \end{pmatrix}
\end{equation*}
where $B \in \br^{m \times m}$ is non-singular and 
$N \in \br^{m \times (n - m)}$. 
Correspondingly, $x$, $c$ are split in the same manner that 
\begin{align*}
    x = \begin{pmatrix}
        x_B \\
        x_N
    \end{pmatrix}, 
    c = \begin{pmatrix}
        c_B \\
        c_N
    \end{pmatrix}.
\end{align*}

Thus $Ax=b$ is equivalent to $Bx_B + Nx_N = b$. Owing to non-singularity of 
$B$, we obtain $x_B = B^{-1} (b - N x_N)$. With the knowledge of linear 
algebra, we contend that 
\begin{equation}
    x = \begin{pmatrix}
        x_B \\
        x_N
    \end{pmatrix}
    = \begin{pmatrix}
        B^{-1}(b - N x_N) \\
        x_N
    \end{pmatrix}
\end{equation}
is the general solution to equation $Ax = b$, and $x_N$ is the free 
variable that can be any vector in $\br^{n - m}$, $x_B$ is the basic 
variable. 
The choice of free variable and basic variable is somewhat arbitrary, 
since interchanging the position of variables do no harm to correctness of 
$Ax = b$, as long as $B$ is non-singular. 
With a fixed non-singular $B$, setting $x_N$ to zero, then we have 
\begin{equation}
    x = \begin{pmatrix}
        B^{-1}b \\
        0
    \end{pmatrix}, 
\end{equation}
which we call a basic solution to \ref{equ: std_lp} or a basic 
feasible solution if further $B^{-1}b \succeq 0$. 

\begin{thm}
A point $x \in \br^n$ is a basic feasible solution to \ref{equ: std_lp} 
if and only if it is a vertice feasible polytope $F = \{x \in \br^n: 
Ax = 0, x \succeq 0\}$.
\end{thm}

The basic idea of simplex method is to construct a sequence of vertices 
$\{x^{(k)}\}$ of $F$ (basic feasible solutions) with descending value of 
the target function $f(x) = c^T x$. 
Suppose now we have the $k$th iterative result $x^{(k)}$, then we are 
confronted with the problem that \emph{how to decide whether we have 
obtained an optimal point}. 

Assume that  
\begin{equation*}
    \label{equ: lp_xk}
    x^{(k)} = \begin{pmatrix}
        x_B^{(k)} \\
        x_N^{(k)}
    \end{pmatrix} = 
    \begin{pmatrix}
        B^{(k)-1}b \\
        0
    \end{pmatrix}
\end{equation*}
where $x_B^{(k)}$ is the basic variable and $x_N^{(k)}$ is the free 
variable. Otherwise, it is always reasonable to interchange the order of 
variables and columns of $A$ to adapt it to the form \ref{equ: lp_xk}.
Thus $f(x^{(k)}) = c_B^{(k)T} B^{(k)-1}b$. For any $x \in F$, $x$ can be 
written in the form 
\begin{equation}
    x = \begin{pmatrix}
        x_B \\
        x_N
    \end{pmatrix}
    = \begin{pmatrix}
        B^{-1}(b - Nx_N) \\
        x_N
    \end{pmatrix}.
\end{equation}
Then 
\begin{equation*}
    \begin{aligned}
        f(x) 
        &= c_B^{T} x_B + c_N^T x_N 
        = c_B^T B^{-1}(b - N x_N) + c_N^T x_N \\
        &= f(x^{(k)}) + (c_N^T - c_B^T B^{-1})x_N
    \end{aligned}
\end{equation*}
% chapter 3
% Convex Functions
\chapter{Convex functions}
\section{Convex functions and examples}
\begin{defn}
  A function $f:\bR^n \to \bR$ is convex if 
  $\mathbf{dom} f$ is a convex set and for all $x,y\in \mathbf{dom} 
  f$, and $\theta \geq 0$, the following equation holds:
  \begin{equation}
    \theta f(x)+(1-\theta ) f(y) \leq f(\theta x+(1-\theta) y)
  \end{equation}
\end{defn}
It can be shown that a function is convex if and only if it is 
convex when its domain is restrictecd to a line.
A convex function is contiuous on the relative interior of its 
domain.

\subsection{First-order condition}
\begin{thm}
  Assume $f$ is differentiable, then $f$ is convex if and only if
  $\mathbf{dom}f$ is convex and
  \begin{equation}
    f(y) \geq f(x)+\nabla f(x)^{\text{T}}(y-x)
  \end{equation}
  holds for all $x, y \in \mathbf{dom}f$.
\end{thm}

\subsection{Second-order condition}
\begin{thm}
  Supppose that $f$ is twice differentiable and $\mathbf{dom}f$ 
  is open, this is, its \emph{Hession Matrix} $\nabla^2f(x)$ exists 
  at each point in its domain. Then $f$ is convex if and only if 
  $mathbf{dom}f$ is convex its Hession Matrix is positive 
  semidefinitite. 
\end{thm}
For strict convexity, we have that if $\nabla^2 f(x) \succ 0$ for
all $x \in \mathbf{dom}f$, $f$ is strict convex. However, the
converse is not true.

\subsection{Examples}
Here are some convex functions on $\bR^2$:
\begin{itemize}
  \item Norms.
  \item Max function.
  \item Quadratic-over-linear function, i.e., $f(x,y)=x^2/y$, with 
    \begin{equation*}
      \mathbf{dom}f=\bR \times \mathbb{R_{++}}.
    \end{equation*}
  \item Log-sum-exp function, i.e., $f(\mathbf{x})=\log (e^{x_1}+
  e^{x_2}+\dots + e^{x_n})$. In fact, this function can be viewed
  as a differentiable approximation of the max function, since
    \begin{equation}
      \max (\mathbf{x}) \leq f(\mathbf{x}) \leq \max (\mathbf{x})+
      \log (n)
    \end{equation}
  holds for any $\mathbf{x}$.
  \item Geometric mean, i.e., $f(x)=(\prod_{i=1}^{n}x_i)^{\frac{1}
  {n}}$ with $\mathbf{dom}f=\bR_{++}$ is concave.
  \item Log-determinant, i.e., $f(\mathbf{X})=\log \det \mathbf{X}$
  on $\mathbf{dom}f=\mathbb{S}^n_{++}$ is concave. \emph{The proof 
  of this part is sophisticated and I don't fully understand so I
  should turn back later.}
\end{itemize}

\subsection{Sublevel sets}
\begin{defn}
  The $\alpha$-sublevel set of a function $f:\bR^n 
  \leftarrow \bR$ is
  \begin{equation*}
    C_\alpha=\{\mathbf{x}\in \mathbf{dom}f\ |\ f(\mathbf{x}) <
    \alpha\}.
  \end{equation*}
\end{defn}
Sublevel sets of a convex function are convex, while the converse is 
not true, and this can be used to tell a set is convex by expressing 
it as a sublevel of a convex function.

\subsection{Epigraph}
\begin{defn}
  The epigraph of a function $f:\bR^n \leftarrow
  \bR$ is
  \begin{equation*}
    \mathbf{epi}=\{(\mathbf{x}, t))\ |\ \mathbf{x}\in
    \mathbf{dom}f, f(\mathbf{x}) \leq t\}.
  \end{equation*}
\end{defn}
The relation between convex function and convex set is that \emph{
a function is convex if and only if its epigraph is a convex set}.
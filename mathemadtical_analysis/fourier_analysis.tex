\chapter{Fourier Analysis}
\section{Fejér's Theorem}
In this section, we consider continuous function $f$ on circle. 
\begin{defn}
Suppose $\sum c_n$ is a complex valued series and $s_n$ the partial sum 
of $c_n$. The Cesàro mean of sequence $\{s_n\}$ is defined as 
\begin{equation}
\sigma_N = \frac{s_1 + s_2 + \ldots + s_N}{N}.
\end{equation}
\end{defn}
\begin{defn}
The Fejér sum of $f$ is the Cesàro mean of the Fourier series of $f$, \ie, 
\begin{equation}
\sigma_N (f)(x) = \frac{S_0(f)(x) + S_1(f)(x) + \ldots + S_{N-1}(f)(x)}
{N}.
\end{equation}
\end{defn}

Fejér's theorem show us the completeness of trigonometric polynomial baasis.

\begin{thm}
Suppose $f$ is a continuous function on circle, then the Fejér sum of $f$ 
converges uniformly to $f$.
\end{thm}
\begin{rmk}
This theorem is stronger than the Wierstrass's theorem; the latter one 
only prove the existence of trigonometric sequence that converges uniformly 
to any continuous function but the former one exactly specify such a
sequence.
\end{rmk}
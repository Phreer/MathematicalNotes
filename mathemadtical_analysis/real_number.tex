\chapter{Real Number Theory}
\section{Field}
Since rational numbers constitute a set showing some arithmetic 
properties, we abstract these properties as the definition of the 
so-called field.

\begin{defn}
A field is a set $F$ with two operations, called addition and 
multiplication, which satisfy the following field axioms:
\begin{description}
    \item[A] \emph{Axioms for addition}
    \begin{enumerate}
        \item \emph{Close}. If $x, y \in F$, then $x + y \in F$;
        \item \emph{Commutativity}. If $x, y \in F$, then $x + y = y + x$;
        \item \emph{Associativity}. If $x, y and z \in F$, then 
        $(x + y) + z = x + (y + z)$.
        \item There is an element $0$ in $F$, named zero element, which 
        satisfies that $x + 0 = x$ for all $x \in F$;
        \item For any $x \in F$, there is corresponding element $-x$, named 
        negative element, which satisfies that $x + (-x) = 0$.
    \end{enumerate}
\end{description}
\begin{description}
    \item[M] \emph{Axioms for multiplication}
    \begin{enumerate}
        \item \emph{Close}. If $x, y \in F$, then $xy \in F$;
        \item \emph{Commutativity}. If $x, y \in F$, then $xy = yx$;
        \item \emph{Associativity}. If $x, y and z \in F$, then 
        $(xy)z = x(yz)$.
        \item There is an element $1$ in $F$, named unit element, which 
        satisfies that $1x = x$ for all $x \in F$;
        \item For any $x \in F$, there is corresponding element $\frac{1}{x}$, 
        named inverse element, which satisfies that $x\frac{1}{x} = 1$.
    \end{enumerate}
    \item[D] \emph{The distributive law}. If $x, y, z \in F$, then 
    $x * (y + z) = xy + xz$.
\end{description}
\end{defn}

\section{The Construction of Real Field}
The main theory of this chapter is given as follows.

\begin{thm}
There exists an ordered field referred as $\bR$, which has the 
least-upper-bound property and $\bR$ contains $\mathbb{Q}$.
\end{thm}

\begin{proof}
We will complete our proof by constructing such a $\bR$. 
We make $\bR$ contains exactly some subsets of $\mathbb{Q}$, 
called cuts (denoted by Greek letters). If $\alpha$ is a cut in 
$\bR$, then by definition, it has the following tree properties:
\begin{enumerate}
\item $\alpha$ is not empty and $\alpha \neq \mathbb{Q}$; 
\item If $p \in \alpha$, $q \in \mathbb{Q}$ and $q < p$, then
$q \in \alpha$;
\item If $p \in \alpha$, then there exists some $r \in \alpha$ such that
$p < r$.
\end{enumerate}
From the definition of cuts, we can make some observations:
\begin{itemize}
\item Every cut of $\mathbb{Q}$ is bounded above;
\item If $p \in \alpha$ and $p \notin \alpha$, then $p < q$;
\item If $r \notin \alpha$ and $r < s$, then $s \notin \alpha$, \ie, 
there is no maximum element of $\alpha$.
\end{itemize}

In order to make $\bR$ an ordered set, we define $\alpha < \beta$ 
as $\alpha \in \beta$. The addition of $\bR$ is defined by
\begin{equation}
\gamma = \alpha + \beta = \{r + s : r \in \alpha, s \in \beta\}.
\end{equation}
Making $\bR$ fits the axioms of addition, we have to find 
a zero element by letting $0^*$ be the set of all negative number in 
$\mathbb{Q}$. Clearly, $0^*$ is a cut, thus in $\bR$.

It would be harder to construct multiplication for $\bR$. First, 
let us confine us on $\bR^+ = \{\alpha \in \bR : 
\alpha > 0^*\}$. Likewise, define\ $\alpha \beta$ as all the number 
less than $rs$ for some $r\in \alpha$, $s \in \beta$, $r > 0$ and $s > 0$. 
Futhermore, we complete the definition of multiplication by letting 
\begin{equation}
    \alpha \beta = 
\begin{cases}
    (-\alpha)(-\beta) &\text{ if } \alpha < 0^* \text{ and } \beta < 0^*, \\
    -[(-\alpha)\beta] &\text{ if } \alpha < 0^* \text{ and } \beta > 0^*, \\
    -[\alpha(-\beta)] &\text{ if } \alpha > 0^* \text{ and } \beta < 0^*.
\end{cases}
\end{equation}
And we define $1^* = \{ p \in \mathbb{Q} : p < 1\}$.

Finally we define a mapping $\phi: \mathbb{Q} \to \bR$ as 
$\phi(p) = \{r \in \mathbb{Q} : r < p\}$. Obviously, $\phi$ is a 
injection and is a homomorphism, \ie, the following conditions is 
satisfied
\begin{enumerate}
\item $\phi(rs) = \phi(r)\phi(s)$;
\item $\phi(r+s) = \phi(r) + \phi(s)$;
\item $r < s$ if and only if $\phi(r) < \phi(s)$.
\end{enumerate}
In fact, these conditions can be expressed as that $\phi(\mathbb{Q})$ is
isomorphic to $\mathbb{Q}$. So, in concerns of the arithmetic and 
ordering properties they are exactly the same.

Then by proving the following lemmas, we show that such $\bR$ 
satisfies our requirements. The proof is rather tedious and omitted 
right now, may be presented someday.
\end{proof}

\begin{lemma}
$\bR$ is an ordered set and has least-upper-bound property.
\end{lemma}
\begin{proof}
Let $A$ be an nonempty bounded above subset of $\bR$, $\beta$ a 
upper bound of $A$ and $\gamma = \prod_{\alpha \in A} \alpha$. Then 
it is trivial to prove that $\gamma = \sup A$.
\end{proof}

\begin{lemma}
$R$ with addition, zero element, multiplication and unit element 
defined above is a field.
\end{lemma}
\chapter{Metric Spaces}
\label{chp:metric_spaces}
\section{The Induction of Metric Spaces}
\label{sec:metric_space}
\begin{defn}
\index{metric}
\index{distance}
\label{defn:metric_space}
A distance on a set $X$ is a function $d: X \times X \to \bR$ satisfying 
the following conditions:
\begin{enumerate}
\item (definitive positiveness) $d(x, y) > 0  \text{  if } 
x \neq y; d(x, x) = 0$
\item (symmetry) $d(x, y) = d(y, x)$
\item (triangle inequality) $d(x, z) \le d(x, y) + d(y, z)$
\end{enumerate}
for any $x$, $y$ and $z \in X$. We call $d(x, y)$ is the distance of $x$ and 
$y$.
\end{defn}

\begin{prop}
Let $(X, d)$ be a metric space. Then for any $x, x_1, y, y_1 \in X$, 
\begin{equation}
\label{equ:metric_spaces:four_points_inequality}
\abs{d(x, y) - d(x_1, y_1)} \le d(x, x_1) + d(y, y_1). 
\end{equation}
\end{prop}
\begin{proof}
From triangle inequality, we have 
\begin{equation*}
    \begin{aligned}
        \abs{d(x, y) - d(x_1, y_1)} 
        &= \abs{d(x, y) - d(x_1, y) + d(x_1, y) - d(x_1, y_1)} \\
        &\le \abs{d(x, y) - d(x_1, y)} + \abs{d(x_1, y) - d(x_1, y_1)} \\
        &\le d(x, x_1) + d(y, y_1). 
    \end{aligned}
\end{equation*}
\end{proof}

Let $(X, d)$ be a metric space. 
The $\epsilon$-ball centered at $x \in X$ is denoted by 
\begin{equation*}
    B_d(x, \epsilon) = \{y \in X: d(x, y) < \epsilon\}. 
\end{equation*}
With the definition of distance, a topology on $X$ can be induced by $d$ 
with basis being the collection $\{B_d(x, \epsilon): x \in X, 
\epsilon > 0, \epsilon \in \bR\}$. 
\begin{thm}
The collection $\{B_d(x, \epsilon): x \in X, \epsilon > 0, \epsilon \in 
\bR\}$ is a basis for a topology. 
\end{thm}
\begin{proof}
$x \in B_d(x, \epsilon)$ for any $\epsilon > 0$, whence the first 
condition of the definition of basis. 
Before verifying the second condition, we first prove the facts: 
if $y \in B_d(x, \epsilon)$, then there is a basis element $B_d(y, \delta)$ 
such that $B_d(y, \delta) \subseteq B_d(x, \epsilon)$. 
Indeed, setting $\delta < \epsilon - d(x, y)$, then for any $z \in 
B_d(y, \delta)$, $d(x, z) \le d(x, y) + d(y, z) < d(x, y) + \delta 
< \epsilon$. Hence $B_d(y, \delta) \subseteq B_d(x, \epsilon)$. 

Now we start to prove the second condition. 
Let $B_1 = B_d(x_1, \epsilon_1)$ and $B_2 = B_d(x_2, \epsilon_2)$ be two 
basis elements, and $x \in B_1 \cap B_2$. 
Then there exist basis elements $B_d(x, \delta_1)$ and $B_d(x, \delta_2)$ 
such that $B_d(x, \delta_1) \in B_1$ and $B_d(x, \delta_2) \subseteq B_2$ 
respectively. 
Thus basis element $B_d(x, \min(\delta_1, \delta_2)) \in B_1 \cap B_2$ and 
the second condition holds.    
\end{proof}

\begin{defn}
\index{metric topology}
\index{distance space}
\index{metric space}
\index{space!metric \~{}}
We call the topology induced by distance $d$ the metric topology. 

A topological space $(X, \mathcal{T})$ is called a metrizable space if 
$\mathcal{T}$ is induced by a distance $d$ of $X$. 
A  metric space (or distance space) is a metrizable space $X$ together with 
a specific metric $d$ that gives the topology of $X$, denoted by $(X, d)$. 
\end{defn}

With the metric topology, we further introduce the concepts of open/closed 
set, limit point, neighborhood and convergence from topological space. 
Note that the following concepts are consistent with those in topological 
spaces, see Chapter \ref{chp:topological_spaces}. 
\begin{defn}
Let $X$ be a metric space. The points and sets mentioned below 
are understood to be elements and subsets of $X$.
\begin{enumerate}
\item A \emph{neighborhood} of $p$ is set $N_r(p)$ consisting of all points 
in $B(p, r) = \{q \in X : d(p, q) < r\}$.
\item A point $p$ is a \emph{limit point} of a set $E$ if every neighborhood 
of $p$ 
consists a point $q \in E$ such that $q \neq p$.
\item A set $E$ is \emph{closed} if every limit point of $E$ is a point 
of $E$.
\item A point $p$ is an \emph{interior point} of a set $E$ there exists a 
neighborhood of $p$ contained in $E$.
\item A set $E$ is \emph{open} if every point of $E$ is an interior point 
of $E$.
\end{enumerate}
\end{defn}

Every metric space is a Hausdorff space. 
Indeed, supposing $x_1$ and $x_2$ are two distinct points of a metric space 
$(X, d)$, then the neighborhoods $B_d(x, \epsilon)$ and $B_d(y, \epsilon)$ 
are disjoint with $\epsilon = \frac{d(x, y)}{2}$ by triangle inequality. 
Therefore, convergence in metric spaces makes sense.
\begin{defn}
Let $(X, d)$ be a metric space. A sequence $\{x_1, x_2, \ldots\}$ of $X$ is said 
to converge to $x \in X$ if $lim_{n \to \infty} d(x_n, x) = 0$. 
In this case, we say that sequence $\{x_1, x_2, \ldots\}$ is convergent 
and $x$ is called a limit of $\{x_1, x_2, \ldots\}$. 
\end{defn}

We have some basic facts about convergence in metric spaces. 
\begin{prop}
\label{prop:metric_spaces:basic_facts_convergence}
Suppose that sequences $\{x_1, x_2, \ldots\}$ and $\{y_1, y_2, \ldots\}$ 
of a metric space $(X, d)$ converges to $x, y \in X$ respectively. Then
\begin{enumerate}
    \item \label{prop:metric_spaces:basic_facts_convergence:1}
    the limit of $\{x_1, x_2, \ldots\}$ is unique. 
    \item \label{prop:metric_spaces:basic_facts_convergence:2}
    any subsequence of $\{x_1, x_2, \ldots\}$ converges to $x$. 
    \item \label{prop:metric_spaces:basic_facts_convergence:3}
    $\lim_{n \to \infty} d(x_n, y_n) = d(x, y)$. 
\end{enumerate}
\end{prop}
\begin{proof}
The proofs of \ref{prop:metric_spaces:basic_facts_convergence:1} 
and \ref{prop:metric_spaces:basic_facts_convergence:2} is trivial 
and have been omitted. 
\ref{prop:metric_spaces:basic_facts_convergence:3} is straitforward result 
of Equation \ref{equ:metric_spaces:four_points_inequality}. 
\end{proof}

% Tree examples of metric space. 
\begin{example}
The sets $\bR^d$ and $\bC^d$ equipped with the Euclidean distance $d(x, y) = 
\sqrt{\sum_{i=1}^d(\xi_i - \zeta_i)^2}$ for $x = (\xi_1, \xi_2, \ldots, 
\xi_d)$ and $y = (\zeta_1, \zeta_2, \ldots, \zeta_d)$ is a metric space and 
convergence in them is equivalent to convergence by coordinates. 
\end{example}

\begin{example}[Discreate space $D$]
Let $X$ be a nonempty set. 
The distance $d$ is defined as 
\begin{equation*}
    d(x, y) = \begin{cases}
        0, & x = y,  \\
        1, & x \neq y.
    \end{cases}
\end{equation*}
Then $\lim_{n \to \infty} x_n = x$ if and only if there exists a positive 
interger $N$ such that for any $n > N$, $x_n = x$. 
\end{example}

\begin{example}
\label{ex:metric_spaces:induction:cab}
Let $C([a, b])$ be the set of all continuous functions defined on $[a, b]$, 
and the distance defind as $d(f, g) = max_{t \in [a, b]} \abs{f(t) - g(t)}$. 
The convergence in $C([a, b])$ is equivalent to uniform convergence. 
\end{example}  

\begin{example}[The space $S$ of Lebesgue mesurable functions]
Let $S$ be the set of all lebesgue measurable functions on a measurable set 
set $E$ such that $0 < m(E) < \infty$ where $m$ is the Lebesgue measure. 
The distance is $d(f, g) = \int_{E} \frac{\abs{f(t) - g(t)}}
{1 + \abs{f(t) - g(t)}} \diff t$. 
We contend that convergence in space $S$ is equivalent to convergence in 
measure. 

Indeed, Let $\{f_1, f_2, \ldots\}$ be a sequence in $S$ and $lim_{n \to 
\infty} = f$. 
Supposing $0 < \epsilon < 1$, then 
\begin{equation*}
    \begin{aligned}
        \int_E \frac{\abs{f_n - f}}{1 + \abs{f_n - f}} \diff t 
        &\ge \int_{E[\abs{f_n - f} > \epsilon]} 
        \frac{\abs{f_n - f}}{1 + \abs{f_n - f}} \diff t \\ 
        &\ge \int_{E[\abs{f_n - f} > \epsilon]} 
        \frac{\epsilon}{1 + \epsilon} \diff t 
        \ge \frac{\epsilon}{2} m\left( m[\abs{f_n - f} > \epsilon] \right).        
    \end{aligned}
\end{equation*}
Thus $m\left( m[\abs{f_n - f} > \epsilon] \right) 
\le \frac{2d(f_n, f)}{\epsilon} \to 0$ as $n \to \infty$. 

Conversely, suppose that $\{f_1, f_2, \ldots\}$ converges to $f \in S$ such 
in measure. 
Setting $\epsilon > 0$, by assumption, $\lim_{n \to \infty} m\left( 
E[\abs{f_n - f} > \epsilon] \right) = 0$. 
That is, there exists a positive interger $N$ such that for any $n > N$, 
$m\left(E[\abs{f_n - f} > \epsilon] \right) < \epsilon$. 
Thus, for $n > N$, 
\begin{equation*}
    \begin{aligned}
        d(f_n, f) &= \int_E \frac{\abs{f_n - f}}{1 + \abs{f_n - f}} \diff t \\
        &= \int_{E[\abs{f_n - f} > \epsilon]} \frac{\abs{f_n - f}}{1 + \abs{f_n - f}} \diff t + 
        \int_{E[\abs{f_n - f} < \epsilon]} \frac{\abs{f_n - f}}{1 + \abs{f_n - f}} \diff t \\
        &\le \epsilon + \epsilon  m(E)
        = \epsilon (1 + m(E)). 
    \end{aligned}
\end{equation*}
Therefore, $\lim_{n \to \infty} f_n = f$. 
\end{example}

%%%%%%%%%%%%%%%%%%%%%%%%%%%%%%%%%%%%%%%%%%%%%%%%%%%%%%%
%%  Two equivalent definitions of continuous function
%%%%%%%%%%%%%%%%%%%%%%%%%%%%%%%%%%%%%%%%%%%%%%%%%%%%%%%
For metric space, we have two equivalent definitions of continuous function. 
\begin{thm}
Let $(X_1, d_1)$ and $(X_2, d_2)$ be two metric spaces. A function 
$f: X_1 \to X_2$ is continuous if and only if for any $x \in X$ and 
$\epsilon > 0$, there exists a $\delta > 0$ such that for any $y \in X$ 
satisfying $d_1(x, y) < \delta$, 
\begin{equation*}
    d_2(f(x), f(y)) < \epsilon. 
\end{equation*}
\end{thm}
\begin{proof}
$\Leftarrow$
Let $V$ be a open subset $Y$ and $x \in f^{-1}(V)$. 
Then $f(x) \in V$ and there exits a open ball $B_{d_Y}(f(x), \epsilon) 
\in V$. 
Hence, by assumption, there exits a open ball $B_{d_X}(x, \delta)$ such 
that for any $y \in B_{d_X}(x, \delta)$, $y \in B_{d_Y}(f(x), \epsilon) 
\subseteq V$. 
That is, for any $B_{d_X}(x, \epsilon) \subseteq f^{-1}(V)$, whence $f$ is 
continuous. 

$\Rightarrow$ 
Let $f$ be a continuous function and $\epsilon$ be an arbitrary positive 
real number. 
By definition, $f^{-1}(B_{d_Y}(f(x), \epsilon))$ is an open set for each 
$x \in X$. 
Thus there exists open ball $B_{d_X}(x, \delta) \subseteq 
f^{-1}(B_{d_Y}(f(x), \epsilon))$. 
\end{proof}

\begin{thm}
\label{thm:metric_space:continuous_function:2}
Let $X$ and $Y$ be topological spaces and $f$ be a function from $X$ to $Y$. 
If $X$ is a metric space, then $f$ is continuous if and only if for every 
convergent sequence $\{x_1, x_2, \ldots\} \subseteq X$ such that 
$\lim_{n \to \infty} x_n = x$, the sequence $\{f(x_1), f(x_2), \ldots\} 
\subseteq Y$ converges to $f(x)$. 
\end{thm}

Before we start to prove this theorem, we introduce the following lemma:
\begin{lemma}[Sequence Lemma]
\label{thm:metric_spaces:sequence_lemma}
Let $X$ be a topological space and $A \subseteq X$. 
If sequence $\{x_1, x_2, \ldots\} \subseteq X$ converges to $x \in X$, then 
$x \in \overline{A}$. 
If $X$ is a metric space, then the converse is true. 
\end{lemma}
\begin{proof}
The first statement is trivial. 
Now we prove the second statement. 
Suppose that $X$ is a metric space and $x \in \overline{A}$. 
Then for every $n \in \bN^\ast$, the neighborhood $B(x, \frac{1}{n})$ 
intersects $A$. 
Thus, for each $n$, we choose a element from $B(x, \frac{1}{n}) \cap A$ 
as $x_n$. 
Obviously, the sequence $\{x_1, x_2, \ldots\}$ converges to $x$. 
\end{proof}

\begin{proof}[Proof of Theorem \ref{thm:metric_space:continuous_function:2}]
$\Leftarrow$ 
Suppose that $f$ is continuous and sequence $\{x_1, x_2, \ldots\} \subseteq X$ 
converging to $x$. 
Then for every neighborhood $V$ of $f(x)$, $U = f^{-1}(V)$ is a 
neighborhood of $x$. 
Thus there exists $N \in \bN^\ast$ such that for $n > N$, $x_n \in V$. 
That is, for $n > N$, $f(x_n) \in V$, whence $\{f(x_1), f(x_2), \ldots\}$ 
converges to $f(x)$. 

$\Rightarrow$ 
Suppose that the convergent sequence condition is satisfied. 
It suffices to prove $f(\overline{A}) \subseteq \overline{f(A)}$ for any 
$A \subseteq X$. 
Let $y \in f(\overline{A})$. 
There exists $x \in \overline{A}$ such that $f(x) = y$. 
Thus, there exists a sequence $\{x_1, x_2, \ldots\}$ converging to $x$, by 
by Sequence Lemma \ref{lemma:rings:finite_chain_of_principal_ideal_domain}. 
Therefore, $\{f(x_1), f(x_2), \ldots\}$ converges to $f(x) = y$, whence 
$y \in \overline{f(A)}$. 
\end{proof}

%%%%%%%%%%%%%%%%%%%%%%%%%%%%%%%%%%%%%%%%%%%%%%%%%%%%%%%
%%  Isometry
%%%%%%%%%%%%%%%%%%%%%%%%%%%%%%%%%%%%%%%%%%%%%%%%%%%%%%%
As homeomorphism in topological space preserves the topological properties, 
we define isometry in metric space that preserves the metric properties. 
\begin{defn}
\index{isometry}
\index{isometric mapping}
Let $(X_1, d_1)$ and $(X_2, d_2)$ be two metric spaces. 
A function $f: X_1 \to X_2$ is said to be an isometry (or isometric mapping) 
if for any $x_1, x_2 \in X_1$, $d_1(x_1, x_2) = d_2(f(x_1), f(x_2))$. 
If there exists an isometry $f$ maps $X_1$ onto $X_2$, then we say $X_1$ is 
isometric to $X_2$. 
\end{defn}
Clearly, isometry is a equivalent relation and a isometric mapping is 
injective. 

%%%%%%%%%%%%%%%%%%%%%%%%%%%%%%%%%%%%%%%%%%%%%%%%%%%%%%%
%%  Section: Complete Metric Spaces
%%%%%%%%%%%%%%%%%%%%%%%%%%%%%%%%%%%%%%%%%%%%%%%%%%%%%%%
\section{Complete Metric Spaces}
\subsection{Cauchy Sequences}
\begin{defn}
\index{Cauchy sequence}
A sequence $\{x_1, x_2, \ldots\}$ in a metric space $(X, d)$ is a 
\emph{Cauchy sequence} if for any $\epsilon > 0$, there exists an $N \in \bN 
^\ast$ such that 
\begin{equation*}
    d(x_m, x_n) < \epsilon, \text{ for any } n, m \ge N. 
\end{equation*}
\end{defn}

There is a geometry perspective of Cauchy sequences.
\begin{defn}
The diameter of a nonempty set $E$ is defined as 
\begin{equation}
    \diam E = \sup_{p, q \in E} d(p, q).
\end{equation}
\end{defn}
By definition, a sequence $\{p_n\}$ is a Cauchy sequence if and only if 
\begin{equation*}
    \lim _{N\to \infty} \diam E_N = 0, 
\end{equation*}
where $E_N = \{p_i : i \ge N\}$.

\begin{prop}
Every convergent sequence is a Cauchy sequence. 
\end{prop}
\begin{proof}
The proof is trivial. 
\end{proof}

\begin{defn}
\index{complete metric space}
A metric space $(X, d)$ is said to be \emph{complete} if every Cauchy 
sequence in $X$ converges. 
\end{defn}

\begin{thm}
\label{thm:metric_spaces:completeness:convergent_subsequence}
If every Cauchy sequence in a metric space $X$ has an convergent 
subsequence, then $X$ is complete. 
\end{thm}
\begin{proof}
Supposing that $\{ x_n \}_{n=1}^{\infty}$ is a Cauchy sequence in $X$, 
we contend that if $\{ x_n \}_{n=1}^{\infty}$ has subsequence $\{x_{n_k}\}
_{k=1}^\infty$ converging to some $x \in X$, then $\{ x_n \}
_{n=1}^{\infty}$ converges to $x \in X$. 
Indeed, for any $\epsilon > 0$, there exist $N_1, N_2 > 0$ such that 
\begin{align*}
    d(x_n, x_m) < \frac{\epsilon}{2}, \quad & \text{for every } n, m \ge N_1, \\
    d(x_{n_k}, x) < \frac{\epsilon}{2}, \quad & \text{for every } k > N_2. 
\end{align*} 
Therefore, it yields that for given $m > \max\{N_1, N_2\}$, 
\begin{equation*}
    d(x_n, x) \le d(x_n, x_m) + d(x_m, x) < \epsilon, \quad 
    \text{for every } n > \max\{N_1, N_2\}. 
\end{equation*} 
This has completed the proof. 
\end{proof}

%%%%%%%%%%%%%%%%%%%%%%%%%%%%%%%%%%%%%%%%%%%%%%%%%%%%%%%
%%  Subsection: Completion of Metric Spaces
%%%%%%%%%%%%%%%%%%%%%%%%%%%%%%%%%%%%%%%%%%%%%%%%%%%%%%%
\subsection{Completion of Metric Spaces}
There is a very interesting fact about metric space: every metric space can 
is isometric to a dense subspace of a complete. 
\begin{thm}[Completion of Metric Spaces]
For every metric space $(X, d)$, there exists a complete metric space 
$\tilde{X}, \tilde{d}$ such that $(X, d)$ is isometric to a dense subspace in 
$(\tilde{X}, \tilde{d})$. 
The subspace $(\tilde{X}, \tilde{d})$ is unique up to an isometric mapping. 
\end{thm}
\begin{proof}
Let $\tilde{X}$ be the set of all Cauchy sequences in $X$ and the distance 
of $\tilde{x} = \{x_n\}$ and $\tilde{y} = \{y_n\}$ is defined as 
\begin{equation}
    \label{equ:metric_spaces:completion:distance}
    \tilde{d}(\tilde{x}, \tilde{y}) = \lim_{n \to \infty} d(x_n, y_n). 
\end{equation}

\textbf{Step 1.} 
The limit in (\ref{equ:metric_spaces:completion:distance}) exists. 
Assuming that $\epsilon > 0$, by Equation 
(\ref{equ:metric_spaces:four_points_inequality}), we have 
\begin{equation*}
    \abs{(x_n, y_n) - d(x_m, y_m)} \le d(x_n, x_m) + d(y_n, y_m) 
    \le \epsilon. 
\end{equation*}
Thus $\{d(x_n, y_n)\}$ is a Cauchy sequence in $\bN$ and thus has a limit. 
For two Cauchy sequences $\tilde{x} = \{x_n\}$ and $\tilde{y} = \{y_n\}$, 
we regard $\tilde{x}$ and $\tilde{y}$ as the same element of $\tilde{X}$ if 
$d(\tilde{x}, \tilde{y})$. 

\textbf{Step 2.}
The metric $d$ is well-defined. 
That is, for elements $\tilde{x} = \tilde{x}'$ and $\tilde{y} = \tilde{y}'$ 
of $\tilde{X}$, $\lim_{n \to \infty} \tilde{d}(\tilde{x}, \tilde{y}) = 
d(\tilde{x}', \tilde{y}')$. 
Indeed, $\abs{d(x_n, y_n) - d(x_n', y_n')} \le d(x_n, x_n') + d(y_n, y_n') 
\to 0$ when $n \to \infty$.

\textbf{Step 3.}
The metric space $(\tilde{X}, \tilde{d})$ is complete. 
Suppose that $\{\tilde{x}_n\}$ is a Cauchy sequence in $\tilde{X}$. 
For any $n \in \bN^\ast$, there exists $N_n \in \bN^\ast$ such that 
\begin{equation*}
    d(x_{nk} - x_{nk'}) < \frac{1}{n} \quad \text{for } k, k' > N. 
\end{equation*}
Let $y_n = x_{n(N_n+1)}$, and set $\tilde{y} = \{y_n\}$. 
Then $\tilde{y}$ is an element of $\tilde{X}$ and $\lim_{n \to \infty} 
\tilde{x}_n = \tilde{y}$. 
Indeed, for any $m, n \in \bN$ and $k > \max(N_n, N_m)$, 
\begin{equation*}
    \begin{aligned}
        d(y_n, y_m) &= d(x_{n(N_n + 1)}, x_{m(N_m + 1)}) \\
        &< d(x_{n(N_n} + 1), x_{nk}) + d(x_{nk}, x_{mk}) 
        +  d(x_{m(N_m + 1)}, x_k) \\
        &< \frac{1}{n} + d(x_{nk}, x_{mk}) + \frac{1}{m}.
    \end{aligned}
\end{equation*}
Thus $d(y_n, y_m) \to 0$ as $n, m \to \infty$. 
And 
\begin{equation*}
    \begin{aligned}
        \tilde{d}(\tilde{y}, \tilde{x}_n) 
        & \le \tilde{d}(\tilde{y}, \tilde{y}_n) 
        + \tilde{d}(\tilde{y}_n, \tilde{x}_n) \\
        &< \lim_{k \to \infty} d(y_k, y_n)+ \frac{1}{n}. 
    \end{aligned}
\end{equation*}
tends to $0$ as $n \to \infty$. 

\textbf{Step 4.}
$(X, d)$ is isometric to a dense subspace in $(\tX, \td)$. 
Let $G$ be the set of all constant sequences in $\tX$ and define 
\begin{equation*}
    \begin{aligned}
        T: X &\to \tX,  \\
        x &\mapsto \tx = \left\{ x, x, \ldots \right\}. 
    \end{aligned}
\end{equation*}
Then obviously $d(x, y) = \td(\tx, \ty)$, whence $(X, d)$ is isometric to 
$(T(X) = G, \td)$. 
Supposing that $\ty = \left\{ y_n\right\}$ is a Cauchy sequence in $\tX$, 
then $\ty_n = \left\{ y_n, y_n, \ldots \right\}$ is a constant sequence. 
Observing that 
\begin{equation*}
    \td(\ty, \ty_n) = \lim_{m \to \infty} d(y_m, n), 
\end{equation*}
we have $\ty_n \to \ty$ as $n \to \infty$, which means $G = T(X)$ is dense 
in $\tX$. 

\textbf{Step 5.}
Suppose that $(\tX, \td)$ and $(\tX', \td')$ are two complete metric spaces 
and there exist two isometric mappings $T$ and $T'$ such that $T: X \to \tX$ 
and $T': X \to \tX'$ are dense in $\tX$ and $\tX'$ respectively. 
We contend that $(\tX, \td)$ is isometric to $(\tX', \td')$. 
Indeed, for any $\tx \in \tX$, there exists a sequence $\left\{ x_n\right\}$ 
such that $\left\{ Tx_n \right\}$ converges to $\tx$. 
Define $T_1; \tX_1 \to \tX_2, \tx \mapsto \lim_{n \to \infty} T'x_n$. 
For any $\tx$ and $\ty$ of $\tX$, 
\begin{equation*}
    \begin{aligned}
        \td'(T_1\tx, T_1\ty) &= \lim_{n \to \infty} \td'(T'\tx_n, T'\ty_n)
        = \lim_{n \to \infty} d(x_n, y_n) \\ 
        &= \lim_{n \to \infty} \td(Tx_n, Ty_n) 
        = \td(\lim_{n \to \infty}Tx_n, \lim_{n \to \infty}Ty_n) \\ 
        &= \td(\tx, \ty). 
    \end{aligned}
\end{equation*}
Thus $T_1$ is an isometry. 

The conclusion now has been established. 
\end{proof}

%%%%%%%%%%%%%%%%%%%%%%%%%%%%%%%%%%%%%%%%%%%%%%%%%%%%%%%
%%  Section: Compact Sets
%%%%%%%%%%%%%%%%%%%%%%%%%%%%%%%%%%%%%%%%%%%%%%%%%%%%%%%
\section{Compact Sets}
\begin{defn}
\index{open cover}
Let $\{G_\alpha\}$, $\alpha \in I$ be a collection of open subsets of 
metric space $X$ and $E$ a set in $X$. 
Then $\{G_\alpha\}$ is a open cover of $E$ if $E \in \bigcup_{\alpha \in I} 
G_\alpha$. 
\end{defn}


\begin{defn}
\index{compact set}
A subset $K$ in a metric space $X$ is a compact if every open cover of
$K$ contains a finite subcover. 
In other words, for every open cover $\{G_\alpha: \alpha \in I\}$ of $K$ 
there exist a finite subcover $\left\{G_{\alpha_n}: 0 \le n \le N, 
\alpha_n \in I \right\}$ such that 
\begin{equation*}
    A \subseteq \bigcup_{n=1}^N G_{\alpha_n}. 
\end{equation*}
\end{defn}

While the concepts of closed set and open set is relative to the 
metric space we considered, compactness is not. We have the following 
theorem.

%%%%%%%%%%%%%%%%%%%%%%%%%%%%%%%%%%%%%%%%%%%%%%%%%%%%%%%
%%  Basic Properties of Compact Sets
%%%%%%%%%%%%%%%%%%%%%%%%%%%%%%%%%%%%%%%%%%%%%%%%%%%%%%%
\subsection{Basic Properties of Compact Sets}
\begin{thm}
Suppose $K\subseteq Y \subseteq X$. Then $K$ is compact relative to $X$ if 
and only if $K$ is compact relative to $Y$. Here, compact relative to 
set $Y$ means that the open cover is considered in $Y$.
\end{thm}
\begin{proof}
$\Rightarrow$ Suppose $K$ is compact relative to $X$ and $\{U_\alpha\}$ is 
an open cover of $K$ relative to Y, \ie, $U_\alpha$ is open relative to 
$Y$ and $K \subseteq \bigcup_\alpha U_\alpha$. From the previously presented 
theorem, we know for any $U_\alpha$, there exists a corresponding open 
set $V_\alpha \in X$ which makes $U_\alpha = V_\alpha \cap Y$. So we 
can choose a finite open cover from $\{V_\alpha\}$, say $V_1, V_2, \ldots, 
V_n$. Evidently, the corresponding $\{U_1, U_2, \ldots, U_n\}$ where 
$U_i = V_i \cap Y$ is an open cover of $K$.

$\Leftarrow$ Conversely, suppose $K$ is compact relative to $Y$ and 
$\{V_\alpha\}$ is an open cover relative to $X$. Then the collection of 
all $U_\alpha = V_\alpha \cap Y$ is an open cover of $K$. Thus we 
can draw a finite open subcover from $U_\alpha$, say $U_1, U_2, \ldots, 
U_n$, which indicates that $V_1, V_2, \ldots, V_n$ where $U_i = V_i \cap 
Y$ is also a finite open subcover of $K$.
\end{proof}

Compactness is a stronger requirement to a set than the closedness.

\begin{thm}
Compact sets in a metric space are closed and bounded.
\end{thm}
\begin{proof}
Assume $X$ is a metric space and $K$ is a compact set in $X$. In order 
to go to a contradiction, suppose $K$ is not closed, \ie, there exists 
a limit point $p$ of $K$ such that $p \notin K$. 
Then $\left\{B(q, \frac{1}{2} d(q, p)) : q \in k\right\}$ is an open cover 
of $K$. 
Note that $p \notin \bigcup_{p \in K} B\left(p, \frac{1}{2}d(p, q)\right)$. 
Since $K$ is compact, we can draw a finite subcover from 
$\left\{B(q, \frac{1}{2} d(q, p)) : q \in K\right\}$, say 
$\{B(q_i, \frac{1}{2} d(q_i, p)) : q_i \in K, i = 1, 2, \ldots , n\}$. 
Let $\delta = \min_{i} \frac{1}{2}d(p, q_i)$ and $N = B(p, \delta)$. 
Then we obtain $N \cap K = \varnothing$, which is a contradiction to the 
condition that $p$ is a limit point of $K$. 

Since $\left\{ B(q, \frac{1}{2}): q \in K \right\}$ is a open cover of $K$, 
there exists a subcover 
\begin{equation*}
    \left\{ B(q_i, \frac{1}{2}): 0 \le i \le N, q_i \in K\right\} 
\end{equation*}
of $K$, which is obviously bounded. 
\end{proof}

\begin{rmk}
Note the theorem above is quite general, illustaring that compactness is a 
stronger condition than closedness. 
The converse holds true for finite-dimensional metric space. 
However, for infinite-dimensional space, the converse is not true.     
\end{rmk}

\begin{thm}
Closed subset of a compact set is compact.
\end{thm}
\begin{proof}
Suppose $A \subseteq X$ be compact and $B \subseteq A$ be closed. 
Let $G = \left\{ G_\alpha: \alpha \in I\right\}$ be an open cover of $B$. 
As $B$ is compact, $B$ is closed, whence $B^C$ is open. 
Thus $G \cup \left\{ B^C \right\}$ is an open cover of $A$, and then it 
contains a finite subcover $G'$ of $A$. 
Hence $G' \backslash \left\{ B^C\right\} \subseteq G$ is a finite subcover of 
$B$. 
\end{proof}

\begin{thm}
\label{thm:compact_intersect_nonempty}
If $\{K_\alpha\}$ is a collection of compact subsets of a metric space $X$ 
such that the intersection of any finite subcollection of $\{K_\alpha\}$ is 
nonempty, then $\bigcap_{\alpha} K_\alpha$ is nonempty.
\end{thm}
\begin{proof}
Fix a subset $K_1$ from $\{K_\alpha\}$ and put $G_\alpha = K_\alpha ^C$.
With no loss of generality, suppose that for any point $p \in K_1$, there 
exists a subset $K_\alpha$ such that $p \notin K_\alpha$. Then $G_\alpha$
is an open cover of $K_1$; thus we can choose a finite number of indices 
such that $\{G_{\alpha_1}, G_{\alpha_2}, \ldots, G_{\alpha_n}\}$ is an 
open subcover of $K_1$, \ie, $K_1 \subseteq \bigcup_{i = 1}^n G_{\alpha_i}$, 
which means $K_1 \cap (\bigcap _{i = 1}^n K_{\alpha_i})$ is empty, 
contradicting the condition that the intersection of any finite 
subcollection of $K_\alpha$ is nonempty.
\end{proof}

\begin{rmk}
This theorem is the key to prove the so-called nested interval theorem 
when we set the metric space to $\bR$.
\end{rmk}

The next theorem is an extension to the classical nested closed ball 
theorem. 
\begin{thm}
Let $K_n$ is a sequence of compact sets in a metric space $X$ such that 
$K_n \supset K_{n+1}, n = 1, 2, \ldots$. If 
\begin{equation}
    \lim_{n \to \infty} \diam K_n = 0, 
\end{equation}
then $K = \bigcap _{n=1}^\infty K_n$ contains exactly one point.
\end{thm}
\begin{proof}
By Theorem \ref{thm:compact_intersect_nonempty}, $\bigcap _{n=1}^\infty K_n$ 
is nonempty. If $K$ consists of more than one point, then $\diam K
= a > 0$. Since $K \subseteq K_n$ for all positive integer $n$, we have 
$0 < a \le \diam K < \diam K_n$, which is a contradiction to 
the condition.
\end{proof}

\begin{thm}
Let $X$ and $Y$ be two metric spaces and $f: X \to Y$ be a continuous 
mapping. 
If $A \subseteq X$ is compact in $X$, then $f(A)$ is compact in $Y$. 
\end{thm}
\begin{proof}
This is trivial with the definition \ref{def:topology:continuous_function} 
of continuous function. 
\end{proof}

\begin{cor}
Let $A \subseteq X$ be compact and $f: A \to \bR$ be continuous. 
Then $f$ achieves its maximum and minimum on $A$. 
\end{cor}

%%%%%%%%%%%%%%%%%%%%%%%%%%%%%%%%%%%%%%%%%%%%%%%%%%%%%%%
%%  Subsection: Compactness in Metric Spaces
%%%%%%%%%%%%%%%%%%%%%%%%%%%%%%%%%%%%%%%%%%%%%%%%%%%%%%%
\subsection{Compactness in Metric Spaces}
\begin{defn}[sequentially compact]
\index{sequentially compact}
Let $X$ be a metric space. 
A subset $A$ in $X$ is said to be \emph{sequentially compact} if every 
sequence in $A$ has a convergent subsequence. 
\end{defn}

\begin{defn}[$\epsilon$-net, totally bounded]
\index{$\epsilon$-net}
\index{totally bounded}
Let $A, B$ be subsets in $X$ and $\epsilon > 0$. 
If 
\begin{equation*}
    A \subseteq \bigcup_{b \in B} B(b, \epsilon), 
\end{equation*}
then we call $B$ an \emph{$\epsilon$-net} of $A$. 
If for any $\epsilon > 0$, $A$ has a finite $\epsilon$-net, then $A$ is 
said to be \emph{totally bounded}. 
\end{defn}

Clearly, a totally bounded set is bounded, while the converse may not be 
true. 
For instance, consider the open ball $B(0, 2)$ in the infinite-dimensional 
metric space $l^\infty$. 
Let $e_i = \left( 0, 0, \ldots, 1, \ldots \right)$ where the the 
nonzero element appear in the $i$-th entry. 
Then $e_i \in B(0, 1)$ for $i \in \bN^\ast$ and $d(e_i, e_j) = 1$ 
if $i \neq j$. 
Thus for all $\epsilon < 1$, there is no $\epsilon$-net of $A$. 

\begin{thm}
\label{thm:metric_spaces:compact_sets:characterizations}
Let $X$ be a complete metric space and $A \subseteq X$ be closed. 
Then the following are equivalent 
\begin{enumerate}
    \item \label{thm:metric_spaces:compact_sets:characterizations:1}
    $A$ is compact. 
    \item \label{thm:metric_spaces:compact_sets:characterizations:2}
    $A$ is totally bounded. 
    \item \label{thm:metric_spaces:compact_sets:characterizations:3}
    $A$ is sequentially compact. 
\end{enumerate}
\end{thm}
\begin{proof}
\ref{thm:metric_spaces:compact_sets:characterizations:1} $\implies$ 
\ref{thm:metric_spaces:compact_sets:characterizations:2}
Supposing that $A$ is compact and $\epsilon > 0$, then 
\begin{equation*}
    \left\{ B(a, \epsilon): a \in A\right\}
\end{equation*}
is an open cover of $A$, and thus has a finite subcover, which is a 
$\epsilon$-net of $A$. 
It follows that $A$ is totally bounded. 

\ref{thm:metric_spaces:compact_sets:characterizations:1} $\implies$
\ref{thm:metric_spaces:compact_sets:characterizations:3}
\footnote{Requires only compactness of $A$. Compactness is stronger than 
sequentially compactness.}
Supposing $A$ is compact and $\{x_n\}_{n=1}^\infty$ is a sequence in $A$ 
consisting of infinitely many distinct elements of $X$. 
We want to prove that $A$ has a limit point of $\{x_n\}_{n=1}^\infty$. 
Assume to the contrary that every $a \in A$ is not a limit point of 
$\{x_n\}_{n=1}^\infty$. 
Then for every $a \in A$, there exists an $\epsilon_a > 0$ such that 
\begin{equation*}
    B(a, \epsilon_a) \cap \{x_n\}_{n=1}^\infty \text{ is finite}. 
\end{equation*}
Thus $\left\{ B(a, \epsilon_a): a \in A \right\}$ is an open cover of $A$, 
which has no finite subcover, contradicting the assumption. 

\ref{thm:metric_spaces:compact_sets:characterizations:3} $\implies$ 
\ref{thm:metric_spaces:compact_sets:characterizations:2}
Suppose that $A$ is sequentially compact and $\epsilon > 0$. 
Assume that $A$ does not have a finite $\epsilon$-net. 
Choose an arbitrary $x_1 \in A$ and consider $B(x_1, \epsilon)$. 
Then $A \not \subseteq B(x_1, \epsilon)$. 
Choose $x_2 \in A$ such that $x_2 \notin B(x_1, \epsilon)$. 
Repeatedly, we obtain a sequence $\{x_n\}$ in $A$ that has no limit point, 
which is a contradiction. 

\ref{thm:metric_spaces:compact_sets:characterizations:2} $\implies$ 
\ref{thm:metric_spaces:compact_sets:characterizations:1}
\footnote{Requires completeness of $X$ and Closedness of $A$.}
Let $A$ be totally bounded and closed. 
Assume $A$ is not compact. 
Then there exists an open cover $\{G_\alpha: \alpha \in I\}$ that does not 
have a finite subcover of $A$. 
Firstly, as $A$ is totally bounded, there exists a finite $1$-net 
covering $A$, \ie, $A \subseteq \bigcup_{i=1}^n B(x_i, 1)$. 
Thus $A = \bigcup_{i=1}^n \left[ A \cap \overline{B(x_i, 1)} \right]$. 
Hence, there exists some $x_i$ such that $A_1 = A \cap \overline{B(x_i, 1)}$ 
cannot be covered by any finitely many $G_\alpha$. 
Secondly, as $A_1 \subseteq A$ is totally bounded, there exists a finite 
$\frac{1}{2}$-net $\{B(y_i, \frac{1}{2}): i=1, 2, \ldots, n_1\}$ covering 
$A_1$. 
Then $A_1 = \bigcup_{i=1}^n \left[ A_1 \cap \overline{B(y_i, \frac{1}{2})} 
\right]$. 
Likewise, there exists $y_i$ such that 
$A_2 = A_1 \cap \overline{B(y_i, \frac{1}{2})}$ cannot be covered by 
finitely many $G_\alpha$. 
Repeating this process, we get a collection $\left\{ A_i \right\}$ 
such that for all $n \in \bN$ 
\begin{equation*}
    \left\{
    \begin{aligned}
        & A_1 \supset A_2 \supset A_3 \supset \cdots,  \\
        &\diam(A_n) \le \frac{2}{n}, \\
        & A_n \text{ cannot be covered by any finitely many } G_\alpha. 
    \end{aligned}
    \right.
\end{equation*}
Choosing $x_n \in A_n$, then for $n > m$, $x_n, x_m \in A_m$, which implies 
$d(x_m, x_n) \frac{2}{m}$ and thus $\{x_n\}$ is a Cauchy sequence in $X$. 
By completeness of $X$, we assume that $\{x_n\}$ converges to $x \in X$. 
Since $A$ is closed and $x_n \in A$, it follows that $x \in A$. 
Hence $x \in G_\alpha$ for some $\alpha \in I$. 
That $G_\alpha$ is open yields that there exists $B(x, \delta) \in G_\alpha$ 
and there exists $N \in \bN$ such that for every $n > N$, 
\begin{equation*}
    d(x_n, a) < \frac{\delta}{2}.
\end{equation*}
Therefore, for $n > \max \left\{ \frac{4}{\delta}, N \right\}$ and $x \in 
A_n$ 
\begin{equation*}
    d(x, a) \le d(x, x_n) + d(x_n, a) < \frac{2}{n} + \frac{\delta}{2} 
    < \delta. 
\end{equation*}
\ie, $A_n \subseteq B(a, \delta) \subseteq G_\alpha$. which contradicts the 
hypothesis. 
\end{proof}

\begin{cor}
\label{cor:metric_spaces:compact_sets:totally_bounded_rn}
A bounded subset $F$ in $\bR^n$ is totally bounded. 
\end{cor}
\begin{proof}
$F$ is bounded in $\bR^n$ $\implies$ $\bar{F}$ is a compact set in $\bR^n$ 
$\implies$ $\bar{F}$ is totally bounded $\implies$ $F$ is totally bounded. 
\end{proof}
%%%%%%%%%%%%%%%%%%%%%%%%%%%%%%%%%%%%%%%%%%%%%%%%%%%%%%%
%%  Subsection: Arzelà–Ascoli Theorem
%%%%%%%%%%%%%%%%%%%%%%%%%%%%%%%%%%%%%%%%%%%%%%%%%%%%%%%
\subsection{Arzelà–Ascoli Theorem}
As metric space $C[a, b]$ is of fundamental importance in analysis, there 
is a theorem describing the compactness of $C[a, b]$, which is the well-
known Arzelà–Ascoli Theorem. 
Before stating the theorem, we need the following definitions: 
\begin{defn}[uniformly bounded, equicontinuous]
\index{uniformly bounded}
\index{equicontinuous}
Let $F$ be a subset of $C[a, b]$ where $-\infty < a < b < \infty$. 
We call functions in $F$ are \emph{uniformly bounded} if there exists a 
constant $M \ge 0$ such that for every $f \in F$, 
\begin{equation*}
    \abs{f(x)} \le M, \quad \text{for any } x \in [a, b].
\end{equation*}

$F$ is called \emph{equicontinuous} if for every $\epsilon > 0$, there 
exists $\delta > 0$ such that for all $t_1, t_2 \in [a, b]$ such that 
$\abs{t_1 - t_2} < \epsilon$ and $f \in F$, 
\begin{equation*}
    \abs{f(t_1) - f(t_2)} < \epsilon. 
\end{equation*}
\end{defn}

\begin{thm}[Arzelà–Ascoli Theorem]
Let $F \subseteq C[a, b]$ be closed where $-\infty < a < b < \infty$. 
Then $F$ is compact if and only if $F$ is uniformly bounded and 
equicontinuous. 
\end{thm}
\begin{proof}
By Theorem \ref{thm:metric_spaces:compact_sets:characterizations}, it 
suffices to prove that $F$ is totally compact if and only if $F$ is 
uniformly bounded and equicontinuous. 

$\Rightarrow$
Suppose $F$ is totally bounded. 
Then $F$ is bounded, implying uniform boundedness of $F$. 
Setting $\epsilon > 0$, there exist $f_1, f_2 \ldots, f_m \in C[a, b]$ 
such that 
\begin{equation*}
    F \subseteq \bigcup_{i=1}^m B(f_i, \frac{\epsilon}{3}). 
\end{equation*}
Noting that for every $i \in \{1, 2, \ldots, m\}$, $f_i$ is uniformly 
continuous, thus there exists $\delta_i > 0$ such that for $t_1, t_2 \in 
[a, b]$ satisfying $\abs{t_1 - t_2} < \delta_i$, 
\begin{equation*}
    \abs{f_i(t_1) - f_i(t_2)} < \epsilon. 
\end{equation*}
Take $\delta = \min \{\delta_1, \delta_2, \ldots, \delta_m\}$. 
Thus for every $f \in F$, there exists $i \in \{1, 2, \ldots, m\}$ 
such that $f \in B(f_i, \frac{\epsilon}{3})$. 
This is, for $t_1, t_2 \in [a, b]$ satisfying $\abs{t_1 - t_2} < \delta$, 
\begin{equation*}
    \abs{f(t_1) - f(t_2)} \le \abs{f(t_1) - f_i(t_1)} 
    + \abs{f_i(t_1) - f_i(t_2)} + \abs{f_i(t_2) - f(t_2)} < \epsilon. 
\end{equation*}
Hence, $F$ is equicontinuous. 

$\Leftarrow$
Suppose that $F$ is uniformly bounded and equicontinuous.
Letting $\epsilon > 0$, as $F$ is equicontinuous, there exists $\delta > 0$ 
such that for any $\abs{t_1 - t_2} < \delta$ and $f \in F$, 
\begin{equation*}
    \abs{f(t_1) - f(t_2)} < \frac{\epsilon}{5}. 
\end{equation*}
Uniformly divide the closed interval $[a, b]$ into $a = t_0 < t_1 < t_2 
< \cdots < t_m = b$ such that 
\begin{equation*}
    \abs{t_{i+1} - t_i} = \frac{b - a}{m} < \delta. 
\end{equation*}
Hence, for every $i \in \{1, 2, \ldots, m\}$, $t \in [t_i, t_{i+1}]$ 
implies $\abs{f(t) - f(t_i)} < \frac{\epsilon}{5}$. 
Consider the set 
\begin{equation*}
    E = \left\{ (f(t_0), f(t_1), \ldots, f(t_m)): f \in \cF \right\} 
    \subseteq \bC^m. 
\end{equation*}
By uniform boundedness of $F$, $E$ is bounded and thus totally bounded with 
Corollary \ref{cor:metric_spaces:compact_sets:totally_bounded_rn}. 
Therefore, there exists $f_1, f_2, \ldots, f_p \in F$ such that for any 
$f \in F$ and $i \in \{1, 2, \ldots, m\}$, there exists $j \in \{1, 2, 
\ldots, p\}$ such that 
\begin{equation*}
    \abs{f(t_i) - f_j(t_i)} < \frac{\epsilon}{5}. 
\end{equation*} 
We claim that 
\begin{equation*}
    F \subseteq \bigcup_{i = 1}^p B(f_i, \epsilon). 
\end{equation*}
Indeed, for every $f \in F$, there exists $f_i \in F$ such that 
$\abs{f(t_j) - f_i(t_j)} < \frac{\epsilon}{5}$ for any 
$j \in \{1, 2, \ldots, m\}$. 
Supposing $t \in [a, b]$ and $t \in [t_j, t_{j+1}]$, then 
\begin{equation*}
    \begin{aligned}
        \abs{f(t) - f_i(t)} \le \abs{f(t) - f(t_j)} + \abs{f(t_j) - f_i(t_j)} 
        + \abs{f_i(t_j) - f_i(t)} \le \frac{3}{5} \epsilon < \epsilon. 
    \end{aligned}
\end{equation*}
Thus we have proven that $F$ is totally bounded. 
\end{proof}

\section{Cauchy Sequences}


\begin{defn}
By the upper limit of a sequence $\{x_k\}$, we mean 
\begin{equation}
    l = \lim_{n\to \infty} \sup _{k\ge n} x_k, 
\end{equation}
denoted by $\limsup_{n\to\infty} x_n$. If $\{x_k\}$ is unbounded above, then 
$\limsup_{n\to\infty} x_n = \infty$.
Likewise, the lower limit of $\{x_k\}$ is 
\begin{equation}
    \liminf _{n \to \infty} x_n = \lim_{n\to\infty}\sup_{k\ge n} x_k.
\end{equation}
\end{defn}

Let $b_n = \sup_{k > n} x_k, \ldots\}$. We have $b_n > b_{n+1}$ for any $n$;
so $\lim_{n\to\infty}b_n = \inf b_n$ is either a finite number or $\infty$.


\chapter{Topological Linear Spaces}
%%%%%%%%%%%%%%%%%%%%%%%%%%%%%%%%%%%%%%%%%%%%%%%%%%%%%%%
%%  Basic Properties of Topological Linear Spaces
%%%%%%%%%%%%%%%%%%%%%%%%%%%%%%%%%%%%%%%%%%%%%%%%%%%%%%%
Norms and inner products are common ways to introduce topology to a linear 
space. 
But there are other ways to endow topology. 
The main aim of this chapter is to generalize the results we have studied 
in Banach spaces and Hilbert spaces to more general topological linear 
space. 

\section{Basic Properties of Topological Linear Spaces}
\begin{defn}[Topological linear space]
\index{topological linear space}
Let $X$ be a linear space over number field $\bK$, and let $\cT$ be a 
topology on $X$. 
The topological space $(X, \cT)$ is said to be a topological linear space 
if the following are satisfied: 
\begin{enumerate}
    \item ($T_1$ axiom) Every set of single element is closed. 
    \item The linear space operations (addition and scalar multiplication) 
    is continuous with respect to $\cT$. 
\end{enumerate}
Under these conditions, $\cT$ is called a \emph{linear topology} on $X$. 
\end{defn}

More specifically, by continuity of addition we mean the mapping 
\begin{equation*}
    (x, y) \mapsto x + y 
\end{equation*}
is a continuous function from $X \times X$ to $X$ is continuous: 
If $x_1, x_2 \in X$ and $V$ is a neighborhood of $x_1 + x_2$, then there 
should exist neighborhoods $V_1, V_2$ of $x_1, x_2$ respectively such that 
$V_1 + V_2 \subseteq V$. 
Likewise, by continuity of scalar multiplication we mean for every 
$x \in X$ nad $\alpha \in \bK$, if $V$ is a neighborhood of $\alpha x$, 
then there should exist $\delta > 0$ and neighborhood $U_1$ of $x$ such 
that $\beta U \subseteq V$ whenever $\abs{\beta - \alpha} < \delta$. 

Whenever the topology is clear according to context, we write a topological 
linear space $(X, \cT)$ for convenience. 
For a topological linear space $X$, we define the translation operator and 
scalar product operator as 
\begin{equation*}
    T_y(x) = x + y, \quad M_\alpha(x) = \alpha x. 
\end{equation*}
One can easily show that for any $y \in X$ and $\alpha \neq 0$, the 
operators $T_y$ and $M_\alpha$ is a homeomorphism between $X$ and $X$. 

Since a subset $E \subseteq X$ is open if and only if for every $y \in X$, 
we have $T_{y}(E)$ is open, we know that the topology of a topological 
linear space is represented by its local base at $0$. 
By local base we mean the the class $\cB$ of neighborhoods of $0$ such that 
any neighborhood of $0$ contains at least one element of $\cB$. 

\subsection{Seperation Properties}
\begin{lemma}
\label{lemma:topological_linear_spaces:seperation_lemma}
Let $X$ be a topological linear space. 
If $W$ is a neighborhood of $0$, then there exists a neighborhood $U$ of $0$ 
which is symmetric (in the sense that $U = -U$) such that $U + U \subseteq 
W$. 
\end{lemma}
\begin{proof}
Note that $0 = 0 + 0$, that addition is continuous, and that $W$ is a 
neighborhood of $0$. 
Therefore, there are neighborhoods $V_1$, $V_2$ of $0$ such that $V_1 + V_2
\subseteq W$. 
Setting $U = V_1 \cap V_2 \cap (-V_1) \cap (-V_2)$, then $U$ is a 
neighborhood of $0$ and has the required properties. 
\end{proof}

\begin{thm}
\label{thm:topological_linear_spaces:seperation_theorem}
Suppose $K$ and $C$ are subsets of a topological linear space $X$, $K$ is 
compact, $C$ is closed, and $K \cap C = \varnothing$. 
Then $0$ has a neighborhood $V$ such that 
\begin{equation*}
    (K + V) \cap (C + V) = \varnothing. 
\end{equation*}
\end{thm}
\begin{proof}
Without loss of generality, assume that $K \neq \varnothing$. 
Let $x \in K$. 
Then $x \notin C$. 
Since $C$ is closed, there exists a neighborhood $W_x$ of $0$ such that 
$x + W_x \cap C = \varnothing$. 
By Lemma \ref{lemma:topological_linear_spaces:seperation_lemma}, $0$ has 
neighborhood $V_x$ which is symmetric such that $V_x + V_x + V_x + V_x 
\subseteq W$. 
Hence, $(x + V_x + V_x + V_x + V_x) \cap C = \varnothing$, which is 
equivalent to $(x + V_x + V_x + V_x) \cap (C + V_x) = \varnothing$. 
This implies 
\begin{equation}
    \label{equ:topological_linear_spaces:seperation_theorem:1}
    (x + V_x + V_x) \cap (C + V_x) = \varnothing.
\end{equation} 

On the other hand, by the comppactness of $K$, there exists finitely many 
elements $x_1, x_2, \ldots, x_n$ of $K$ such that $K \subset \bigcup_{k=1}
^n (x_k + V_{x_k})$. 
Putting $V = \bigcap_{k=1}^n V_{x_k}$, then 
\begin{equation*}
    K + V \subset \bigcup_{k=1}^n (x_k + V_{x_k} + V) 
    \subset \bigcup_{k=1}^n (x_k + V_{x_k} + V_{x_k}).
\end{equation*}
By the Equation (\ref{equ:topological_linear_spaces:seperation_theorem:1}), 
$(K + V) \cap (C + V_{x_i}) = \varnothing$, and therefore $(K + V) \cap 
(C + V) = \varnothing$. 
\end{proof}

The theorem above demonstrates that in a topological linear space, disjoint 
compact set and closed set can be seperated by two disjoint open sets. 

\begin{cor}
Topological linear spaces are Hausdorff spaces. 
\end{cor}
\begin{proof}
For any $x, y \in X$, let $K = \{x\}$ and $C = \{y\}$, and apply the 
preceding theorem. 
\end{proof}

\begin{cor}
If $\cB$ is a local base of a topological linear space $X$, then every 
member of $\cB$ contains the closure of some member of $\cB$. 
\end{cor}
\begin{proof}
In the proof of Theorem \ref{thm:topological_linear_spaces:seperation_theorem}, 
since $C + V$ is open, it is true that $\closure{(K + V)}$ does not intersect 
$C + V$. 
This corollary follows if we set $K = \{ 0 \}$. 
\end{proof}

\subsection{Balanced Sets and Bounded Sets}
The concept of balanced set is a generalization of the concept of ball in 
normed space. 
\begin{defn}[Balanced set]
\index{balanced set}
Let $X$ be a linear space, and let $B$ be a subset of $X$. 
We say $B$ is \emph{balanced} if $\alpha B \subset B$ whenever $\abs{\alpha}
 < 1$. 
\end{defn}

\begin{prop}
\label{prop:topological_linear_spaces:balanced_set_properties}
Let $X$ be a topological vector space. 
The following hold true. 
\begin{enumerate}
    \item \label{prop:topological_linear_spaces:balanced_set_properties:1}
    If $B \subseteq X$ is balanced, then $\closure{B}$ is balanced. 
    \item \label{prop:topological_linear_spaces:balanced_set_properties:2}
    If $B \subseteq X$ is balanced, and $0 \in \interior{A}$, then 
    $\interior{B}$ is balanced. 
    \item \label{prop:topological_linear_spaces:balanced_set_properties:3}
    Every neighborhood of $0$ contains some neighborhood of $0$ which 
    is balanced. 
    \item \label{prop:topological_linear_spaces:balanced_set_properties:4}
    Every convex neighborhood of $0$ contains some convex neighborhood 
    of $0$ which is balanced. 
\end{enumerate}
\end{prop}
\begin{proof}
\ref{prop:topological_linear_spaces:balanced_set_properties:1}
Omitted. 

\ref{prop:topological_linear_spaces:balanced_set_properties:2}
Suppose that $0 < \abs{\alpha} \le 1$. 
Then 
\begin{equation*}
    \alpha \interior{B} = \interior{(\alpha B)} 
    \subseteq \alpha B \subseteq B. 
\end{equation*}
Since $\alpha \interior{B}$ is open, it follows that $\alpha \interior{A} 
\subseteq \interior{A}$. 
Thanks to the assumption that $0 \in \interior{A}$, $\alpha \interior{A} 
\subseteq \interior{A}$ holds true for the case $\alpha = 0$. 
Hence, $\interior{A}$ is also a balanced set. 

\ref{prop:topological_linear_spaces:balanced_set_properties:3}
Let $U$ be a neighborhood of $0$. 
By the continuity of scalar product, there is a $\delta > 0$ and a 
neighborhood $V$ of $0$ such that $\alpha V \subseteq U$ whenever 
$\abs{\alpha} < \delta$. 
Put $W = \bigcup_{\abs{\alpha} < \delta} \alpha V$, which is clearly 
neighborhood of $0$ that $W \subseteq U$. 
Now we verify that $W$ is balanced. 
Indeed, for any $x \in W$ and $\beta$ with $\abs{\beta} < 1$, there is an 
$\alpha$ with $\abs{\alpha} < \delta$ such that $x \in \alpha V$. 
Hence, $\abs{\beta \alpha} \le \abs{\alpha} \le \delta$, and $\beta x 
\in \beta \alpha V \subset W$. 

\ref{prop:topological_linear_spaces:balanced_set_properties:4}
Let $U$ be a convex neighborhood of $0$, and let $A = \bigcap_
{\abs{\alpha} = 1} \alpha U$. 
Then $0 \in A$, and $A$ is convex since $\alpha U$ is convex for any 
$\alpha \in \bK$. 
From \ref{prop:topological_linear_spaces:balanced_set_properties:3} in this 
proposition, there is a balanced neighborhood $W \subseteq U$. 
For any $\alpha \in \bK$ with $\abs{\alpha} = 1$, we have $\alpha^{-1} W 
\subseteq W$, and thus $W \subseteq \alpha W \subseteq \alpha U$. 
Therefore, $W \subseteq W$, and $0 \in W \in \interior{A}$. 
If we can verify that $A$ is a balanced set, then it follows that 
$\interior{A}$ is balanced by \ref{prop:topological_linear_spaces:balanced_set_properties:2}
of this proposition. 
Indeed, let $r \in [0, 1]$, $\beta \in \bK$ with $\abs{\beta} = 1$, then 
\begin{equation*}
    \begin{aligned}
        r\beta A = \bigcap_{\abs{\alpha} = 1} r \beta \alpha U 
        = \bigcap_{\abs{\alpha} = 1} r \alpha U 
        \subseteq \bigcap_{\abs{\alpha} = 1} \alpha U 
        = A. 
    \end{aligned}
\end{equation*}
Finally, it is not hard to show that $\interior{A}$ is convex. 
Hence, $\interior{A}$ is what we want. 
\end{proof}

\begin{defn}[Locally convex space]
Let $\cB$ be a local base of a topological linear space $X$. 
Then $\cB$ is said to be balanced if every member of $\cB$ is balanced; 
$\cB$ is said to be convex if every member of $\cB$ is convex. 

If $X$ has a convex local base, then $X$ is said to be locally convex. 
\end{defn}
Locally convex topological linear spaces are of considerable interest. 
\begin{cor}
\begin{enumerate}
    \item Every topological linear topological space has a balanced local 
    base. 
    \item Every locally convex topological linear space has a balanced 
    convex local base. 
\end{enumerate}
\end{cor}

Now we try to generalize the concept of boundedness to topological linear 
spaces. 
\begin{defn}[Bounded set]
Let $E$ be a subset of a topological linear space $X$. 
$E$ is said to be bounded if for every neighborhood $V$ of $0$, there 
exists $s > 0$ such that $E \subseteq t V$ whenever $t > s$. 
\end{defn}
\begin{prop}
\label{prop:topological_linear_spaces:bounded_sets_property}
Let $E$ be a subset of a topological linear space $X$. 
Then $E$ is bounded if and only if for every $\{ x_n \}_{n=1}^{\infty} 
\subseteq E$ and $\{ \alpha_n \}_{n=1}^{\infty} \subset \bK$, 
\begin{equation}
    \label{equ:topological_linear_spaces:bounded_sets_property}
    \alpha_n \to 0 \quad (n \to \infty) 
    \implies \alpha_n x_n \to 0 \quad (n \to \infty).
\end{equation} 
\end{prop}
\begin{proof}
($\implies$)
Suppose that $E$ is bounded, $V$ is balanced neighborhood of $0$, $\{ x_n \}
_{n=1}^{\infty} \subseteq E$ and $\{ \alpha_n \}_{n=1}^{\infty} \subseteq 
\bK$ with $\alpha_n \to \infty \quad (n \to \infty)$. 
Then there is a $t > 0$ such that $E \subseteq tV$, and there is $N \in 
\bNs$ such that $\abs{\alpha_n} t < 1$ whenever $n > N$. 
Hence, since $V$ is balanced, 
\begin{equation*}
    \alpha_n x_n = \alpha_n t \times \frac{1}{t} x_n \in V, 
\end{equation*}
which implies $\alpha_n x_n \to \infty$ as $n \to \infty$ by definition. 

Conversely, assume that condition 
(\ref{prop:topological_linear_spaces:bounded_sets_property}) is satisfied, 
and for the sake of contradiction $E$ is unbounded. 
Then there exists neighborhood $V$ of $0$ such that for every $n \in \bNs$, 
there is some $x_n \in E$ with $x_n \notin n V$. 
Then $\frac{1}{n} x_n \notin V$ for all $n \in \bNs$, and thus 
$\frac{1}{n}x_n$ does not converges $0$, which contradicts the condition  
(\ref{prop:topological_linear_spaces:bounded_sets_property}). 
Therefore, $E$ must be bounded. 
\end{proof}

\begin{thm}
\label{thm:topological_linear_spaces:bounded_sets_base}
Let $X$ be a topological linear space, and let $V$ be a neighborhood of $0$. 
Then the following hold true. 
\begin{enumerate}
    \item \label{thm:topological_linear_spaces:bounded_sets_base:1}
    If $\{ \alpha_n \}_{n=1}^{\infty} \subseteq \bR$ is a monotonically 
    increasing sequence, then $X = \bigcup_{n=1}^\infty \alpha_n V$. 
    \item \label{thm:topological_linear_spaces:bounded_sets_base:2}
    Every compact set in $X$ is bounded. 
    \item \label{thm:topological_linear_spaces:bounded_sets_base:3}
    If $\{ \alpha_n \}_{n=1}^{\infty} \subseteq \bR$ is a 
    monotonically decreasing sequence converging to $0$, then $\left\{ 
    \alpha_n V: n \in \bNs \right\}$ is a local base of $X$. 
\end{enumerate}
\end{thm}
\begin{proof}
\ref{thm:topological_linear_spaces:bounded_sets_base:1}
For any $x \in X$, the mapping $\alpha \mapsto \alpha x$ that maps $\bK$ 
into $X$ is continuous. 
Then $A_x = \{\alpha \in \bK: \alpha x \in V$ is open, since $V$ is open. 
Noting that $0 \in A_x$, then for sufficient large $n$, $\frac{1}{\alpha_n} 
x \in V$, namely, $x \in r_n V$. 

\ref{thm:topological_linear_spaces:bounded_sets_base:2}
Let $K$ be a compact set in $X$, and let $U$ be a neighborhood of $0$. 
Then there is some balanced neighborhood $W$ of $0$ such that 
$W \subseteq U$. 
From \ref{thm:topological_linear_spaces:bounded_sets_base:1}, 
\begin{equation*}
    K \subseteq X = \bigcup_{n= 1}^\infty n W. 
\end{equation*}
By the compactness of $K$, there exists finitely many $n_1, n_2, \ldots, 
n_N$ such that 
\begin{equation*}
    K \subseteq \bigcup_{k=1}^N n_k W. 
\end{equation*}
Assume without loss of generality that $n_1 < n_2 < \cdots < n_N$. 
Since $0 < \frac{n_k}{n_{k+1}} < 1$  for $k = 1, 2, \ldots, N - 1$ and 
$W$ is balanced, 
\begin{equation*}
    n_1 W \subseteq n_2 W \subseteq \cdots \subseteq n_N W. 
\end{equation*}
Hence, $K \subset n_N W \subset n_N U$. 

\ref{thm:topological_linear_spaces:bounded_sets_base:3}
Let $U$ be any neighborhood of $0$. 
By the boundedness of $V$, there exists $s > 0$ such that $V \subseteq tU$ 
for $t > s$. 
Then it is valid to choose $n \in \bNs$ such that $\frac{1}{\delta_n} > s$. 
In this case, $V \subseteq \frac{1}{\delta_n} U$, namely, $\delta_n V 
\subseteq U$. 
\end{proof}

\subsection{Metriczability}
\begin{thm}
\label{thm:topological_linear_spaces:Metriczability}
Let $(X, \cT)$ be a topological linear space which has a countable base. 
Then there exists a distance function $d: X \times X \to \bR$ such that 
\begin{enumerate}
    \item $\cT$ is induced by $d$. 
    \item Every open ball (in the sense of $d$ centered at $0$ is balanced. 
    \item (Translation invariance) $d(x + z, y + z) = d(x, y)$ for any 
    $x, y, z \in X$. 
    \item If $(X, \cT)$ is locally convex, then every open ball in $X$ is 
    convex. 
\end{enumerate}
\end{thm}
\begin{proof}

\end{proof}

\subsection{Bounded Linear Operators}
Having given the definition of bounded set, it is time to generalize the 
concept of bounded linear operator to topological linear spaces. 
\begin{defn}[Bounded linear operator in topological linear space]
Let $X$, $Y$ be topological linear space.
An linear operator $T: X \to Y$ is \emph{bounded} if $T$ maps every bounded 
set to a bounded set. 
\end{defn}

\begin{thm}
\label{thm:topological_linear_spaces:bounded_linear_operators_properties}
Let $X$, $Y$ be topological linear space, and let $T: X \to Y$ be a linear 
operator. 
Then consider the following statements:
\begin{enumerate}
    \item \label{thm:topological_linear_spaces:bounded_linear_operators_properties:1}
    $T$ is continuous at $0$. 
    \item \label{thm:topological_linear_spaces:bounded_linear_operators_properties:2}
    $T$ is continuous. 
    \item \label{thm:topological_linear_spaces:bounded_linear_operators_properties:3}
    $T$ is bounded. 
    \item \label{thm:topological_linear_spaces:bounded_linear_operators_properties:4}
    If $\{ x_n \}_{n=1}^{\infty} \subseteq X$ converges to $0$, then 
    $\{ Tx_n: n \in \bNs \}$ is a bounded set. 
    \item \label{thm:topological_linear_spaces:bounded_linear_operators_properties:5}
    If $\{ x_n \}_{n=1}^{\infty} \subseteq X$ converges to $0$, then 
    $Tx_n \to 0$ as $n \to \infty$. 
\end{enumerate}
We have the implications:
\begin{equation*}
    \ref{thm:topological_linear_spaces:bounded_linear_operators_properties:1} 
    \iff 
    \ref{thm:topological_linear_spaces:bounded_linear_operators_properties:2} 
    \implies 
    \ref{thm:topological_linear_spaces:bounded_linear_operators_properties:3} 
    \implies 
    \ref{thm:topological_linear_spaces:bounded_linear_operators_properties:4}. 
\end{equation*}
If further $X$ is metrizable, then 
\begin{equation*}
    \ref{thm:topological_linear_spaces:bounded_linear_operators_properties:4} 
    \implies 
    \ref{thm:topological_linear_spaces:bounded_linear_operators_properties:5} 
    \implies
    \ref{thm:topological_linear_spaces:bounded_linear_operators_properties:1}. 
\end{equation*}
\end{thm}
\begin{proof}
\ref{thm:topological_linear_spaces:bounded_linear_operators_properties:1}
$\implies$ 
\ref{thm:topological_linear_spaces:bounded_linear_operators_properties:2}
Let $W$ be a neighborhood of $0$. 
Then by the continuity of $T$, there is some neighborhood $W$ of $0$ 
such that $TV \subseteq W$. 
Hence, for any $x \in X$, if $y - x \in V$, it follows that $Ty - Tx = 
T(y - x) \subset TV \subset W$, which implies $Ty \subset Tx + W$. 
Thus, $T$ maps the neighborhood $x + V$ of $x$ into the neighborhood $Tx+W$ 
of $Tx$, saying that $T$ is continuous at $x$. 

\ref{thm:topological_linear_spaces:bounded_linear_operators_properties:2}
$\implies$ 
\ref{thm:topological_linear_spaces:bounded_linear_operators_properties:1}
Trivial. 

\ref{thm:topological_linear_spaces:bounded_linear_operators_properties:2}
$\implies$ 
\ref{thm:topological_linear_spaces:bounded_linear_operators_properties:3}
Suppose that $E$ is a bounded set in $X$, and $W \subseteq Y$ is a 
neighborhood of $0$. 
There is a neighborhood $V\subseteq X$ of $9$ such that $TV \subseteq W$. 
Since $E$ is bounded, there exists a positive number $s \in \bR$ such that 
$E \subseteq tV$ whenever $t > s$, which implies that $TE \subseteq T(tV) 
\subseteq t TV \subseteq t W$. 
This means $TE$ is a bounded set, and thus $T$ is bounded. 

\ref{thm:topological_linear_spaces:bounded_linear_operators_properties:3}
$\implies$ 
\ref{thm:topological_linear_spaces:bounded_linear_operators_properties:4}
This follows from boundedness of the set $\{x_n: n \in \bNs\}$.

Now assume that $T$ is metrizable in the rest of the proof. 
\ref{thm:topological_linear_spaces:bounded_linear_operators_properties:4}
$\implies$ 
\ref{thm:topological_linear_spaces:bounded_linear_operators_properties:5}
Suppose that \ref{thm:topological_linear_spaces:bounded_linear_operators_properties:4}
is satisfied, and that $\{ x_n \}_{n=1}^{\infty}$ is a sequence in $X$ 
converging to $0$. 
By Theorem, there is some sequence $\{ \alpha_n \}_{n=1}^{\infty}$ in $\bR$ 
converging to $\infty$ such that $\alpha_n x_n \to 0$ as $n \to \infty$. 
Then by assumption, $\{T \alpha_n x_n: n \in \bNs\}$ is a bounded set. 
The computation 
\begin{equation*}
    Tx_n = \frac{1}{\alpha_n} T(\alpha_n x_n)
\end{equation*}
yields that $Tx_n \to 0$ as $n \to \infty$. 

\ref{thm:topological_linear_spaces:bounded_linear_operators_properties:5}
$\implies$ 
\ref{thm:topological_linear_spaces:bounded_linear_operators_properties:1}
Suppose that condition statement 
\ref{thm:topological_linear_spaces:bounded_linear_operators_properties:5} 
is satisfied, but $T$ is discontinuous at $0$. 
That is, there exists a neighborhood $W \subseteq Y$ of $0$ such that 
$T^{-1}(W)$ contains no neighborhood of $0 \in X$. 
Since $X$ is metrizable, there exists a countable base $\{V_n: n \in \bNs\}$ 
satisfying $V_{n + 1} + V_{n + 1} \subseteq V_n$ for all $n \in \bNs$. 
Hence, it is valid to find a sequence $\{ x_n \}_{n=1}^{\infty}$ in $X$ 
such that $x_n \in V_n \backslash T^{-1}(W)$. 
Then $x_n \to 0$ while $Tx_n \not \to 0$, contradicting the condition 
\ref{thm:topological_linear_spaces:bounded_linear_operators_properties:5}. 
\end{proof}

\begin{thm}
\label{thm:topological_linear_spaces:linear_functional}
Let $f$ be a non-zero linear functional on a topological linear space $X$. 
Then the following following are equivalent. 
\begin{enumerate}
    \item \label{thm:topological_linear_spaces:linear_functional:1}
    Functional $f$ is continuous. 
    \item \label{thm:topological_linear_spaces:linear_functional:2}
    The null space $\sN(f)$ is a closed subspace of $X$. 
    \item \label{thm:topological_linear_spaces:linear_functional:3}
    $\sN(f)$ is not dense in $X$, \ie, $\closure{(\sN(f))} \neq X$. 
    \item \label{thm:topological_linear_spaces:linear_functional:4}
    Functional $f$ is bounded in some neighborhood $V \subseteq X$ 
    of $0$. 
\end{enumerate}
\end{thm}
\begin{proof}
\ref{thm:topological_linear_spaces:linear_functional:1}
$\implies$ 
\ref{thm:topological_linear_spaces:linear_functional:2}
By the linearity of $f$, we know $\sN(f)$ is subspace of $X$; 
And by the continuity of $f$, we know $\sN(f)$ is closed. 

\ref{thm:topological_linear_spaces:linear_functional:2}
$\implies$ 
\ref{thm:topological_linear_spaces:linear_functional:3}
As $\sN(f)$ is closed and $f$ is non-zero, we have $\closure{(\sN(f))} 
= \sN(f) \neq X$. 

\ref{thm:topological_linear_spaces:linear_functional:2}
$\implies$ 
\ref{thm:topological_linear_spaces:linear_functional:3}
Suppose that $\sN(f)$ is not dense in $X$, then there is some $x_0 \in X$ 
and neighborhood $V \subseteq X$ of $0$ such that $x_0 + V \cap 
\closure{(\sN(f))} = \varnothing$. 
With no harm to generality, assume that $V$ is balanced. 
Now we show that $f(V)$ is bounded by contradiction. 
Suppose that $f(V)$ is unbounded. 
Then for any $\alpha \in \bK$, there is a $x \in V$ such that $\abs{f(x)} > 
\alpha$. 
Assuming that $f(x) = \abs{f(x)} \me^{\mi \theta}$, we have 
\begin{equation*}
    f\left( \frac{\alpha \me^{-\mi \theta} x}{\abs{f(x)}} \right) 
    = \alpha. 
\end{equation*}
Since $\abs{\frac{\alpha \me^{-\mi \theta}}{\abs{f(x)}}} < 1$, we have 
$\frac{\alpha \me^{-\mi \theta} x}{\abs{f(x)}} \in V$. 
Hence, $f(V) = \bK$. 
Therefore, there is some $y_0 \in V$ such that $f(y_0) = -f(x_0)$, or 
equivalently $x_0 + y_0 \in \sN(f)$, in contradiction to our assumption. 

\ref{thm:topological_linear_spaces:linear_functional:4}
$\implies$ 
\ref{thm:topological_linear_spaces:linear_functional:1}
If statement \ref{thm:topological_linear_spaces:linear_functional:4} is 
holds, then there exists a neighborhood $V$ of $0$ and positive real number 
$M$ such that $\abs{f(x)} < M$ for any $x \in V$. 
For any $\epsilon > 0$, put $W = \frac{\epsilon}{M} V$. 
Then for any $y \in W$, there is a correspongding $x \in V$ such that 
$y = \frac{\epsilon}{M} x$. 
Hence, $\abs{f(y)} = \frac{\epsilon}{M}\abs{f(x)} < \epsilon$, which implies 
that $f$ is continuous. 
\end{proof}
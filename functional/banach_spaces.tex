\chapter{Banach Spaces}
\label{chp:banach_spaces}
%%%%%%%%%%%%%%%%%%%%%%%%%%%%%%%%%%%%%%%%%%%%%%%%%%%%%%%
%%  Section: Linear Spaces
%%%%%%%%%%%%%%%%%%%%%%%%%%%%%%%%%%%%%%%%%%%%%%%%%%%%%%%
\section{Norms}
\begin{defn}
\index{semi-norm}
A real-valued function $p$ defined on a linear space $X$ is called a 
semi-norm on $X$, if the following conditions are satisfied 
\begin{enumerate}
    \item (Subadditivity) $p(x+y) \le p(x) + p(y)$;
    \item $p(\alpha x) = \abs{\alpha} p(x)$, 
\end{enumerate}
for any $x, y \in X$ and $\alpha \in \mathbb{K}$. 
\end{defn}

\begin{defn}[norm]
Let $\bK = \bR$ or $\bC$ and $X$ be a linear space over $K$. 
A norm $\norm{\wdot}: X \to \bR$ is a function such that 
\begin{enumerate}
    \item (positive definiteness) $\norm{x} \ge 0$ for every $x \in X$ and 
    $\norm{x} = 0 \iff x = 0$. 
    \item (absolute homogeneousness) $\norm{\alpha x} = \abs{\alpha} 
    \norm{x}$ for any $x \in X$ and $\alpha \in \bK$. 
    \item (triangle inequality) $\norm{x + y} \le \norm{x} + \norm{y}$ for 
    any $x, y \in X$.
\end{enumerate}
\end{defn}

\begin{defn}[normed space]
\index{normed (linear) space}
\index{space!normed (linear) \~{}}
A linear space $X$ equipped with a norm $\norm{\wdot}$ is called a 
\emph{normed (linear) space}, denoted by $(X, \norm{\wdot})$. 
\end{defn}

For the sake of conciseness, we simply write $X$ for a normed space when it 
is unambiguous. 

Obviously, a normed space $(X, \norm{\wdot})$ is a metric space together 
with the distance defined by $d(x, y) = \norm{x - y}$, called the induced 
metric from the norm. 
From now on, normed spaces mentioned will be implicitly equipped with the 
distances induced from the norm. 
Therefore, normed spaces are finer than metric spaces. 
All definitions and theorems for metric spaces can be transferred to normed 
spaces at once. 
\index{Banach space}
\index{space!Banach \~{}}
Specifically, a complete normed space is called a \emph{Banach space}. 

We have some basic facts about normed spaces, whose proofs has been omitted: 
\begin{prop}
The norm in a normed space is continuous.
\end{prop}

\begin{example}
Consider the set $\bK^n$. 
Define
\begin{equation*}
    \begin{aligned}
        \norm{x}_2 &= \sqrt{\sum_{k=1}^{\infty} x_k^2},  \\ 
        \norm{x}_1 &= \sum_{k=1}^{\infty} \abs{x_k},  \\
        \norm{x}_\infty &= \max_{1 \le k \le n} \abs{x_k}.
    \end{aligned}
\end{equation*}
where $x = \left( x_1, x_2, \ldots, x_n \right) \in \bK^n$. 
It can be verified that all of them above are norms. 
They are known as $2$-norm, $1$-norm and $\infty$-norm respectively. 
\end{example}

\begin{example}
Consider the set $C[a, b]$. 
For $f \in C[a, b]$, define 
\begin{equation*}
    \norm{f} = \max_{t \in [a, b]} \abs{f(t)}. 
\end{equation*}
The induced distance coincides with that given in Example 
\ref{ex:metric_spaces:induction:cab}. 
Under this norm, $C[a, b]$ is a Banach space. 
\end{example}

\begin{example}
The linear space $l^\infty = \left\{ x = \left(x_1, x_2, \ldots \right): 
x \text{ is bounded.} \right\}$ equipped with the norm 
\begin{equation*}
    \norm{x} = \sup_{n \in \bN^\ast} \abs{x_n} 
\end{equation*}
is an inseparable Banach space. 
\end{example}

\begin{example}
The set $V[a, b]$ of all BV function (Definition 
\ref{def:bounded_variation_function}) along with the norm 
\begin{equation}
    \label{equ:banach_spaces:norms:vab_norm}
    \norm{x} = \abs{f(a)} + V_a^b(f)
\end{equation}
is a Banach space. 
\end{example}
\begin{proof}
\textbf{Step 1. }
Obviously, $\norm{f} \ge 0$ for every $f \in V[a, b]$ and $\norm{f} = 0$ 
if $f = 0$. Assuming that $\norm{f} = 0$, then for any $t \in [a, b]$, 
\begin{equation*}
    \abs{f(t)} \le \abs{f(t) - f(a)} + f(a) \le V_a^b(f) + f(a) = 0. 
\end{equation*}
The absolute homogeneousness is easy to verified. 

\textbf{Step 2. }
Let $f, g \in V[a, b]$. We need to show $\norm{f + g} = \abs{f(a) + g(a)} 
+ V_a^b(f + g) \le \norm{f} + \norm{g} = \abs{f(a)} + \abs{g(a)} + 
V_a^b(f) + V_a^b(g)$.
For any partition $T: a = x_0 < x_1 < \cdots < x_n = b$, 
\begin{equation*}
    \begin{aligned}
        V_T(f + g) &= \sum_{k = 1}^{n} 
            \abs{f(x_k) + g(x_k) - f(x_{k-1} - g(x_{k-1}))} \\
        &\le \sum_{k=1}^{n} \left[ \abs{f(x_k) - f(x_{k-1})} 
            + \abs{g(x_k) - g(x_{k-1})} \right] \\
        &= V_T(f) + V_T(g). 
    \end{aligned}
\end{equation*}
Therefore, $V_a^b(f + g) \le V_a^b(f) + V_a^b(g)$. 
Consequently, the function defined by Equation 
(\ref{equ:banach_spaces:norms:vab_norm}) is a norm. 

\textbf{Step 3. }
Let $\{ f_n \}_{n=1}^{\infty}$ be a Cauchy sequence in $V[a, b]$. 
Then for any $t \in [a, b]$, there exists $n, m \in \bN$ such that 
\begin{equation*}
    \begin{aligned}
        \abs{f_n(t) - f(t)} &= \abs{f_n(t) - f_m(t) - f_n(a) - f_m(a) 
            + f_m(a) + f_n(a)} \\
        &\le V_a^b(f_n - f_m) + \abs{f_m(a) - f_n(a)} = \norm{f_n - f_m}. 
    \end{aligned}
\end{equation*}
Therefore, for any $t \in [a, b]$, $\{ f_n(t)_n \}_{n=1}^{\infty}$ 
converges, to $f(t)$ say. 
We claim that $\{ f_n \}_{n=1}^{\infty}$ converges to $f$, which is a BV 
function. 
Indeed, for any $\epsilon > 0$, there exists $N \in \bN$ such that 
$\norm{f_n - f_m} < \epsilon$ for any $n, m > N$, whence for any partition 
$T: a = x_0 < x_1 < \cdots < x_p = b$, we have 
\begin{equation*}
    \abs{f_n(a) - f_m(a)} + V_T(f_n - f_m) < \epsilon. 
\end{equation*}
Letting $n \to \infty$ yields 
\begin{equation*}
    \abs{f(a) - f_m(a)} + V_T(f - f_m) \le \epsilon. 
\end{equation*}
It follows that for $m > N$, 
\begin{equation*}
    \abs{f(a) - f_m(a)} + V_a^b(f - f_m) \le \epsilon, 
\end{equation*}
which means $f \in V[a, b]$ (sum of BV functions is a BV function) 
$\lim_{n \to \infty} f_n = f$. 
\end{proof}

\begin{defn}
The linear space $X = C[a, b]$ along with the norm 
\begin{equation*}
    \norm{x} = \int_{a}^{b} \abs{x(t)} \diff t
\end{equation*}
is not complete. 
\end{defn}

\begin{prop}
Suppose that $\{x_n\}_{n=1}^\infty, \{y_n\}_{n=1}^\infty$ are sequences in 
a normed space $(X, \norm{\wdot})$ and $\{\alpha_n\}_{n=1}^\infty$ is a 
sequence in $\bK$. 
If $x_n \to x \in X$, $y_n \to y \in X$ and $\alpha_n \to \alpha$ as $n \to 
\infty$, then $x_n + y_n \to x + y$ and $\alpha_n x_n \to \alpha x$. 
\end{prop}

\begin{thm}
Let $X$ be a normed space and $\{x_n\}_{n=1}^\infty$ is a sequence in $X$. 
If $X$ is complete and series $\sum_{n=1}^\infty \norm{x_n}$ converges, 
then the series $\sum_{n=1}^\infty x_n$ converges and 
\begin{equation*}
    \norm{\sum_{n=1}^{\infty} x_n} \le \sum_{n}^{\infty} \norm{x_n}. 
\end{equation*}

Conversely, if $\sum_{n = 1}^{\infty} x_n$ converges always implies that 
$\sum_{n = 1}^{\infty} x_n$ converges, then $X$ is complete. 
\end{thm}
\begin{proof}
If $X$ is complete and $\sum_{n=1}^{\infty} \norm{x_n}$ is convergent, 
setting $S_N = \sum_{n=1}^{N}$, then it suffices to prove that $\{S_n\}$ 
is a Cauchy sequence, which is shown by 
\begin{equation*}
    \begin{aligned}
        \norm{S_m - S_n} &= \norm{x_{n+1} + x_{n+2} + \ldots + x_{m}} \\ 
        &\le \norm{x_{n+1}} + \norm{x_{n+2}} + \ldots + \norm{x_{m}} 
            \to 0, \quad \text{as } n, m \to \infty. 
    \end{aligned}
\end{equation*}

Suppose conversely, for every $\{ x_n \}_{n=1}^{\infty}$, convergence of 
$\sum_{n=1}^{\infty} \abs{x_n}$ implies $\sum_{n=1}^{\infty} x_n$. 
Assume that $\{ x_n \}_{n=1}^{\infty}$ is a Cauchy sequence in $X$. 
Then for every $i \in \bN$, there exists $N_i \in \bN$ such that 
\begin{equation*}
    \norm{x_n - x_m} < \frac{1}{2^i}, \quad \text{for any } n, m > N_i. 
\end{equation*}
Choosing $\{ n_k \}_{k=1}^{\infty}$ such that $n_k > N_k$ and 
$n_{k+1} > n_{k}$, then 
\begin{equation*}
    \norm{x_{n_{k+1}} - x_{n_k}} < \frac{1}{2^k}. 
\end{equation*}
Consider the series $x_{n_1} + \sum_{k=1}^{\infty} 
(x_{n_{k + 1}} - x_{n_k})$, which is absolutely convergent by the above 
Equation. 
By our assumption, $x_{n_1} + \sum_{k=1}^{\infty} (x_{n_{k + 1}} - x_{n_k})$ 
is convergent. 
As $x_{n_{m+1}} = x_{n_1} + \sum_{k = 1}^{m} (x_{n_{k+1}} - x_{n_{k}})$, 
$\{x_{n_k}\}_{k=1}^\infty$ converges to some $x \in X$. 
Therefore, by Theorem 
\ref{thm:metric_spaces:completeness:convergent_subsequence}, 
$X$ is complete. 
\end{proof}

\begin{rmk}
In general, absolute convergence in an arbitrary normed space does not 
imply convergence. 
\end{rmk}

%%%%%%%%%%%%%%%%%%%%%%%%%%%%%%%%%%%%%%%%%%%%%%%%%%%%%%%
%%  Section: Convex Sets
%%%%%%%%%%%%%%%%%%%%%%%%%%%%%%%%%%%%%%%%%%%%%%%%%%%%%%%
\section{Convex Sets}
\begin{defn}
\index{convex set}
Let $X$ be a linear space and $A \subseteq X$. 
$A$ is a \emph{convex set} if for all $x, y \in A$ and for all $t \in [0, 1]$, 
\begin{equation*}
    tx + (1-t) y \in A.
\end{equation*}
\end{defn}

About convex sets, we have a simple fact. 
\begin{prop}
The intersection of any class of convex sets remains convex. 
\end{prop}
Henceforce, it is fine to define the \emph{convex hull} of arbitrary set 
$A$ as the intersection of all convex sets containing $A$, denoted by 
$\co(A)$. 
Clearly, the convex hull of a set $A$ is the smallest set that contains $A$. 
\begin{figure}[ht]
    \centering
    \incfig{6cm}{functional}{convex_hull}
    \caption{Illustration of convex hull. }
\end{figure}

\begin{example}
The unit ball $B(x, 1)$ in a normed space $X$ is a convex set 
for any given $x \in X$. 
\end{example}

%%%%%%%%%%%%%%%%%%%%%%%%%%%%%%%%%%%%%%%%%%%%%%%%%%%%%%%
%%  Section: L^p Spaces
%%%%%%%%%%%%%%%%%%%%%%%%%%%%%%%%%%%%%%%%%%%%%%%%%%%%%%%
\section{$L^p$ Spaces}
Recall that given $p \in [1, \infty]$ and measure space $(X, \cF, \mu)$, 
space $L^p(X, \cF, \mu)$ consisting of all real valued measurable functions 
$f$ such that $\abs{f} ^p$ is integrable ($1 \le p < \infty$) or that is  
bounded a.e. (essentially bounded) ($p = \infty$), along with the norm 
\begin{equation*}
    \norm{f}_p = \left( \int_{X} \abs{f}^p \right) ^\frac{1}{p}, \quad 
    p \in [0, 1)
\end{equation*}
or 
\begin{equation*}
    \norm{f}_\infty = \esssup \norm{f} 
    \coloneqq \inf \left\{ a \in \bR_+: \mu(\abs{f} > a) = 0 \right\}
\end{equation*}
is a normed space. 
See Chapter \ref{chp:lp_spaces}. 
For the sake of conciseness, we simply write $L^p$ when it is unambiguous. 

\begin{thm}
The $L^p (1 \le p \le \infty)$ space is a Banach space.  
\end{thm}
\begin{proof}
We first deal with the case $p \in [0, \infty)$. 
Suppose that $\{ f_n \}_{n=1}^{\infty}$ is a Cauchy sequence in $L^p$. 
Then by definition of Cauchy sequence we can find a subsequence 
$\{ f_{n_k} \}_{k=1}^{\infty}$ of it such that 
\begin{equation*}
    \norm{f_{n_{i+1}} - f_{n_i}} < \frac{1}{2^i}, \quad i \in \bN^\ast.
\end{equation*}
For $k \in \bN^\ast$, put 
\begin{equation*}
    g_k = \abs{f_{n_1}} + \sum_{i=1}^{k} \abs{f_{n_{i+1}} - f_{n+i}} 
\end{equation*}
and 
\begin{equation*}
    g = \abs{f_{n_1}} + \sum_{i=1}^{\infty} \abs{f_{n_{i+1}} - f_{n+i}}. 
\end{equation*}
It follows from triangle inequality (Minkowski's inequality) that 
\begin{equation*}
    \norm{g_k}_p \le \norm{f_{n_1}}_p + 
        \sum_{i=1}^{k} \norm{f_{n_{k+1}} - f_{n_k}}_p 
    \le \norm{f_{n_1}}_p + 1. 
\end{equation*}
Since $g_k$ is monotonic and non negative, by Monotone Convergence Theorem 
we have 
\begin{equation*}
    \int_{X} g^p \diff x = \lim_{k \to \infty} \int_{X} g_k^p \diff x 
    \le \left( \norm{f_{n_1}} + 1 \right) ^p < \infty, 
\end{equation*}
which implies $g \in L^p$ and thus $g < \infty$ a.e.. 
Therefore, $\lim_{i \to \infty} f_{n_i} = f_{n_1} + \sum_{i=1}^{\infty}$ 
converges a.e.. 
Setting $f =  \lim_{i \to \infty} f_{n_i}$, it follows from 
\begin{equation*}
    \abs{f} = \lim_{k \to \infty} \abs{f_{n_1} 
        + \sum_{i=1}^{k} \left( f_{n_{i+1}} - f_{n_i} \right)} 
    \le \lim_{k \to \infty} g_k = g, \quad \text{a.e.}
\end{equation*}
that $f \in L^p$. 
Furhermore, 
\begin{equation*}
    \norm{f - f_{n_k}}_p = \norm{\sum_{i=k}^{\infty} 
        (f_{n_{i+1}} - f_{n_k})}_p 
    \le \sum_{i=k}^{\infty} \norm{ f_{n_{k+1}} - f_{n_k}}_p 
    \le \frac{1}{2^{k-1}} \to 0
\end{equation*}
as $k \to \infty$. 
In this way, we have found a convergent subsequence of $\{ f_n \}_{n=1}
^{\infty}$. 
Hence $L^p (1 \le p < \infty)$ is complete. 


\end{proof}

\begin{thm}
\begin{enumerate}
    \item For $p \in [0, \infty$, $L^p$ is separable. 
    \item $L^\infty$ is inseparable. 
\end{enumerate}
\end{thm}
\begin{proof}

\end{proof}

\section{Properties of Normed Spaces}
\begin{thm}
Every normed space is isometric to a dense subspace of a Banach space. 
\end{thm}
\begin{proof}
The process of completion of normed space is the same as of metric space, 
except that we have to make the set of all elements a vector space, by 
introducing addition and scalar multiplication to it. 
\end{proof}

Suppose that $X$ is a linear space and $M$ is a subspace of $X$. 
Define a relation $\sim$ in $X$ as $x \sim y \iff x - y \in M$. 
It is easy to see that $\sim$ is a equivalence relationship. 

\begin{defn}[Equivalent norms]
Let $X$ be a linear space. 
Two norms $\norm{\wdot}_1$, $\norm{\wdot}_2$ are said to be 
\emph{equivalent} if there exists $c_1$, $c_2 \in \bR_+$ such that 
\begin{equation*}
    c_1 \norm{x}_1 \le \norm{x}_2 \le c_2 \norm{x}_1, \quad 
    \text{for any } x \in X. 
\end{equation*}
\end{defn}
It is easy to see that equivalence of norms is a equivalence relation. 

%%%%%%%%%%%%%%%%%%%%%%%%%%%%%%%%%%%%%%%%%%%%%%%%%%%%%%%
%%  Section: Finite-dimensional Normed Spaces
%%%%%%%%%%%%%%%%%%%%%%%%%%%%%%%%%%%%%%%%%%%%%%%%%%%%%%%
\section{Finite-dimensional Normed Spaces}
\begin{thm}
The norms on a finite-dimensional normed space are equivalent. 
\end{thm}
\begin{proof}
Let $\norm{\wdot}$ be a norm on a $n$-dimensional linear space $X$. 
There exists $n$ linearly independent vectors $e_i$, $1 \le n \le n$ 
in $X$ such that $X = \mspan \left\{ e_1, e_2, \ldots, e_n \right\}$. 
Then it suffices to prove that $\norm{\wdot}$ on $X$ is equivalent to 
a specified norm $\norm{\wdot}_\infty$, which is defined as 
\begin{equation*}
    \norm{x}_\infty = \max_{1 \le i \le n} \abs{\alpha_i} 
\end{equation*}
for any $x = \sum_{i=1}^{n} \alpha_i e_i \in X$, $\alpha_i \in \bK$, 
$1 \le i \le n$. 
It is easy to show that $\norm{\wdot}_\infty$ is a norm on $X$. 

For $x = \sum_{i=1}^{n} \alpha_i e_i$, 
\begin{equation}
    \label{equ:banach_spaces:finite_dimensional:equivalent_norms:1}
    \begin{aligned}
        \norm{x} &= \norm{\sum_{i=1}^{n}\alpha_i e_i} 
        \le \sum_{i=1}^{n} \norm{\alpha_i e_i} \\ 
        &= \sum_{i=1}^{n} \abs{\alpha_i} \norm{e_i} 
        \le \max_{1 \le i \le n} \sum_{i=1}^{n} \norm{e_i} \
        = \norm{x}_\infty c_1, 
    \end{aligned}
\end{equation}
where $c_1 \coloneqq \sum_{i=1}^{n} \norm{e_i}$. 
Next, consider $f: \bK^n \to \bR$, $f(\alpha_1, \alpha_2, \ldots, \alpha_n) 
= \norm{\sum_{i=1}^{n} \alpha_i e_i}$. 
We claim that $f$ is continuous, since for for $(\alpha_1, \alpha_2, \ldots, 
\alpha_n)$, $(\beta_1, \beta_2, \ldots, \beta_n) \in \bK^n$, 
\begin{equation*}
    \begin{aligned}
        \abs{f(\alpha_1, \alpha_2, \ldots, \alpha_n) - 
            f(\beta_1, \beta_2, \ldots, \beta_n)} 
        &= \abs{\norm{\sum_{i=1}^{n} \alpha_i e_i} 
            - \norm{\sum_{i=1}^{n} \beta_i e_i}} \\ 
        &\le \norm{\sum_{i=1}^{n} \alpha_i e_i - 
            \sum_{i=1}^{n} \beta_i e_i} 
        = \norm{\sum_{i=1}^{n} (\alpha_i - \beta_i) e_i} \\
        &\le \sum_{i=1}^{n} \abs{\alpha_i - \beta_i} \norm{e_i} 
        \le \left( \max_{1 \le i \le n} \right) \sum_{i=1}^{n} \norm{e_i}. 
    \end{aligned}
\end{equation*}
Let $\Omega = \left\{ \left(\alpha_1, \alpha_2, \ldots, \alpha_n \right): 
\max_{1 \le i \le n} \abs{\alpha_i} = 1\right\}$, which is obviously is 
a bounded closed subset of $\bK^n$ and thus compact in $\bK^n$. 
This implies that $f$ attains its minimum on $\Omega$. 
Let $c_2 = \min_{(\alpha_1, \alpha_2, \ldots, \alpha_n) \in \Omega} 
\norm{\sum_{i=1}^{n} e_i}$. 
Then we have for any $x = \sum_{i=1}^{n} \alpha_i e_i$, 
\begin{equation}
    \label{equ:banach_spaces:finite_dimensional:equivalent_norms:2}
    \norm{x} = \max_{1 \le i \le n} \abs{\alpha_i} 
        \norm{\frac{x}{\max_{1 \le i \le n} \abs{\alpha_i}}}
    \ge c_2 \norm{x}_\infty.
\end{equation}
Combining, Equation (\ref{equ:banach_spaces:finite_dimensional:equivalent_norms:1}) 
and (\ref{equ:banach_spaces:finite_dimensional:equivalent_norms:2}), it 
follows that $\norm{\wdot}$ is equivalent to $\norm{\wdot}_\infty$. 
\end{proof}

Let $X$ be an $n$-dimensional normed space and define 
\begin{equation*}
    \begin{aligned}
        T: X &\to \bK^n,  \\ 
        x = \sum_{i=1}^{n} \alpha_i e_i &\mapsto 
            (\alpha_1, \alpha_2, \ldots, \alpha_n) 
    \end{aligned}
\end{equation*}
where $\left\{ e_1, e_2, \ldots, e_n \right\}$ is a basis of $X$. 
And we further equip $\bK^n$ with the standard Euclidean norm. 
Then by the theorem above, we know that $T$ is a topological homeomorphism 
from $X \to \bK^n$. 
In this sense, call $n$-dimensional normed spaces are topologically the same 
as $\bK^n$. 

We have already known that bounded closed sets in $\bK^n$ are compact, 
and by the theorem above, so does in finite-dimensional normed space. 
We can further show that this property is also a characterization of 
finite-dimensional normed space. 
Before stating the miraculous result, we need the following lemma: 
\begin{lemma}[F.Riesz's Lemma]
\label{lemma:banach_spaces:finite_dimensional:f_riesz_lemma}
Let $X_9$ be a proper closed subspace in a normed space $X$. 
Then for any $\epsilon > 0$, there exists $y_0 \in X$ such that 
$\norm{y_0} = 1$ and $\norm{y_0 - x} > 1 - \epsilon$ for every $x \in X_0$. 
\end{lemma}
\begin{proof}
As $X_0$ is a proper closed subsequence of $X$, we can find some point $y 
\in X \ X_0$ and thus $d(y, X_0) = \inf_{x \in X_0} \norm{y - x} > 0$. 
Setting $\epsilon > 0$, there exists $x_1 \in X_0$ such that $d(y, X_0) \le 
\norm{y - x_1} < \frac{d(y, X_0)}{1 - \epsilon}$. 
Set $y_0 = \frac{y - x_1}{\norm{y - x_1}}$. 
Then $\norm{y_0} = 1$ and for any $x \in X_0$, 
\begin{equation*}
    \begin{aligned}
        \norm{y_0 - x} 
        &= \norm{\frac{y-x_1}{\norm{y - x_1}} - x} 
        = \frac{1}{\norm{y - x_1}} \norm{y - \left( x_1 + \norm{y - x_1}x \right)} \\
        &\ge \frac{d}{\norm{y - x_1}} > \frac{d}{\frac{d}{1-\epsilon}} = 1 - \epsilon. 
    \end{aligned}
\end{equation*}
This completes the proof. 
\end{proof}

\begin{thm}
A normed space is finite-dimensional if and only if every bounded closed 
set in the space is compact. 
\end{thm}
\begin{proof}
The necessity is the famous Heine-Borel Theorem. 

For sufficiency, assume that every bounded closed subset in $X$ is compact 
while $X$ is infinite-dimensional. 
Let $S = \left\{ x \in X: \norm{x} = 1\right\}$. 
Then $S$ is bounded and closed (Why?). 
Firstly, choose some $x_1 \in S$ and let $X_0 = \mspan{x_1}$. 
Then $X_0$ is a proper closed subspace of $X$, whence there exists $x_2 \in 
S$ such that $\norm{x_2 - x_1} > \frac{1}{2}$ by Lemma 
\ref{lemma:banach_spaces:finite_dimensional:f_riesz_lemma}. 
Next, set $X_1 = \mspan \left\{ x_1, x_2\right\}$. 
$X_1$ is a proper closed subspace of $X$, whence there exists $x_3 \in 
S$ such that $\norm{x_2 - x_1} > \frac{1}{2}$, $\norm{x_3 - x_1} > 
\frac{1}{2}$ by Lemma \ref{lemma:banach_spaces:finite_dimensional:f_riesz_lemma}. 
Repeating this process, we obtain a sequence $\{ x_n \}_{n=1}^{\infty}$ 
such that $\norm{x_n} = 1$ and $\norm{x_n - x_m} > \frac{1}{2}$ for any 
$n, m \in \bN$, $n \neq m$. 
Therefore, $\{ x_n \}_{n=1}^{\infty}$ is not compact because any subsequence 
of $\{ x_n \}_{n=1}^{\infty}$ does not converges. 
However, $\{ x_n \}_{n=1}^{\infty}$ is a bounded and closed subset in $X$, 
which contradicting our hypothesis. 
\end{proof}
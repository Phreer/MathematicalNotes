\chapter{Generalized Functions}
\section{Background}
Generalized functions (also called distributions) are extensions to the 
classical notion of functions. Distribution theory reinterprets functions 
as linear functions acting on a space of \emph{test functions}. The 
different choices of the spaces of test functions lead to different spaces 
of distributions. Conventional functions act by integration against a test 
function, but many other linear functions do not arise in this way, and 
these are the generalized functions.

In physics, people find it convenient to introduce the so-called dirac 
$\delta$ to describe formally the distributions of physical quantities. 
The Dirac $\delta$ function is defined by 
\begin{equation}
\label{equ: dirac_cond1}
    \delta_y (x) = \begin{cases}
        \infty, & x = y \\
        0, & x \neq y, 
    \end{cases}
\end{equation}
along with 
\begin{equation}
\label{equ: dirac_cond2}
    \int _{-\infty} ^\infty \delta_y(x) \diff x = 1.
\end{equation}

However, no classical function satisfies conditions \ref{equ: dirac_cond1} 
and \ref{equ: dirac_cond2}. We can due with this problem with generalized 
functions.

\section{The space $D(\Omega)$}
First, let us introduce some concepts and notations.

\begin{defn}
Let a multi-index $\alpha = (\alpha_1, \alpha_2, \ldots, 
\alpha_n)$. Denote $\partial _j = \frac{\partial}{\partial_j}$, $j \in 
\bZ ^+_n$ and 
\begin{equation}
    \partial_\alpha = \partial_{\alpha_i} \partial_{\alpha_i}  \dots 
    \partial_{\alpha_n} u.
\end{equation}
For $\alpha = (0, 0, \ldots, 0)$, we set $\partial_\alpha$ the identity 
mapping.
\end{defn}

By $C^{\infty} _0(\Omega)$, we denote those functions in $C^\infty(\Omega)$ 
with compact support, \ie, those functions in $C^\infty(\Omega)$ vanishes 
outside a compact set of $\Omega$.

\begin{example}
A standard example of the space $C^{\infty} _0(\bR ^n)$ is 
\begin{equation}
    j(x) = \begin{cases}
        C_n \exp( - \frac{1}{1-\norm{x}^2 }) , & \norm{x} \le 1 \\
        0, & \norm{x} > 1, 
    \end{cases}
\end{equation}
where $C_n$ is defined as
\begin{equation}
    C_n = \left(\int _{\norm{x} \le 1} \exp(-\frac{1}{1 - \norm{x}^2})\diff 
    x\right) ^{-1}.
\end{equation}
\end{example}

\begin{defn}
Denote $D(\Omega)$ a space consisting of exactly the functions in 
$C_0^\infty(\Omega)$ along with the understanding of convergence as $\phi_n
\in D(\Omega)$ converges to $\phi \in D(\Omega)$ if the following 
requirements are satisfied:
\begin{enumerate}
    \item there exists a compact subset $K \in \Omega$ such that 
    \begin{equation}
        \phi(x) = 0, \text{for any } x \in \Omega \backslash K \text{ and 
        any } n, 
    \end{equation}
    \item for any fixed multi-index $\alpha$, $\partial _\alpha \phi_n$ 
    converges uniformly to $\partial _\alpha \phi$ on $K$, \ie
    \begin{equation}
        \max _{x\in K} \abs{\partial _\alpha \phi _n - \partial _\alpha \phi}
        \to 0 (n \to \infty).
    \end{equation}

\end{enumerate}
\end{defn}
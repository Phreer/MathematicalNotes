\chapter{Bounded Linear Operators}
The operator theory is one of the most essential topics in functional 
analysis. 
In this chapter, we will encounter the four ``great theorems'' in 
functional analysis, \ie, 
\begin{enumerate*}
    \item Banach-Steinhaus Theorem (Uniform Boundedness Theorem); 
    \item Banach Open Mapping Theorem; 
    \item Inverse Mapping Theorem; 
    \item Hahn-Banach Theorem. 
\end{enumerate*}

\section{Bounded Linear Operators and Bounded Linear Functionals} 
\begin{defn}[Linear operator, linear functional]
\index{linear functional}
\index{linear operator}
Let $X$ and $Y$ be two normed spaces and $D(T)$ be a subspace of $X$. 
A \emph{linear operator} $T$ from $D(T)$ to $Y$ is a mapping that is linear, 
\ie, 
\begin{enumerate}
    \item $T(x + y) = Tx + Ty$ for any $x, y \in D(T)$; 
    \item $T(\alpha x) = \alpha Tx$ for any $\alpha \in \bK$ and 
    $x \in D(T)$. 
\end{enumerate}
Denote by $R(T)$ the range of $T$. 

A linear operator $f: X \to Y$ is said to be a \emph{linear functional} 
If $Y = \bK$. 
\end{defn}

\begin{defn}[bounded linear operator]
Let $X$ and $Y$ be two normed spaces and $T: D(T) \subseteq X \to Y$. 
$T$ is said to be bounded if there exists a constant $M \in \bR_+$ such that 
$\norm{Tx} \le M \norm{x}$ for every $x \in D(T)$. 
\end{defn}

An linear operator $T: X \to Y$ is bounded if and only if there exists $M 
\in \bR_+$ such that $Tx \le M$ for every $x \in X$ satisfying $\norm{x} \le 
1$. 

\begin{prop}
If linear operator $T: X \to Y$ is continuous at some $x_0 \in X$, then $T$ 
is continuous on $X$. 
\end{prop}
\begin{proof}
This is trivial. 
\end{proof}

\begin{prop}
A linear operator $T: X \to Y$ is injective if and only if for every 
$x \in X$ such that $x \neq 0$, $Tx \neq 0$. 
\end{prop}

\begin{thm}
Let $X$ be a finite dimensional linear space and $T: X \to X$ be a linear 
operator. 
$T$ is a injective if and only if $T$ is surjective. 
\end{thm}
\begin{proof}
To be completed. 
\end{proof}
\begin{rmk}
However, an injective operator on infinite dimensional space may not be 
surjective. 
For example, consider $X = l^2$ and $T: X \to X, x = \left(x_1, x_2, 
\ldots \right) \mapsto \left(0, x_1, x_2, \ldots \right)$. 
Then obviously $T$ is injective but not surjective. 
\end{rmk}

\begin{thm}
A linear operator $T: X \to Y$ is continuous if and only if $T$ is bounded. 
\end{thm}
\begin{proof}
Supposing that $T$ is bounded, then $\norm{Tx} \le M\norm{x}$ implies the 
continuity of $T$. 

Conversely, supposing $T$ is continuous, then there exists sequence 
$\{ x_n \}_{n=1}^{\infty}$ such that $\norm{Tx_n} > n\norm{x_n}$ for 
$n \in \bN^\ast$. 
Setting $y_n = \frac{x_n}{n \norm{x_n}}$, $n \in \bN^\ast$, then $\norm{y_n} 
= \frac{1}{n}$. 
Thus $y_n \to 0$ as $n \to \infty$. 
However, $\norm{Ty_n} = \frac{\norm{Tx_n}}{n \norm{x_n}} > 1$, whence $Ty_n$ 
does not converges to $T0$, contradicting the assumption that $T$ is 
continuous.
\end{proof}

\begin{defn}[operator norm]
\index{operator norm}
Let $T$ be a bounded linear operator from $X$ to $Y$. 
Define 
\begin{equation*}
    \norm{T} = \sup_{\substack{x \in X \\ x \neq 0}} \frac{\norm{Tx}}{\norm{x}}.
\end{equation*}
We call $\norm{T}$ the operator norm of $T$.
\end{defn}
It can be proven that 
\begin{equation*}
    \norm{T} = \inf \left\{ M \in \bR_{+}: \norm{Tx} \le M \norm{x} 
        \text{ for every } x \in X \right\}.
\end{equation*}
Moreover, 
\begin{equation*}
    \norm{T} = \sup_{\norm{x} = 1} \norm{Tx} 
    = \sup_{\norm{x} \le 1} \norm{Tx}. 
\end{equation*}
Indeed, on the one hand, by definition 
\begin{equation*}
    \begin{aligned}
        \norm{T} &= \sup_{x \neq 0} \frac{\norm{Tx}}{\norm{x}} 
        \ge \sup_{\norm{x} \le 1} \frac{\norm{Tx}}{\norm{x}} 
        &\ge \sup_{\norm{x} \le 1} \norm{Tx} 
        \ge \sup_{\norm{x} = 1} \norm{Tx}. 
    \end{aligned}
\end{equation*}
On the other hand, for $x \neq 0$, 
\begin{equation*}
    \frac{\norm{Tx}}{\norm{x}} = \norm{T\left( \frac{x}{\norm{x}} \right)} 
    \le \sup_{\norm{x} = 1} \norm{Tx}, 
\end{equation*}
implying that $\norm{T} = \sup_{x \neq 0} \frac{\norm{Tx}}{\norm{x}} 
\le \sup_{\norm{x} = 1} \norm{Tx}$. 

\begin{example}
Consider the $n$-dimensional Euclidean space $\bR^n$. 
Let $A = (a_{ij})_{n \times n}$ be an $n \times n$ matrix and define $T_A: X 
\to X, x \mapsto y$ where $y$ is obtained by matrix multiplication $Ax$. 
Then $A$ is a bounded as by conventional Cauchy-Schwarz's Inequality 
(\ref{equ:inequality:cauchy_schwarz_inequality_rn}), 
\begin{equation*}
    \begin{aligned}
        \norm{T_A x} &= \left( \sum_{i=1}^{n} 
            \abs{\sum_{j=1}^{n} a_{ij}x_j}^2 \right)^\frac{1}{2} 
        \le \left( \sum_{i=1}^{n} 
                \left( \sum_{j=1}^{n} \abs{a_{ij}}^2 
                    \sum_{j=1}^{n} \abs{x_{j}}^2 \right) 
            \right) \frac{1}{2} \\
        &= \norm{x}_2 
            \left( \sum_{i=1}^{n} \sum_{j=1}^{n} 
            \abs{a_{ij}}^2 \right) ^\frac{1}{2}. 
    \end{aligned}
\end{equation*}
\end{example}

\begin{example}
\label{ex:bounded_linear_operators:integral_functional}
Let $g \in C[a, b]$ and define $T: C[a, b] \to \bK, f \mapsto 
\int_{a}^{b} f(t)g(t) \diff t$. 
Then $T$ is a linear functional on $C[a, b]$ and 
\begin{equation*}
    \abs{Tf} \le \int_{a}^{b} \abs{f g} \diff t 
    \le \max_{a \le t \le b} \abs{f(t)} \int_{a}^{b} \abs{g} \diff t 
    = \norm{f} \int_{a}^{b} \abs{g} \diff t. 
\end{equation*}
Thus $T$ is bounded and $\norm{T} \le \int_{a}^{b} \abs{g} \diff t$. 
We next show that $\norm{T} = \int_{a}^{b} \abs{g} \diff t$. 
\begin{proof}
Letting $h(t) = \sgn g(t) = \frac{g(t)}{\abs{g(t)}}$ if $g(t) \neq 0$ 
else $0$, then we have $\abs{h(t)} \le 1$ and $\conj{h(t)} g(t) = 
\abs{g(t)}$ for every $t \in [a, b]$. 
By Luzin's Theorem, for any $\epsilon > 0$, there exists $f \in C[a, b]$ 
such that $\norm{f} \le 1$ and $m(\Omega) \le \epsilon$ where $m$ is the 
Lebesgue measure and 
\begin{equation*}
    \Omega = \{t \in [a, b] : f(t) \neq \conj{h(t)} \}.
\end{equation*}
Then 
\begin{equation*}
    \begin{aligned}
        \norm{T} \ge \abs{Tf} &= \abs{\int_{a}^{b} f g \diff t} 
        = \abs{\int_{a}^{b} \conj{h} g \diff t 
            + \int_{a}^{b} (f - \conj{h})g \diff t} \\
        &\ge \int_{a}^{b} \abs{g} \diff t 
            - \int_{a}^{b} \abs{f - \conj{h}} \abs{g} \diff t 
        \ge \int_{a}^{b} \abs{g} \diff t - 2 \epsilon \norm{g}. 
    \end{aligned}
\end{equation*}
Since $\epsilon$ is arbitrary, we arrive at $\norm{T} \ge
\int_{a}^{b} g \diff t$, which completes the proof.
\end{proof}
\end{example}

\begin{example}
Let $X = l^p$ and $Y = l^q$ where $p, q$ are conjugate indices. 
Define $T: X \to Y, x = \left(x_1, x_2, \ldots \right) \mapsto 
y = \left(y_1, y_2, \ldots \right)$, where $y_i = \sum_{j=1}^{\infty} 
a_{ij} x_j$ and the infinite matrix $A = (a_{ij})_{1 \le i, j < \infty}$ 
satisfies 
\begin{equation*}
    \sum_{i=1}^{\infty} \sum_{j=1}^{\infty} \abs{a_{ij}}^q < \infty.
\end{equation*}
We shall prove $\norm{T} \ge \left( \sum_{i=1}^{\infty} \sum_{j=1}^{\infty} 
\abs{a_{ij}}^q \right) ^\frac{1}{q}$. 
\end{example}
\begin{proof}
By Hölder's Inequality (\ref{equ:inequalities:holder_inequality}), 
\begin{equation*}
    \begin{aligned}
        \norm{Tx}^q = \sum_{i=1}^{\infty} 
            \abs{\sum_{j=1}^{\infty} a_{ij} x_j}^q 
        &\le \sum_{i=1}^{\infty} \left( 
                \left( \sum_{j=1}^{\infty} \abs{a_{ij}}^q \right)^\frac{1}{q}
                \left( \sum_{j=1}^{\infty} \abs{x_j}^p \right)^\frac{1}{p}
            \right) ^q \\ 
        &= \sum_{i=1}^{\infty} \left( 
            \left( \sum_{j=1}^{\infty} \abs{a_{ij}}^q \right)^\frac{1}{q}
        \right) ^q \norm{x}_p ^q.
    \end{aligned}
\end{equation*}
This implies $\norm{Tx} \le \left( \sum_{i=1}^{\infty} \sum_{j=1}^{\infty} 
\abs{a_{ij}}^q \right) ^\frac{1}{q} \norm{x}_p$.
\end{proof}

\begin{example}
Let $T: C^1[a, b] \to C[a, b], f \mapsto \frac{\diff f}{\diff t}$. 
We point out that $T$ is unbounded. 
Indeed, letting $g_n = \sin(nt)$, then $\norm{g_n} \le 1$ but 
\begin{equation*}
    \norm{\frac{\diff g_n}{\diff t}} = \norm{n \cos nt} = n \to \infty 
\end{equation*}
as $n \to \infty$. 
\end{example}

\begin{example}
\label{ex:bounded_linear_operators:integral_kernel}
Let $K(s, t)$ be a continuous function on $[a, b] \times [a, b]$ and define 
$T: C[a, b] \to C[a, b], (Tf)(t) = \int_{a}^{b} K(t, s) f(s) \diff s$. 
We claim that 
\begin{equation*}
    \norm{T} = \max_{a \le t \le b} \int_{a}^{b} \abs{K(t, s)} \diff s. 
\end{equation*}
Firstly, for $f \in C[a, b]$, 
\begin{equation*}
    \begin{aligned}
        \abs{(Tf)(t)} &= \abs{\int_{a}^{b} K(t, s) f(s) \diff s} 
        \le \int_{a}^{b} \abs{K(t, s) f(s)} \diff s \\
        &\le \norm{f} \int_{a}^{b} \abs{K(t, s)} \diff s.
    \end{aligned}
\end{equation*}
Thus, $\norm{Tf} \le \norm{f} \max_{t \in [a, b]} \int_{a}^{b} 
\abs{K(t, s)} \diff s$ and therefore $\norm{T} \le \max_{t \in [a, b]} 
\int_{a}^{b} \abs{K(t, s)} \diff s$. 
Letting $\beta = \max_{t \in [a, b]} \int_{a}^{b} \abs{K(t, s)} \diff s$, 
As $\int_{a}^{b} \abs{K(t, s)} \diff s$ is continuous on $[a, b]$, 
$t_0 = \argmax_{t \in [a, b]} \int_{a}^{b} \abs{K(t, s)} \diff s$ exists. 
Hence, 
\begin{equation*}
    \norm{Tf} \ge \abs{(Tf)(t_0)} = \int_{a}^{b} K(t_0, s) f(s) \diff s 
    = \abs{Af}
\end{equation*}
where $A \coloneqq \int_{a}^{b} K(t_0, s) f(s) \diff s$ is a continuous 
linear functional. 
And by Example \ref{ex:bounded_linear_operators:integral_functional}, 
\begin{equation*}
    \norm{A} = \int_{a}^{b} \abs{K(t_0, s)} \diff s.
\end{equation*}
Therefore, $\norm{T} \ge \norm{A} = \int_{a}^{b} \abs{K(t_0, s)} \diff s = 
\max_{t \in [a, b]} \int_{a}^{b} \abs{K(t, s)} \diff s$. 
\end{example}

Now we present some classical results on finite dimensional normed space. 
Let $A = (a_{ij})_{m \times n}$ be an $m \times n$ matrix and $X = \bK^n$, 
$Y = \bK^m$ equipped with $p$-norm. 
Define $T: X \to Y, x \mapsto Ax$. 
We want to study $\norm{A}_p \coloneqq \sup_{x \neq 0} 
\frac{\norm{Ax}}{\norm{x}}$ for $p = 1, 2, \infty$. 
\begin{thm}
It holds true that
\begin{enumerate}
    \item \label{thm:bounded_linear_operator:matrix_norms:p2}
    $\norm{A}_2 = \left( \rho_1(A^\ast A) \right) ^\frac{1}{2} 
    = \sigma_1(A)$ where $\rho_i(A)$ (\resp, $\sigma(A)$) is the $i$-th largest eigenvalue 
    (\resp, singular value) of matrix $A$. 
    \item \label{thm:bounded_linear_operator:matrix_norms:p1}
    $\norm{A}_1 = \norm{A}_{1, \infty} \coloneqq \max_{1 \le j \le n} \sum_{i=1}^{n} \abs{a_{ij}}$. 
    \item \label{thm:bounded_linear_operator:matrix_norms:p_infinity}
    $\norm{A}_\infty = \max_{1 \le i \le m} \sum_{j=1}^{n} \abs{a_{ij}}$. 
\end{enumerate} 
\end{thm}
\begin{proof}
Assume that $A$ is not a zero matrix. 

\ref{thm:bounded_linear_operator:matrix_norms:p2}
As we know $A^\ast A$ is a hermitian and positive semi-definite matrix, 
there exists $\left\{ e_1, e_2, \ldots, e_3 \right\} \subseteq \bK^n$ and 
$\lambda_1 \ge \lambda_2 \ge \ldots \ge\lambda_n \ge 0$ such that 
$A^\ast A e_i = \lambda e_i, \quad 1 \le i \le n$ and $e_1, e_2, \ldots, 
e_n$ are orthogonal and normalized. 

For $x \in \bK^n$, we have the decomposition $x = \sum_{i=1}^{n} c_i e_i$ 
for some $c_1, c_2, \ldots, c_n \in \bK$ and $\norm{x}_2 = \left( 
\sum_{i=1}^{n} \abs{c_i}^2 \right) ^\frac{1}{2}$. 
Then $\norm{Ax}_2 = (Ax)^\ast (Ax) = x^\ast A^\ast A x = \sum_{i=1}^{n} 
\lambda_i \abs{c_i}^2$. 
It follows that 
\begin{equation*}
    \frac{\norm{Ax}_2}{\norm{x}_2} = \left( \frac{\sum_{i=1}^{n} \lambda_i 
    \abs{c_i}^2}{\sum_{i=1}^{n}\abs{c_i}^2} \right)^\frac{1}{2} 
    \le \sqrt{\lambda_1}. 
\end{equation*}
In the meantime, the above inequality can be achieved if $c_i = 0$ for all 
$i \in \{2, 2, \ldots, n\}$, this completes the proof of 
(\ref{thm:bounded_linear_operator:matrix_norms:p2}). 

\ref{thm:bounded_linear_operator:matrix_norms:p1}
For $x \in \bK^n$, 
\begin{equation*}
    \begin{aligned}
        \norm{Ax}_1 &= \sum_{i=1}^{m} \abs{(Ax)_i} 
        = \sum_{i=1}^{m} \abs{\sum_{j=1}^{n} a_{ij} x_j} \\ 
        &\le \sum_{i=1}^{m} \sum_{i=j}^{n} \abs{a_{ij}} \abs{x_j} 
        = \sum_{j=1}^{n} \abs{x_j} \sum_{i=1}^{m} \abs{a_{ij}} 
        \le \norm{A}_{1, \infty} \norm{x}_1, 
    \end{aligned}
\end{equation*}
and this achieves equality for $x = e_{j_0}$ whose $i$-th entry is equal to 
$1$ while others are equal to $0$.
This implies $\norm{A}_1 \le \norm{A}_{1, \infty}$. 

\ref{thm:bounded_linear_operator:matrix_norms:p_infinity}
Firstly, for $x \in \bK^n$, 
\begin{equation*}
    \norm{Ax}_\infty = \max_{1 \le i \le m} \abs{\sum_{j=1}^{n} a_{ij} x_j} 
    \le \max_{1 \le i \le m} \sum_{j=1}^{n} \abs{a_{ij}} \abs{x_j} 
    \le \left( \max_{1 \le i \le m} \sum_{j=1}^{n} \abs{a_{ij}} \right) 
        \norm{x}_\infty, 
\end{equation*}
which implies $\norm{A}_\infty \le \max_{1 \le i \le m} \sum_{j=1}^{n} 
\abs{a_{ij}}$. 
And the inequality above achieves equality for $x = \left(x_1, x_2, 
\ldots, x_n \right)$ such that 
\begin{equation*}
x_j = \conj{(\sgn a_{i_0j})}
\end{equation*} 
where $i_0 = \argmax_{1 \le i \le m} \sum_{j=1}^{n} \abs{a_ij}$. 
Indeed, 
\begin{equation*}
    \begin{aligned}
        \norm{Ax}_\infty = \max_{1 \le i \le m} \abs{\sum_{j=1}^{n} a_{ij} x_j} 
        \ge \sum_{j=1}^{n} \abs{a_{i_0j} x_j} 
        = \sum_{j=1}^{n} \abs{a_{i_0j}} 
        \ge \sum_{j=1}^{n} \abs{a_{i_0j}} \norm{x}_\infty.
    \end{aligned}
\end{equation*}
This completes our proof. 
\end{proof}

Let $X, Y$ be two normed space. 
Denote by $\sB(X, Y)$ the set of all bounded linear operators from $X$ to 
$Y$. 
Define for $A, B \in \sB(X, Y)$, $x \in X$ and $\alpha \in \bK$, 
\begin{equation*}
    \begin{aligned}
        (A + B)(x) = Ax + Bx, \quad
        (\alpha A)(x) = \alpha (Ax). 
    \end{aligned}
\end{equation*}
Then $\sB(X, Y)$ is a linear space. 
For the sake of simplicity, if $X = Y$, we write $\sB(X)$ for $\sB(X, X)$. 

\begin{thm}
Equipped with the operator norm $\norm{\wdot}$, $\sB (X, Y)$ is a normed 
space. 
\end{thm}
\begin{proof}
It suffices to prove that $\norm{\wdot}$ is a norm on $\sB(X, Y)$. 
Here we only verify the triangle inequality since the other two condition 
is obvious. 
For $A, B \in \sB(X, Y)$, 
\begin{equation*}
    \begin{aligned}
        \norm{A + B} &= \sup_{\norm{x} = 1} \norm{Ax + Bx} 
        \le \sup_{\norm{x} = 1} \left( \norm{Ax} + \norm{Bx} \right) \\
        &\le \sup_{\norm{x} = 1} \norm{Ax} + \sup_{\norm{x} = 1} \norm{Bx} 
        = \norm{A} + \norm{B}. 
    \end{aligned}
\end{equation*}
\end{proof}

\begin{thm}
\label{thm:linear_operators:completeness_bounded_linear_spaces_operator_norm}
Let $X, Y$ be two normed space. 
If $Y$ is a Banach space, then $\cB(X, Y)$ is a Banach space. 
\end{thm}
\begin{proof}
Let $\{ A_n \}_{n=1}^{\infty}$ be a Cauchy sequence in $B(X, Y)$. 
Then for any $\epsilon > 0$, there exists $N \in \bN^\ast$ such that 
for $n, m > N$, 
\begin{equation}
    \label{equ:linear_operators:banach_space_bounded_linear_space:1}
    \norm{A_n - A_m} < \epsilon
\end{equation}
which implies 
\begin{equation*}
    \begin{aligned}
        \norm{A_n x - A_m x} &= \norm{(A_n - A_m)x} 
        \le \norm{A_n - A_m} \norm{x} \\ 
        &< \epsilon \norm{x}, \quad \text{for any } x \in X. 
    \end{aligned}
\end{equation*}
Thus $\{ A_nx \}_{n=1}^{\infty}$ is a Cauchy sequence in $Y$ for every $x \in X$. 
Since $Y$ is a Banach space, these Cauchy sequences converges. 
Denote $Ax = \lim_{n \to \infty} A_n x$ for every $x \in X$. 
Clearly, $A$ is a linear operator from $X$ to $Y$. 
By Equation 
(\ref{equ:linear_operators:banach_space_bounded_linear_space:1}), there 
for $\norm{x} = 1$, 
\begin{equation*}
    \norm{A_nx - A_m x} < \epsilon, \quad \text{for any } n, m > N. 
\end{equation*} 
Letting $m \to \infty$ yields for $\norm{x} = 1$, 
\begin{equation*}
    \norm{A_n x - Ax} \le \epsilon, \quad \text{for any } n > N, 
\end{equation*}
which means $\norm{A_n - A} \to 0$ as $n \to \infty$, $A_n - A \in B(X, Y)$ 
and thus $A \in \sB(X, Y)$. 
\end{proof}

In particular, the set $\sB(X, \bK)$ of all bounded linear functionals on 
$X$ is denoted by $X^\ast$, which is also known as the \emph{dual space} or 
\emph{conjugate space} of $X$. 
Since $\bK$ is complete for the case $\bK = \bR$ or $\bK = \bC$, the dual 
space of $X$ is always a Banach space. 

\begin{defn}
Let $\{ A_n \}_{n=1}^{\infty}$ be a sequence in $\sB(X, Y)$ and $A \in 
\sB(X, Y)$. 
If $\lim_{n \to \infty} \norm{A_n - A} = 0$, then we say that $\{ A_n \}
_{n=1}^{\infty}$ \emph{converges uniformly} to $A$. 
If for every $x \in X$, 
\begin{equation*}
    \lim_{n \to \infty} A_n x = Ax, 
\end{equation*}
then we say that $\{ A_n \}_{n=1}^{\infty}$, then we say $\{ A_n \}
_{n=1}^{\infty}$ \emph{converges strongly} to $A$. 
\end{defn}
\begin{rmk}
It is not hard to verify that uniform convergence implies strong convergence. 
The converse holds true in finite dimensional space, but false in infinite 
dimensional space. 
For instance, let $X = Y = l^p (p \ge 1)$ and for all $n \in \bN^\ast$, 
\begin{equation*}
    A_n x = (x_n, x_{n+1}, \ldots), \quad 
    \text{for any } x = (x_1, x_2, \ldots) \in X. 
\end{equation*}
For every $x \in X$ and $\epsilon > 0$, there exists $N \in \bN^\ast$ 
such that 
\begin{equation*}
    \sum_{k = n}^{\infty} \abs{x_k}^p < \epsilon, \quad 
    \text{for any } n > N. 
\end{equation*}
Therefore, for any $n > N$ and $x \in X$, 
\begin{equation*}
    \norm{A_nx} = \left( \sum_{k = n}^{\infty} \abs{x_k} \right)^\frac{1}{p} 
    < \epsilon ^\frac{1}{p}, 
\end{equation*}
implying that $\{ A_n x \}_{n=1}^{\infty}$ converges strongly to zero. 
However, since $\norm{A_n} = 1$ for all $n \in \bNs$, sequence 
$\{ A_n \}_{n=1}^{\infty}$ would never converges to zero operator. 
\end{rmk}


%%%%%%%%%%%%%%%%%%%%%%%%%%%%%%%%%%%%%%%%%%%%%%%%%%%%%%%
%%  Section: Banach-Steinhaus Theorem
%%%%%%%%%%%%%%%%%%%%%%%%%%%%%%%%%%%%%%%%%%%%%%%%%%%%%%%
\section{Banach-Steinhaus Theorem}
\begin{thm}[Banach-Steinhaus Theorem, Uniform Bounded Principle, 
    The Resonance Theorem]
\index{theorem!Banach-Steinhaus \~{}}
\index{theorem!The Resonance \~{}}
\index{Uniform Bounded Principle}
\label{thm:banach_steinhaus_theorem}
Let $X$ be a Banach space and $Y$ be a normed space. 
Suppose that $I$ is a index set and $\left\{ T_\alpha: 
\alpha \in I \right\}$ is a subset of $\sB(X, Y)$. 
If for every $x \in X$, 
\begin{equation*}
    \sup_{\alpha \in I} \norm{T_\alpha x} < \infty, 
\end{equation*}
then $\{ T_n \}_{n=1}^{\infty}$ is uniformly bounded, \ie, 
\begin{equation*}
    \sup_{\alpha \in I} \norm{T _\alpha} < \infty.
\end{equation*}
\end{thm}
\begin{proof}
Define 
\begin{equation*}
    p(x) = \sup_{\alpha \in I} \norm{T_\alpha x}, \quad x \in X. 
\end{equation*}
It is not difficult to verify that for every $x, y \in X$ and $\lambda \in 
\bK$, 
\begin{enumerate}
    \item $p(x) < \infty$; 
    \item $p(\lambda x) = \abs{\lambda} p(x)$; 
    \item $p(x + y) \le p(x) + p(y)$. 
\end{enumerate}
For every $k \in \bN$, let 
\begin{equation*}
    \begin{aligned}
        M_k = \left\{ x \in X: p(x) \le k \right\} 
        = \bigcap_{\alpha \in I} 
            \left\{ x \in X: \norm{T_\alpha x} \le k \right\}. 
    \end{aligned}
\end{equation*}
As $T_\alpha$ is continuous on $X$, 
\begin{equation*}
    \left\{ x \in X: \norm{T_\alpha} \le k \right\}
\end{equation*}
is closed in $X$ and consequently, $M_k$ is closed in $X$. 

Noticing that 
\begin{equation*}
    \bigcup_{k \in \bN} M_k = X, 
\end{equation*}
by Baire's Category Theorem, $X$ is the second category set, since $X$ is a 
Banach space. 
Thus there exists some $k_0$ and nonempty closed ball $\overline{B} = 
\overline{B(x_0, r)}$ such that $\overline{B} \subseteq M_{k_0}$. 

For any $x \in X$, $x_0 \pm \frac{x}{\norm{x}} r_0 \in \overline{B}$. 
Hence, 
\begin{equation*}
    \begin{aligned}
        p(\frac{2r_0 x}{\norm{x}}) &= p\left( 
            \left( x_0 + \frac{x}{\norm{x}} r_0 \right) 
            - \left( x_0 - \frac{x}{\norm{x}} r_0 \right) \right) \\
        &\le p \left( x_0 + \frac{x}{\norm{x}} r_0 \right) 
            + p \left( x_0 - \frac{x}{\norm{x}} r_0 \right) 
        \le 2k_0, 
    \end{aligned}
\end{equation*}
which implies 
\begin{equation*}
    \begin{aligned}
        & p(x) \le \frac{2k_0}{2r_0} \norm{x} = \frac{k_0}{r_0} \norm{x} \\
        & \implies
        \norm{T_\alpha x} \le \frac{k_0}{r_0} \norm{x}, \quad 
            \text{for any } \alpha \in I \\
        & \implies
        \norm{T_\alpha} \le \frac{k_0}{r_0}, \quad 
            \text{for any } \alpha \in I \\
        & \implies \sup_{\alpha \in I} \norm{T_\alpha} < \infty. 
    \end{aligned}
\end{equation*}
This completes the proof. 
\end{proof}

Banach-Steinhaus Theorem is one of the most essential theorems in functional 
analysis. 
Here are some applications of it. 

\begin{thm}
\label{thm:bounded_linear_operators:strong_convergence_uniformly_bounded_dense}
Let $X$ be a normed space and $Y$ be a Banach space. 
Suppose that $\{ T_n \}_{n=1}^{\infty}$ is a sequence in $\sB(X, Y)$ which 
satisfies 
\begin{enumerate}
    \item $\{ T_n \}_{n=1}^{\infty}$ is uniformly bounded, \ie, 
    $\sup_{n \in \bNs} \norm{T_n} < \infty$; 
    \item there exists a dense subset $G$ of $X$ such that $\{ T_n x \}
    _{n=1}^{\infty}$ converges on $G$ (\ie, for all $x \in G$). 
\end{enumerate}
Then $\{ T_n \}_{n=1}^{\infty}$ converges strongly to some $T \in 
\sB(X, Y)$ and 
\begin{equation*}
    \norm{T} \le \liminf_{n \to \infty} \norm{T_n}. 
\end{equation*}
\end{thm}
\begin{proof}
Let $\epsilon > 0$ and 
\begin{equation*}
    M = \sup_{n \in \bNs} \norm{T_n} < \infty. 
\end{equation*}
Since $G$ is dense, for every $x \in X$, there exists $y \in G$ such that 
$\norm{y - x} < \frac{\epsilon}{3M}$. 
By assumption, there exists $N \in \bNs$ such that for every $n, m > N$, 
\begin{equation*}
    \norm{T_n y - T_m y} < \frac{\epsilon}{3}.
\end{equation*}
It follows from 
\begin{equation*}
    \begin{aligned}
        \norm{T_n x - T_m x} &\le \norm{T_n x - T_n y} 
            + \norm{T_n y - T_m y} + \norm{T_m x - T_m y} \\
        &< \norm{T_n} \norm{x - y} + \frac{\epsilon}{3} 
            + \norm{T_m} \norm{x - y} 
        \le 2 M \norm{x - y} + \frac{\epsilon}{3} < \epsilon 
    \end{aligned}
\end{equation*}
that $\{ T_n x \}_{n=1}^{\infty}$ is a Cauchy sequence, whence it converges 
to some element of $X$, since $Y$ is a Banach space. 
Denoting 
\begin{equation*}
    Tx = \lim_{n \to \infty} T_n x, 
\end{equation*}
it is clear that $T$ is a linear operator. 
Also, 
\begin{equation*}
    \norm{Tx} = \lim_{n \to \infty} \norm{T_n x} 
    \le \liminf_{n \to \infty} \norm{T_n} \norm{x}, 
\end{equation*}
implying that $\norm{T} \le \liminf_{n \to \infty} \norm{Tn}$. 
\end{proof}

In Theorem 
\ref{thm:linear_operators:completeness_bounded_linear_spaces_operator_norm}, 
we have illustrated that if $Y$ is a Banach space, then $\sB(X, Y)$ is 
complete in the sense of uniform convergence. 
As an application of the Banach-Steinhaus Theorem, if we further assume that 
$X$ is also a Banach space, then $\sB(X, Y)$ is complete in the sense of 
strong convergence. 
\begin{thm}
Let $X, Y$ be two Banach spaces. 
Then $\sB(X, Y)$ is complete in the sense of strong convergence. 
\end{thm}
\begin{proof}
Let $\left\{ T_n \right\}$ be such that 
\begin{equation*}
    \lim_{n \to \infty} T_n x
\end{equation*}
exists for every $x \in X$. 
Then $\sup_{n \in \bNs} \norm{T_n x} < \infty$ for every $x \in X$. 
By the Banach-Steinhaus Theorem \ref{thm:banach_steinhaus_theorem}, 
$\sup_{n \in \bNs} \norm{T_n} < \infty$. 
And by the theorem above, there $\{T_n\}$ converges strongly to some linear 
bounded operator from $X$ to $Y$. 
\end{proof}

Now we apply the Banach-Steinhaus Theorem to prove the convergence of 
quadrature formula of integral. 
In numerical analysis, given a function on $[a, b] \subseteq \bR$ and 
$\alpha_{nk} \in \bK, t_{nk} \in \bR, k \in \{ 1, 2, \ldots, n \}, 
n \in \bNs$ such that $a \le t_{n1} \le t_{n2} \le \cdots \le t_{nn} \le b$, 
the quadrature formula 
\begin{equation*}
    \sum_{k=1}^{n} \alpha_{nk} f(t_{nk}) 
\end{equation*}
is widely used to approximate the integral $\int_{a}^{b} f(t) \diff$. 
It remains a problem that how can we choose coefficients $\alpha_{nk}$ 
such that 
\begin{equation*}
    A_n(f) = \sum_{k=1}^{n} \alpha_{nk} f(t_{nk}) 
\end{equation*}
converges to $\int_{a}^{b} f(t) \diff t$ as $n \to \infty$. 

\begin{thm}
The quadrature formula converges for every every $f \in C[a, b]$, \ie, 
\begin{equation*}
    \lim_{n \to \infty} A_n(f) = \int_{a}^{b} f(t) \diff t
\end{equation*}
if and only if the following conditions hold true: 
\begin{enumerate}
    \item \label{enu:bounded_linear_operator:quadrature_formula:1}
    there exists $M > 0$ such that $\sum_{k=1}^{k} 
    \abs{\alpha_{nk}} \le M$; 
    \item \label{enu:bounded_linear_operator:quadrature_formula:2}
    for every polynomial function $p$, $\{ A_n(p) \}_{n=1}^{\infty}$ 
    converges to $\int_{a}^{b} p(t) \diff t$. 
\end{enumerate}
\end{thm}
\begin{proof}
($\Leftarrow$ )
Let $X = C[a, b]$, $Y = \bK$ and define 
\begin{equation*}
    Tf = \int_{a}^{b} f(t) \diff t, \quad 
    T_n f = \sum_{k=1}^{n} \alpha_{nk} f(t_{nk}) 
\end{equation*}
for every $n \in \bNs$. 
Observing that $\norm{T_n} = \sum_{k=1}^{n} \abs{\alpha_{nk}} < M$ 
for every $n \in \bN$ and polynomial functions are dense in $C[a, b]$, 
it follows from Theorem 
\ref{thm:bounded_linear_operators:strong_convergence_uniformly_bounded_dense}
that $\{ T_n \}_{n=1}^{\infty}$ converges strongly to $T$. 

($\Rightarrow$)
Assuming that $\{ T_n \}_{n=1}^{\infty}$ converges strongly on $C[a, b]$, 
$\{ T_n x \}_{n=1}^{\infty}$converges bounded for every $x \in X$. 
Thus $\{ \norm{T_n x} = \sum_{k=1}^{n} \abs{\alpha_{nk}} \}_{n=1}^{\infty}$ 
is bounded for every $x \in C[a, b]$. 
By the Banach-Steinhaus Theorem, condition 
\ref{enu:bounded_linear_operator:quadrature_formula:1} holds true. 
Condition \ref{enu:bounded_linear_operator:quadrature_formula:2} is obvious. 
\end{proof}

Another well-known result of the Banach-Steinhaus Theorem is the divergence 
of the Fourier series. 
Denote by $C_{2 \uppi}$ the linear subspace of $C[-\infty, \infty]$ consisting 
of all $2\uppi$-periodic functions. 
We make $C_{2 \uppi}$ a normed space by equipping it with norm 
\begin{equation*}
    \norm{f}_\infty = \max_{t \in \bR} \abs{f(t)}, \quad 
    \text{for every } f \in C_{2 \uppi}. 
\end{equation*}
One can show that $C_{2 \uppi}$ is a Banach space. 
Recall that for $f \in C_{2 \uppi}$, its Fourier series is 
\begin{equation*}
    f(t) \sim \frac{a_0}{2} + \sum_{k=1}^{\infty} 
        \left( a_k \cos kt + b_k \sin kt \right) 
\end{equation*}
where 
\begin{equation*}
    \begin{aligned}
        a_k &= \frac{1}{\uppi} \int_{-\uppi}^{\uppi} f(t) \cos kt \diff t, \\ 
        b_k &= \frac{1}{\uppi} \int_{-\uppi}^{\uppi} f(t) \sin kt \diff t. 
    \end{aligned}
\end{equation*}
We are concerned with the convergence of 
\begin{equation*}
    \begin{aligned}
        (S_nf)(t) &= \frac{a_0}{2} + \sum_{k=1}^{n} 
            \left( a_k \cos kt + b_k \sin kt \right) \\ 
        &= \frac{1}{2 \uppi} \int_{-\uppi}^{\uppi} f(s) \diff s 
            + \sum_{k=1}^{n} \frac{1}{\uppi} 
            \int_{-\uppi}^{\uppi} f(s) 
                \left( \cos ks \cos kt + \sin ks \sin kt 
                \right) \diff s \\
        &= \frac{1}{\uppi} \int_{-\uppi}^{\uppi} f(s) \left( \frac{1}{2} 
            + \sum_{k=1}^{n} \cos k(s - t) \right) \diff s \\ 
        &= \int_{-\uppi}^{\uppi} f(s) K_n(s, t) \diff s
    \end{aligned}
\end{equation*}
where $K_n(s, t) = \frac{\sin (n + \frac{1}{2}) (s - t)}
{2 \uppi \sin \frac{1}{2}(s - t)}$ is the Dirichlet kernel. 
We show that for every $t \in [\uppi, 2\uppi]$, there exists $f \in C_{2\uppi}$ 
such that $\{ S_n(f) \}_{n=1}^{\infty}$ is divergent. 
Without loss of generality, assume that $t = 0$. 
Let 
\begin{equation*}
    T_n f = (S_n f)(0) = \int_{-\uppi}^{\uppi} f(s) K_n(s, 0) \diff s. 
\end{equation*}
By Example \ref{ex:bounded_linear_operators:integral_kernel}, 
\begin{equation*}
    \begin{aligned}
        \norm{T_n} &= \frac{1}{2 \uppi} K_n(s, 0) \diff s \\
        &\ge \frac{1}{2 \uppi} \int_{-\uppi}^{\uppi} 
            \frac{\abs{\sin (n+\frac{1}{2} s)}}{\abs{\frac{s}{2}}} \diff s\\ 
        &= \frac{2}{\uppi} \int_{0}^{\uppi} 
            \frac{\abs{\sin (n + \frac{1}{2} s)}}{s} \diff s \\
        &= \int_{0}^{(2n + 1) \uppi} \frac{\abs{\sin u}}{u} \diff u \to \infty
    \end{aligned}
\end{equation*}
as $n \to \infty$. 
Here, 
\begin{equation*}
    \begin{aligned}
        \int_{0}^{\infty} &\ge \sum_{n=1}^{\infty} 
            \int_{2(n-1) \uppi}^{2n \uppi} \frac{\sin u}{u} \diff u \\
        &\ge \sum_{n=1}^{\infty} 
            \int_{2(n-1) \uppi + \frac{4}{\uppi}}^{2n \uppi + \frac{4}{\uppi}} 
                \frac{\sin u}{u} \diff u \\
        &\ge \frac{\sqrt{2}}{2} \sum_{n=1}^{\infty} 
            \frac{1}{2n \uppi} \frac{\uppi}{4} \to \infty. 
    \end{aligned}
\end{equation*}
Thus $\{ \norm{T_n} \}_{n=1}^{\infty}$ is unbounded. 
By the Banach-Steinhaus Theorem, there exists $f \in C_{2 \uppi}$ such that 
$\norm{T_n f}$ is unbounded. 
Hence for such $f$, $T_n f$ does not converge. 

%%%%%%%%%%%%%%%%%%%%%%%%%%%%%%%%%%%%%%%%%%%%%%%%%%%%%%%
%%  Section: Open Mapping Theorem and Closed Graph Theorem
%%%%%%%%%%%%%%%%%%%%%%%%%%%%%%%%%%%%%%%%%%%%%%%%%%%%%%%
\section{Open Mapping Theorem and Closed Graph Theorem}
Let $X, Y, Z$ be normed space, $T_1 \in \sB(X, Y)$ and $T_2 \in \sB(Y, Z)$. 
Then we define the product of operators as $T = T_2 T_1 = T_2 \circ T_1$. 
Clearly, $T \in \sB(X, Z)$. 
Also, 
\begin{equation*}
    \begin{aligned}
        \norm{Tx} = \norm{T_2 (T_1 x)} 
        \le \norm{T_2} \norm{T_1 x} 
        \le \norm{T_2} \norm{T_1} \norm{x}, 
    \end{aligned}
\end{equation*}
which implies that $\norm{T} = \norm{T_2 T_1} \le \norm{T_2}\norm{T_1}$. 

\begin{defn}[inverse operator]
Let $X, Y$ be linear operator and $T$ is a bounded linear operator from 
$D(T) \subseteq X$ to $Y$. 
We say $T$ is \emph{invertible} if there exists a linear operator 
$T^{-1}: R(T) \to D(T)$ such that 
\begin{equation*}
    T^{-1} \circ T = I_{D(T)}, \quad 
    T \circ T^{-1} = I_{R(T)}
\end{equation*}
where $I_{D(T)}: D(T) \to D(T), I_R(T): R(T) \to R(T)$ are the identity 
operators. 
We call $T^{-1}$ is the \emph{inverse (operator)} of $T$. 
\end{defn}

Clearly, $T$ admits the inverse $T ^{-1}$ if and only if $T$ is one-to-one.
Besides, the inverse operator of a invertible operator is unique. 

\begin{thm}
Let $X$, $Y$ be two normed spaces and $T: X \to Y$ be a surjective linear 
operator. 
If there exists a positive real number $m$ such that for every 
$x \in X$. 
\begin{equation*}
    \norm{Tx} \ge m \norm{x}, 
\end{equation*}
then $T$ has a bounded inverse operator $T^{-1}$. 
\end{thm}
\begin{proof}
Obviously, $T$ is injective and thus has an inverse operator $T^{-1}$. 
It remains to show that $T^{-1}$ is bounded. 
For $y \in Y$, 
\begin{equation*}
    \norm{y} = \norm{T T^{-1} y} \ge m \norm{T^{-1} y}, 
\end{equation*}
which implies that 
\begin{equation*}
    \norm{T^{-1} y} \le \frac{1}{m} \norm{y}. 
\end{equation*}
It follows that $T^{-1} \in \sB(X, Y)$. 
\end{proof}

For $T \in \sB(X)$, denote $T^n = T \circ T^{n-1}$ for $n = \bNs$ and 
$T^0 = I_X$. 
Then $\norm{T^n} \le \norm{T}^n$ and $T^n \in \sB(X)$. 

\begin{thm}
\label{thm:bounded_linear_operators:i_substract_t_invertible}
Let $X$ be a Banach space and $T \in \sB(X)$. 
If $\norm{T} < 1$, then $I - T$ has a bounded inverse operator and 
$\norm{(I - T)^{-1}} \le \frac{1}{1 - \norm{T}}$. 
\end{thm}
\begin{proof}
Consider the series $S = \sum_{k=1}^{\infty} T^k$ and let 
$S_n = \sum_{k=1}^{n} T^k$. 
For $n, m \in \bNs$, $m > n$, 
\begin{equation*}
    \norm{S_m - S_n} = \norm{\sum_{k=n}^{m-1} T^k} 
    \le \sum_{k=n}^{m-1} \norm{T}^k 
\end{equation*}
is a Cauchy sequence as $\norm{T} < 1$. 
Since $\sB(X, Y)$ is a Banach space, $\{ S_n \}_{n=1}^{\infty}$ has a limit, 
say $S$ in $\sB(X, Y)$. 
Observe that 
\begin{equation*}
    \begin{aligned}
        \begin{aligned}
            (I - T) S &= \lim_{n \to \infty} (I - T) \sum_{k=1}^{n} T^k \\
            &= \lim_{n \to \infty} (I + T + \cdots + T^{n-1}) 
                - (T + T^2 + \ldots + T^n) \\
            &= \lim_{n \to \infty} (I - T^n) = I. 
        \end{aligned}
    \end{aligned}
\end{equation*}
Likewise, $S(I - T) = I$. 
Therefore, $I - T)^{-1} = S$ and 
\begin{equation*}
    \norm{S} = \norm{\sum_{n=0}^{\infty} T^n} 
    \le \sum_{n=0}^{\infty} \norm{T}^n = \frac{1}{1-\norm{T}}. 
\end{equation*}
This completes the proof. 
\end{proof}

\begin{cor}
\label{cor:bounded_linear_operators:t_plus_delta_t_invertible}
Let $X$ be a Banach space and $T \in \sB(X)$ have a bounded inverse operator. 
Then for any operator $\Delta T \in \sB(X)$ with 
\begin{equation*}
    \norm{\Delta T} < \frac{1}{\norm{T^{-1}}}, 
\end{equation*}
$T + \Delta T$ has a bounded inverse operator and 
\begin{equation*}
    (T + \Delta T)^{-1} = \sum_{k=0}^{\infty} (-1)^k 
        \left( T^{-1} \Delta T \right)^k T^{-1}. 
\end{equation*}
\end{cor}
\begin{proof}
Let $S = T + \Delta T = T (I + T^{-1} \Delta T)$. 
As $\norm{T^{-1}\Delta T} \le \norm{T ^{-1}} \norm{\Delta T} < 1$, 
$I + T ^{-1} \Delta T$ has the bounded inverse 
\begin{equation*}
    \left( I + T ^{-1} \Delta T \right) ^{-1} 
    = \sum_{n=0}^{\infty} (-1)^n \left( T ^{-1} \Delta T \right)^n. 
\end{equation*}
Then 
\begin{equation*}
    S ^{-1} = \left( I + T ^{-1} \Delta T \right) ^{-1} T ^{-1} 
    = \sum_{k=0}^{\infty} (-1)^k \left( T ^{-1} \Delta T \right)^k T ^{-1} 
\end{equation*}
is bounded. 
\end{proof}

Now we proceed with the main theorems in this section: the Open Mapping 
Theorem and the Closed Graph Theorem. 
Note that most theorems requires completeness of the space. 

\begin{defn}[Open mapping]
A mapping that maps every open set to an open set is called an open mapping. 
\end{defn}

\begin{thm}[Banach Open Mapping Theorem]
\index{theorem!Banach Open Mappint \~{}}
Let $X, Y$ be Banach spaces and $T: X \to Y \in \sB(X, Y)$. 
If $T$ is surjective, then $T$ is an open mapping. 
\end{thm}
\begin{proof}
\textbf{Step 1. }
As $X = \bigcup_{k=1}^\infty \overline{B(0, k)}$, $Y = TX = \bigcup_{k=1}
^\infty T\overline{B(0, k)}$ and $Y$ is complete, by Baire's Category 
Theorem, there exists some $k_0 \in \bNs$ such that $T\overline{B(0, k_0)}$ 
is dense in some open ball $B_Y(y_0, r_0)$ in $Y$. 
Here, we use $B$ and $B_Y$ to denote the open balls in $X$ and $Y$ 
respectively. 

\textbf{Step 2. }
We claim that there exists $\delta > 0$ such that for any $\epsilon > 0$, 
$T \overline{B(0, \epsilon)}$ is dense in $\overline{B_Y(0, \delta 
\epsilon)}$. 
Setting $\delta = \frac{r_0}{k_0}$, $\epsilon > 0$ and letting $y \in 
\overline{B_Y(0, \epsilon \delta)}$, then $y_0 \pm \frac{k_0}{\epsilon} y$ 
satisfies 
\begin{equation*}
    \norm{y_0 \pm \frac{k_0}{\epsilon} y - y_0} 
    = \frac{k_0}{\epsilon} \norm{y} 
    < \frac{k_0}{\epsilon} \epsilon \delta = r_0, 
\end{equation*}
which is $y_0 \pm \frac{k_0}{\epsilon} y \in B_Y(y_0, r_0)$. 
Hence, we can find a sequence $\{ x_n \}_{n=1}^{\infty}$, 
$\{ x_n' \}_{n=1}^{\infty}$ in $\overline{B(0, k_0)}$ such that 
\begin{equation*}
    Tx_n \to y_0 + \frac{k_0}{\epsilon} y, \quad 
    Tx_n' \to y_0 - \frac{k_0}{\epsilon} y \quad (n \to \infty)
\end{equation*}
since $T \overline{B(0, k_0)}$ is dense in $\overline{B(0, k_0)}$. 
Then 
\begin{equation*}
    T \left( \frac{\epsilon}{2k_0} (x_n - x_n') \right) \to y 
    \quad (n \to \infty)
\end{equation*}
Note that 
\begin{equation*}
    \frac{\epsilon}{2k_0}(x_n - x_n') \subseteq \overline{B(0, \epsilon)}.
\end{equation*}
Thus we succeed to prove our claim that $T \overline{B(0, \epsilon \delta)}$ 
is dense in $B_Y(0, \epsilon \delta)$ for any $\epsilon > 0$. 

\textbf{Step 3. }
Now we try to prove the key claim: for the $\delta$ given above and 
any $r > 0$, it holds that $T \overline{B(0, r)} \supseteq B_Y(0, 
\frac{1}{2} r \delta)$. 
By the linearity of $T$, it suffices to prove the case $r = 1$. 
Choosing $y_0 \in B_Y(0, \frac{1}{2} \delta)$, by previous claim, there 
exists $x_1 \in \overline{B(0, \frac{1}{2})}$ such that 
\begin{equation*}
    \norm{T x_1 - y_0}< \frac{1}{2^2} \delta. 
\end{equation*}
Further, let $y_1 = y_0 - Tx_1 \in B_Y(0, \frac{1}{2^2} \delta)$, 
there exists $x_2 \in \overline{B(0, k_0)}$ such that 
\begin{equation*}
    \norm{y_1 - T x_2} < \frac{1}{2^3} \delta. 
\end{equation*}
Repeating the process, we obtain sequences $\{ x_n \}_{n=1}^{\infty} 
\subseteq X$ and $\{ y_n \}_{n=1}^{\infty} \subseteq Y$ such that 
$\norm{x_n} \le \frac{1}{2^n}$, $y_n = y_{n-1} - Tx_n = y_0 - 
T\left( \sum_{i=1}^{n} x_n \right)$ and $\norm{y_n} < \frac{1}{2^{n+1}} 
\delta$. 
Tt follows from the completeness of $X$ and the fact that $\sum_{n=1}^
{\infty} \le \sum_{n=1}^{\infty} \frac{1}{2^n} = 1$ that $\sum_{n=1}^{\infty} 
x_n$ is convergent to some element $x_0$ say, of $X$ such that $\norm{x_0} 
< 1$. 
Letting $n \to \infty$ yields that $y_0 = Tx_0$. 

\textbf{Step 4. }
Now we are prepared to show that $T$ is a open mapping. 
Let $G$ be an open subset in $X$ and $x \in G$. 
Then there exists $r_0 > 0$ such that $\overline{B(x, \frac{r_0}{2})} 
\subseteq B(x, r_0) \subseteq G$. 
By Step 3, $T \overline{B(0, r)} \supseteq B_Y(0, \frac{1}{2}r \delta)$. 
Hence, 
\begin{equation*}
    \begin{aligned}
        T \overline{B(x, \frac{r_0}{2})} 
        &= T\left( x + \overline{B(0, \frac{r_0}{2})} \right) \\
        &= Tx + T \overline{B(0, r)} 
        \supseteq Tx + S_1(0, \frac{1}{4} r_0 \delta), 
    \end{aligned}
\end{equation*}
which implies that $Tx$ is a inner point of $TG$. 
Thus $TG$ is an open subset in $Y$ and we establish the conclusion. 
\end{proof}

The famous Inverse Mapping Theorem is an application of the Open Mapping 
Theorem. 
\begin{thm}[Inverse Mapping Theorem]
Let $T$ be a bijective linear operator from a Banach space $X$ to a Banach 
space $Y$. 
Then $T^{-1} \subset \sB(Y, X)$. 
\end{thm}
\begin{proof}
It suffices to show that for every open subset $G \subseteq X$, 
$\left( T^{-1} \right) ^{-1}(G)$ is open in $Y$. 
Note that $\left( T^{-1} \right) ^{-1}(G) = T(G)$. 
By the Open Mapping Theorem, $TG$ is open in $Y$, which proves the result. 
\end{proof}

\begin{cor}
Let $\norm{\wdot}$ and $\norm{\wdot}$ be two norms on a linear space $X$ 
that is complete under each of these two norms. 
If there exists a constant $C > 0$ such that for every $x \in  X$, 
\begin{equation*}
    \norm{\wdot}_2 \le C \norm{\wdot}_1, 
\end{equation*} 
then $\norm{\wdot}_1$ and $\norm{\wdot}_2$ are equivalent. 
\end{cor}
\begin{proof}
Letting $X = (X, \norm{\wdot}_1), X_2 = (X, \norm{\wdot}_2)$, then by 
assumption, both $X_1$, $X_2$ are Banach space. 
Define 
\begin{equation*}
    \begin{aligned}
        T: X_1 &\to X_2 \\
        x &\mapsto x.
    \end{aligned}
\end{equation*}
It follows that for every $x \in X$, $\norm{Tx}_2 \le C \norm{x}_1$, namely, 
$T \in \sB(X_1, X_2)$. 
Since $T$ is bijective, by the Inverse Mapping Theorem, $T^{-1}$ is a 
bounded linear operator from $X_2$ to $X_1$. 
This means there exists a constant $D > 0$ such that for every $x \in X$, 
\begin{equation*}
    \norm{x}_1 \le \norm{T^{-1}x} \le D \norm{x}_2.
\end{equation*}
Thus the conclusion follows immediately. 
\end{proof}

\begin{defn}[Graph]
Let $X, Y$ be normed space and $T: X \to Y$ is a linear operator. 
The \emph{graph} of $T$ is the set 
\begin{equation*}
    G(T) = \{(x, Tx): x \in X\}.
\end{equation*}
Then $G(T)$ is a subset of $X \times Y$ on which we define a norm as 
\begin{equation*}
    \norm{(x, y)} = \norm{x}_X + \norm{y}_Y, \quad 
    \text{for every } (x, y) \in X \times Y.
\end{equation*}
\end{defn}

If $T$ is a linear operator, then $G(T)$ is clearly a linear subspace of 
$X \times Y$. 
$T$ is said to be a \emph{closed operator} if $G(T)$ is a closed subspace 
of $X \times Y$. 

\begin{prop}
Let $X, Y$ be normed spaces and $T: X \to Y$ be a linear operator. 
Then $T$ is a closed operator if and only if for every sequence 
$\{ x_n \}_{n=1}^{\infty} \subseteq X$, 
\begin{equation*}
    x_n \to x, \quad T x_n \to y \quad (n \to \infty) 
    \implies y = Tx. 
\end{equation*}
\end{prop}
\begin{proof}
This is a direct result from the definition of closed operator. 
\end{proof}

\begin{thm}[Closed Graph Theorem]
Let $X$, $Y$ be Banach space and $T: X \to Y$ is a linear operator. 
Then $T$ is bounded if and only if its graph $G(T)$ is closed in 
$X \times Y$, \ie, $T$ is a closed operator. 
\end{thm}
\begin{proof}
($\Rightarrow$)
Suppose $T$ is bounded and let $\{ x_n \}_{n=1}^{\infty}$ be a sequence in 
$X$ such that $(x_n, Tx_n) \to (x, y)$ as $n \to \infty$. 
Then 
\begin{equation*}
    y = \lim_{n \to \infty} T x_n 
    = T\left( \lim_{n \to \infty} x_n \right) 
    = Tx, 
\end{equation*}
which implies $(x, y) = (x, Tx) \in G(T)$. 
Hence $T$ is a closed operator. 

($\Leftarrow$)
Suppose that $T$ is a closed operator. 
Then $G(T)$ is a linear closed subspace in $X \times Y$. 
Since $X, Y$ are Banach spaces, so is $X \times Y$. 
Define 
\begin{equation*}
    \begin{aligned}
        \tilde T: G(T) &\to X, \\
        (x, Tx) &\mapsto x. 
    \end{aligned}
\end{equation*}
It is clear that $\tilde T$ is bijective. 
As $\norm{\tilde T (x, Tx)} = \norm{x} \le \norm{x} + \norm{Tx} = 
\norm{(x, Tx)}$, $\tilde T$ is bounded. 
By the Inverse Mapping Theorem, there exists $C > 0$ such that 
\begin{equation*}
    \norm{\tilde T^{-1} x} = \norm{(x, Tx)} \le C \norm{x}, 
\end{equation*}
which is 
\begin{equation*}
    \norm{Tx} \le (C - 1) \norm{x}. 
\end{equation*}
This completes the proof. 
\end{proof}
%%%%%%%%%%%%%%%%%%%%%%%%%%%%%%%%%%%%%%%%%%%%%%%%%%%%%%%
%% Section: Spectrum of Linear Operators
%%%%%%%%%%%%%%%%%%%%%%%%%%%%%%%%%%%%%%%%%%%%%%%%%%%%%%%
\section{Spectrum of Linear Operators}
We consider the linear operator 
\begin{equation*}
    T_\lambda = \lambda I - T. 
\end{equation*}
\begin{defn}
\index{spectrum}
\index{spectrum!point \~{}}
\index{spectrum!continuous \~{}}
\index{spectrum!residual \~{}}
\index{regular point}
\index{resolvent set}
\index{resolvent}
Let $X$ be a normed space and $T$ be a linear operator from $D(T) \subseteq X$ 
to $X$. 
For $\lambda \in \bK$, there are several possible cases: 
\begin{enumerate}
    \item 
    $R(T_\lambda)$ has a dense range in $X$ and admits a bounded inverse. 
    We call $\lambda$ a \emph{regular point} of $T$ and denote by $\rho(T)$ 
    the set of all regular points of $T$. 
    Then $\rho(T)$ is called the \emph{resolvent set} of $T$ and $\sigma(T) 
    \coloneqq \bK \backslash \rho(T)$ is called the \emph{spectrum} of $T$. 
    And the inverse $(I - \lambda T)^{-1}$ denoted by $R(\lambda, T)$ is 
    called the \emph{resolvent} at $\lambda$ of $T$. 

    \item $T_\lambda$ is not injective, namely, there exists $x \in X, 
    x \neq 0$ such that $(T - \lambda I)x = 0$. 
    In this case, $T_\lambda$ does not have an inverse. 
    We call $\lambda$ is an \emph{eigenvalue} of $T$ and $x$ an 
    \emph{eigenvector} of $T$. 
    The nullspace of $T_\lambda$ is called the \emph{eigenspace} of $T$ 
    corresponding to the eigenvalue $\lambda$ of $T$. 
    The dimension of the eigenspace corresponding to the eigenvalue 
    $\lambda$ of $T$ is called the \emph{multiplicity} of $\lambda$. 
    The set $\sigma_p(T)$ of all eigenvalues of $T$ is called the 
    \emph{point spectrum} of $T$. 

    \item $T_\lambda$ has an unbounded inverse whose domain 
    $R(\lambda I - T)$ is dense in $X$.
    We call the totality of such $\lambda$ the \emph{continuous spectrum} 
    of $T$, denoted by $\sigma_C(T)$. 

    \item $T_\lambda$ has an inverse whose domian is not dense in $X$. 
    We call the totality of such $\lambda$ the \emph{residual spectrum} of 
    $T$, denoted by $\sigma_R(T)$.
    
\end{enumerate}
\end{defn}

\begin{thm}
Let $X$ be a Banach space and $T \in B(X)$. 
Then $\sigma(T)$ is bounded and closed. 
In fact, $\sigma(T) \subseteq \{ \lambda \in \bK: \abs{\lambda} \le \norm{T} 
\}$. 
\end{thm}
\begin{proof}
For $\abs{\lambda_0} > \norm{T}$, then $\lambda_0 I - T = \lambda_0 
(I - \frac{1}{\lambda_0} T)$. 
Since $\norm{\frac{1}{\lambda_0} T} = \frac{\norm{T}}{\lambda_0}< 1$, by Theorem 
\ref{thm:bounded_linear_operators:i_substract_t_invertible}, $I - \lambda_0 T$ 
has an bounded inverse and 
\begin{equation*}
    (\lambda_0 I - T)^{-1} = \frac{1}{\lambda_0} \sum_{k=0}^{\infty} 
        \left( \frac{T}{\lambda_0} \right) ^k.
\end{equation*}
Therefore, $\lambda_0 \in \rho(T)$ and $\sigma(T) \subseteq \left\{ 
\lambda \in \bK: \abs{\lambda} \le \norm{T} \right\}$. 
To prove that $\sigma(T)$ is closed, it suffices to show that $\rho(T)$ is 
open. 
Letting $\lambda_0$ be a resolvent point of $T$, then by Corollary 
\ref{cor:bounded_linear_operators:t_plus_delta_t_invertible}, for 
$\delta < \norm{(\lambda I - T) ^{-1}}$, $\lambda I - T - \delta I$ has 
bounded inverse. 
Thus $(\lambda_0 - \delta, \lambda_0 + \delta)$ is contained in $\rho(T)$, 
whence $\rho(T)$ is a open set. 
\end{proof}

\begin{example}
Consider $T: C[a, b] \to C[a, b], T(f)(t) = t f(t)$. 

For $\lambda \in \bR$, $\left[ (T - \lambda T) f \right](t) 
= (t - \lambda) f(t)$. 
Thus if $\lambda \notin [a, b]$, $\left[ (T - \lambda T) ^{-1} f \right](t) 
= \frac{1}{t - \lambda} f(t)$. 
If $\lambda \in [a, b]$, then $\left[ (T - \lambda T) f \right](\lambda) 
= (\lambda - \lambda) f(\lambda) = 0$. 
Thus $T - \lambda I$ is not surjective and $\sigma(T) = [a, b]$. 
Suppose that $(I - \lambda T)f = 0$. 
Then 
\begin{equation*}
    (t - \lambda) f(t) \equiv 0 \implies 
    f(t) = 0 \quad \text{for every } t \neq \lambda \implies 
    f(t) \equiv 0. 
\end{equation*}
Hence, $\sigma_p(T) = \varnothing$. 
\end{example}

%%%%%%%%%%%%%%%%%%%%%%%%%%%%%%%%%%%%%%%%%%%%%%%%%%%%%%%
%%  Hahn-Banach Theorem
%%%%%%%%%%%%%%%%%%%%%%%%%%%%%%%%%%%%%%%%%%%%%%%%%%%%%%%
\section{Hahn-Banach Theorem and its Applications}
For convenience, we introduce the notation of restriction of operators. 
Let $X, Y$ be sets and let $f: X \to Y$. 
If $D$ is a subset of $X$, then we refer to $f_{\chi_D}$ as a function from 
$D$ to $Y$ such that $f_{\chi_D}(x) = f(x)$ for all $x \in D$. 
If we further let $X$ be equipped with a norm $\norm{\wdot}$, then $(G, 
\norm{\wdot}_D)$ denotes the normed subspace of $(X, \norm{\wdot})$ with 
$\norm{x}_D = \norm{x}$ for all $x \in D$. 
If $f \in D^\ast$, then the operator norm of $f$ is denoted by $\norm{f}_D$. 

\begin{thm}[Hahn-Banach Theorem over $\bR$]
\label{thm:bounded_linear_operator:hahn_banach_real}
Let $M$ be linear subspace of a linear space $X$ over $\bR$, and let $p: X 
\to \bR$ be a sublinear functional, \ie, for every $x, y \in X$ and $\alpha 
> 0$, it holds true that 
\begin{enumerate}
    \item $p(\alpha x) = \alpha p(x)$. 
    \item $(x + y) \le p(x) + p(y)$. 
\end{enumerate}
If $f$ is a linear functional on $M$ such that $f(x) \le p(x)$ for all 
$x \in M$, then there exists a linear functional $F$ on $X$ such that 
\begin{enumerate}
    \item $F_{\chi_M} = f$. 
    \item $F(x) \le p(x)$ for all $x \in X$. 
\end{enumerate}
\end{thm}
\begin{proof}
Let 
\begin{equation*}
    \begin{aligned}
        \Omega = \{ (Y, F_Y): 
        &Y \text{ is a linear subspace of } X, 
        F_Y \text{ is a linear functional on } Y, \\
        &M \subseteq Y, 
        F_{Y \chi_M} = f, 
        F_Y(x) \le p(x) \text{ for all } x \in X \}.
    \end{aligned}
\end{equation*}
We introduce an ordering relation $\le$ on $\Omega$ as 
\begin{equation*}
    (Y_1, F_{Y_1}) \le (Y_2, F_{Y_2}) \iff 
    Y_1 \subseteq Y_2, \quad F_{Y_1 \chi_{Y_1}} = F_{Y_2}. 
\end{equation*}
We shall use the Zorn's Lemma \ref{lemma:zorn_lemma} on $\Omega$. 
To this end, we first show that every totally ordered subset of $\Omega$ 
has an upper bound. 
Let $\Omega_0$ be a totally ordered subset of $\Omega$, and define 
\begin{equation*}
    \begin{aligned}
        Y_0 &= \bigcup_{Y \subseteq \Omega_0} Y \\
        F_{Y_0}(x) &= F_{Y_n}(x), \quad x \in Y_n. 
    \end{aligned}
\end{equation*}
Then $(Y_0, F_{Y_0}) \subseteq \Omega$ and for every $Y \subset 
\Omega$, $(Y, F_Y) \le (Y_0, F_{Y_0})$. 
It follows from Zorn's Lemma that $\Omega$ contains a maximal element 
$(Y_\infty, F_{Y_\infty})$. 

We contend that $Y_\infty = X$. 
For the sake of contradiction we assume that there exists $x_0 \in X$ such 
that $x_0 \in Y_\infty$. 
Letting $\tilde Y = \mspan Y \cup \{x_0\}$, then every $x \in tilde Y$ can 
be written uniquely as the form $x = y + \alpha x_0$ with $y \in Y_\infty$ 
and $\alpha \in \bR$. 
Define a linear functional extending $F_{Y_\infty}$ on $\tilde Y$ as 
\begin{equation*}
    \tilde F (x) = \tilde F (y + \alpha x_0) 
    = \tilde F (y) + \alpha \tilde F (x_0) 
    = F_\infty (y) + \alpha \beta
\end{equation*}
where $\beta = \tilde F(x_0)$ should be a constant real number. 
Besides, we hope that $\tilde F$ also obeys 
\begin{equation}
    \label{equ:bounded_linear_operator:hahn_banach:1}
    \tilde F(y + \alpha x_0) 
    = F_{Y_\infty} (y) + \alpha \beta 
    \le p(y + \alpha x_0) \quad 
    \text{for all } y \in Y_\infty, \alpha \in \bR.
\end{equation}
If $\alpha = 0$, then Equation (\ref{equ:bounded_linear_operator:hahn_banach:1}) 
is satisfied by definition; 
if $\alpha > 0$, then it follows that 
\begin{equation*}
    \begin{aligned}
        &F_{Y_\infty} (y) + \alpha \beta 
        \le p(y + \alpha x_0) \quad 
        \text{for all } y \in Y_\infty, \alpha \in \bR \\
        \iff &\beta \le -F_{Y_\infty}(\frac{y}{\alpha}) 
            + p(\frac{y}{\alpha} + x_0) \quad 
            \text{for all } y \in Y_\infty, \alpha \in \bR \\
        \iff &\beta \le -F_{Y_\infty}(y) + p(y + x_0) \quad 
            \text{for all } y \in Y_\infty;
    \end{aligned}
\end{equation*}
If $\alpha < 0$, likewise it follows that 
\begin{equation*}
    \begin{aligned}
        &F_{Y_\infty} (y) + \alpha \beta 
        \le p(y + \alpha x_0) \quad 
        \text{for all } y \in Y_\infty, \alpha \in \bR \\
        \iff &F_{Y_\infty}(-\frac{y}{\alpha}) 
            - p(-\frac{y}{\alpha} - x_0) 
            \le \beta \quad 
            \text{for all } y \in Y_\infty, \alpha \in \bR \\
        \iff &F_{Y_\infty}(y) - p(y - x_0) 
            \le \beta \quad 
            \text{for all } y \in Y_\infty.
    \end{aligned}
\end{equation*}
Therefore, such $\beta$ exists if and only if for any $y, y' \in Y_\infty$, 
\begin{equation*}
    F_{Y_\infty}(y') - p(y' - x_0) \le \beta \le -F(y) + p(y + x_0)
\end{equation*}
which is equivalent to 
\begin{equation*}
    \sup_{y \in Y_\infty} (F_{Y_\infty}(y) - p(y - x_0)) 
    \le \beta 
    \le \inf_{y \in Y_\infty} (-F_{Y_\infty}(y) + p(y + x_0)).
\end{equation*}
We claim that this is indeed the case by checking that for any 
$y, y' \in Y_\infty$, 
\begin{equation*}
    \begin{aligned}
        -F_{Y_\infty}(y) &+ p(y + x_0) 
            - F_{Y_\infty}(y') + p(y' - x_0) \\
        &= -F_{Y_\infty}(y + y') + p(y + x_0) + p(y' - x_0) \\
        &\ge -F_{Y_\infty}(y + y') + p(y + y') 
        \ge 0.
    \end{aligned}
\end{equation*}
This means $(\tilde Y, \tilde F) \in \Omega$ and $(Y_\infty, F_{Y_\infty})
< (\tilde Y, \tilde F)$, contradicting the fact that $(Y_\infty, 
F_{Y_\infty})$ is an maximal element. 
Hence $Y_\infty = X$, which completes the proof. 
\end{proof}

\begin{thm}[Hahn-Banach Theorem]
Let $X$ be a normed space, $G \subseteq X$ be a linear subspace, and let 
$f \in G^\ast$. 
Then there exists $F \in X^\ast$ such that $F_{\chi_G} = f$ and $\norm{F} 
= \norm{f}_G$.
\end{thm}
\begin{proof}
We first prove the case that $\bK = \bR$. 
Define $p: X \to \bR, x \mapsto \norm{f}_G \norm{x}$. 
Then for any $x \in G$, 
\begin{equation*}
    f(x) \le \abs{f(x)} \le \norm{f}_G \norm{x} = p(x). 
\end{equation*}
By the Hahn-Banach Theorem over $\bR$ 
\ref{thm:bounded_linear_operator:hahn_banach_real}, there exists a linear 
functional $F$ such that $F_{\chi_{G}} = f$ and $F(x) \le p(x)$ for all 
$x \in X$. 
This means $F(x) \le p(x) = \norm{f}_G \norm{x}$ and $F(-x) \le p(x) = 
\norm{f}_G \norm{x}$, implying $\norm{F} \le \norm{f}_G$ and $F \in X^\ast$. 
Since $\norm{F} \ge \norm{f}_G$, we have $\norm{f}_G = \norm{F}$. 
This is exactly what we want. 

Next, we proceed with the case that $\bK = \bC$. 
Suppose $f(x) = \phi(x) + \mi \psi(x)$, where $\phi$, $\psi$ are real-valued 
functions, which can be viewed as linear functionals over $\bR$. 
Noticing that 
\begin{equation*}
    f(\mi x) = \phi(\mi x) + \mi \psi(\mi x) = \mi \phi(x) - \psi(x) 
    = \mi f(x),  
\end{equation*}
we get $\phi(x) = \psi(ix)$ for all $x \in X$. 
Then it follows from the fact that for $x \in G$, 
\begin{equation*}
    \phi(x) \le \abs{f(x)} \le \norm{f}_G \norm{x} = p(x)
\end{equation*}
that there exists a bounded linear functional $\Phi$ such that 
\begin{enumerate}
    \item $\Phi_{\chi_G} = f$. 
    \item $\Phi(x) \le p(x) = \norm{f}_G \norm{x}$ for all $x \in X$. 
\end{enumerate}
by the Hahn-Banach Theorem over $\bR$. 

Define $F(x) = \Phi(x) - \mi \Phi(\mi x)$. 
Then it is easy to show the following three facts: 
\begin{enumerate}
    \item $F$ is a linear functional over $\bK$. 
    \item $F_{\chi_{G}} = f$. 
    \item $\abs{f(x)} \le p(x)$ for all $x \in X$. 
    Indeed, for any $x \in X$, assuming that $F(x) = \me^{\mi \theta} 
    \abs{F(x)}$, 
    then 
    \begin{equation*}
        \abs{F(x)} = e^{- \mi \theta} F(x)
        = F(\me ^{- \mi \theta} x) = \Phi(\me ^{- \mi \theta} x) 
        \le \norm{f}_G \norm{\me ^{- \mi \theta} x} 
        = \norm{f}_G \norm{x}.
    \end{equation*}
\end{enumerate}
Repeating the process with which we proved the case of real linear space 
finally completes the proof. 
\end{proof}

\begin{cor}
LEt $X$ be a normed space, and let $x_0 \in X$ with $x_0 \neq 0$. 
Then there exists an $f \in X^\ast$ such that $f(x_0) = \norm{x_0}$ and 
$\norm{f} = 1$. 
\end{cor}
\begin{proof}
Let $G = \mspan \{x_0\}$ and define $f_0: G \to \bK$ as $f_0(\alpha x_0) 
= \alpha \norm{x_0}$ for all $x \in X$. 
Then $f_0 \in G^\ast$. 
By the Hahn-Banach Theorem, $f_0$ extends to some $f \in X^\ast$ that 
satisfies our requirement. 
\end{proof}

\begin{cor}
Let $X$ be a normed space, and let $G$ be a subspace of $X$. 
If $x \in X$ satisfies $d = d(x_0, G) = \inf_{x \in G} \norm{x, x_0} > 0$, 
then there exists $f \in X^\ast$ such that $f(x_0) = 1$, $f(x) = 0$ for any
$x \in G$ and $\norm{f} = \frac{1}{\alpha}$. 
\end{cor}
\begin{proof}
Let $G_1 = \mspan G \cup \{x_0\}$ and define $f_1: G_1 \to \bR$ as 
$(\alpha x_0 + x) = \alpha$, which clearly belongs to $G_1$. 
We compute 
\begin{equation*}
    \begin{aligned}
        \norm{f}_{G_1} &= \sup_{\alpha x_0 + x \neq 0} 
        \frac{\abs{f_1(\alpha x_0 + x)}}{\norm{\alpha x_0 + x}} 
        = \sup_{\substack{\alpha \neq 0 \\ x \in G}} \frac{\abs{\alpha}}{\norm{\alpha x_0 + x}} \\
        &= \sup_{\substack{\alpha \neq 0 \\ x \in G}} \frac{1}{\norm{x_0 + \frac{x}{\alpha}}} 
        = \sup_{x \in G} \frac{1}{\norm{x - x_0}} = \frac{1}{d}. 
    \end{aligned}
\end{equation*}
Then the conclusion follows as a result of the Hahn-Banach Theorem.
\end{proof}

\begin{prop}
Let $X$ be a normed space. 
Then 
\begin{equation*}
\norm{x} = \sup_{\substack{f \in X^\ast \\ f \neq 0}} 
\frac{f(x)}{\norm{f}} 
= \sup_{\substack{f \in X^\ast \\ \norm{f} = 1}} f(x)
\end{equation*}
\end{prop}
\begin{proof}
Here we only deal with the first equality. 
We assume without loss of generality that $x \in X$ is such that $x \neq 0$. 
For any $f \in X^\ast$ such that $f \neq 0$, $\abs{f(x)} \le \norm{f} 
\norm{x}$ yields $\frac{\abs{f(x)}}{\norm{f}} \le \norm{x}$. 
Therefore, $\norm{x} \ge \sup_{\substack{f \in X^\ast \\ f \neq 0}} 
\frac{f(x)}{\norm{f}}$. 

On the other hand, by the Hahn-Banach Theorem, there exists an 
$f_0 \in X^\ast$ such that $f_0(x) = 1$ and $\norm{f_0} = 1$. 
Hence, 
\begin{equation*}
    \sup_{\substack{f \in X^\ast \\ f \neq 0}} 
    \frac{f(x)}{\norm{f}} 
    \ge \frac{\norm{x} f_0(x)}{\norm{f_0}} = \norm{x}, 
\end{equation*}
which completes the proof. 
\end{proof}
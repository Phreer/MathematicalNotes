\chapter{Metric Spaces}
This chapter is based on metric space, and the basic definition of 
metric space is presented in chapter \ref{sec:metric_space}.

\section{Fixed Point Theorem of Banach}
The fixed point theorem of Banach, also known as 
contraction principle, plays a fundamental role in
the analysis of iterative algorithm. 

\begin{thm}[Fixed Point Theorem of Banach]
\index{theorem!Banach fixed-point \~{}}
\index{fixed point theorem of Banach}
\label{thm:metric_spaces:fixed_point_theorem_of_banach}
Suppose that $X$ is a nonempty complete metric space and 
operator $A: X \to X$ is $k$-contractive, \ie, 
\begin{equation}
d(Ax, Ay) \le kd(x, y) \text{  for any } x, y \in X
\end{equation}
where $k \in [0, 1)$. Then we have the following results:
\begin{enumerate}
\item \emph{Existence and Uniqueness of the fixed-point}. 
\label{thm:metric_spaces:fixed_point:existence_uniqueness}
The equation 
\begin{equation}
Ax = x
\end{equation}
has a unique solution in $X$.
\item \emph{Convergence of the Iterative Method}. 
\label{thm:metric_spaces:fixed_point:convergence}
For a given $x_0 \in X$, the sequence $\left\{ x_n \right\}$ generated 
by the iterative equation 
\begin{equation*}
    x_{n+1} = Ax_n
\end{equation*}
converges to the unique solution to equation $Ax = x$ (fixed point). 
\item \emph{Error estimates.}
\label{thm:metric_spaces:fixed_point:error}
For all $n \in \bN$, we have the so-called \apri error estimate 
\begin{equation}
    \label{equ:metric_spaces:fixed_point:apriori_error}
    d(x_n, x) \le \frac{k^n}{1-k} d(x_1, x_0), 
\end{equation}
and the so-called \apost error estimate 
\begin{equation}
    \label{equ:metric_spaces:fixed_point:aposterior_error}
    d(x_{n+1}, x) \le \frac{k}{1-k} d(x_{n+1}, x_n). 
\end{equation}
\item \emph{Rate of Convergence}.
\label{thm:metric_spaces:fixed_point:convergent_rate}
For all $n \in \bn$, we have 
\begin{equation*}
    d(x_{n+1}, u) \le k d(x_n, x). 
\end{equation*}
\end{enumerate}
\end{thm}

\begin{proof}
\ref{thm:metric_spaces:fixed_point:existence_uniqueness}
\ref{thm:metric_spaces:fixed_point:convergence}
\textbf{Step 1. $\{x_n\}$ is a Cauchy sequence in $X$.}
Since $A$ is a $k$-contractive operator, we have 
\begin{equation*}
    \begin{aligned}
        d(x_{n+1}, x_n) &= d(Ax_n, Ax_{n-1}) 
        \le k d(x_n, x_{n-1}) \\ 
        &\le k^2 d(x_{n-1}, x_{n-2}) 
        \le k^n d(x_1, x_0). 
    \end{aligned}
\end{equation*}
Now let $n, m \in \bN$, by triangle inequality and the sum formula for 
geometric series, it yields that 
\begin{equation*}
    \begin{aligned}
        d(x_n, x_m) &\le \sum_{i=0}^{m-n-1} d(x_{n+i}, x_{n+i+1}) 
        \le \sum_{i=0}^{m-n-1} k^{n+i} d(x_1, x_0) \\ 
        &\le \frac{k^n}{1-k} d(x_1, x_0). 
    \end{aligned}
\end{equation*}
It follows that $d(x_n, x_m) \to 0$ as $n \to \infty$ since $0 \le k < 1$. 
Hence $\left\{ x_n\right\}$ is Cauchy and $\{x_n\}$ converges to some point 
$u \in X$ since $X$ is complete. 

\textbf{Step 2. $x$ is a solution to fixed-point of $A$. }
The definition of contractive operator yields that $A$ is a continuous 
operator. 
Thus $x = \lim_{n \to \infty} x_n = \lim_{N \to \infty} Ax_{n-1} = A\left( 
\lim_{n \to \infty} x_{n-1} \right) = Ax$. 

\textbf{Step 3. Uniqueness of fixed-point. }
Supposing that $x, x' \in X$ such that $x = Ax$ and $x' = Ax'$, it follows 
that $d(x, x') = d(Ax, Ax') \le kd(x, x') < d(x, x')$, whence 
$d(x, x') = 0$. 

\ref{thm:metric_spaces:fixed_point:error} 
Letting $m \to \infty$, it follows from 
\begin{equation*}
    d(x_n, x_{n+m}) \le \frac{k^n}{1 - k} d(x_1, x_0)
\end{equation*}
that for every $n \in \bn$, 
\begin{equation*}
    d(x_n, x) \le \frac{k^n}{1 - k} d(x_1, x_0). 
\end{equation*}
This is the estimate (\ref{equ:metric_spaces:fixed_point:apriori_error}).

To prove estimate (\ref{equ:metric_spaces:fixed_point:aposterior_error}), 
observe that 
\begin{equation*}
    d(x_{n+1}, x_{n+m+1}) \le \sum_{i=1}^{m} d(x_{n+i}, x_{n+i+1}) 
    \le \sum_{i=1}^{m} k^{i} d(x_{n}, x_{n+1}). 
\end{equation*}
Letting $m \to \infty$, we get 
\begin{equation*}
    d(x_{n+1}, x) \le \frac{k}{1-k} d(x_n, x_{n+1}).  
\end{equation*}

\ref{thm:metric_spaces:fixed_point:convergent_rate}
Observing that 
\begin{equation*}
    d(x_{n+1}, x) = d(Ax_n, Ax) \le k d(x_n, x). 
\end{equation*}
\end{proof}

\begin{rmk}
The condition that $A: X \to X$ can be replaced with that $A: M \to M$, 
where $M$ is a closed subset of $X$. 
\end{rmk}

\begin{cor}
Supposing that $(X, d)$ is a complete metric space and $T: X \to X$ is an 
operator such that $T^{n_0}$ is a contractive operator for given $n_0 \in 
\bN$, then $T$ has a unique fixed-point. 
\end{cor}
\begin{proof}
As $T^{n_0}$ is a contractive operator, it has a unique fixed point $x_0$. 
Observe that 
\begin{equation*}
    T^{n_0}(Tx_0) = T(T^{n_0}x_0) = T(x_0), 
\end{equation*}
which implies that $Tx_0$ is a fixed-point of $T^{n_0}$. 
Thus $Tx_0 = x_0$. 

Assume that $x_0$ and $x_0'$ are fixed-points of $T$. 
It follows that $T^{n_0}(x_0) = x_0$ and $T^{n_0}(x_0') = x_0'$, 
whence $x_0 = x_0'$. 
\end{proof}

\begin{example}[Existence and Uniqueness of Solution to ODE]
For a given point $(t_0, x_0$ in $\bR^2$ consider the initial value problem 
of ordinary differential equation 
\begin{equation}
    \label{equ:metric_spaces:fixed_point_theorem_ode}
    \left\{
    \begin{aligned}
        \frac{\diff x}{\diff t} &= f(t, x), \quad
        t \in \left[ t_0 - a, t_0 + a \right] \\
        x(t_0) &= x_0
    \end{aligned}
    \right.
\end{equation}
where $f$ is continuous and satisfies a Lipschitz condition with respect to 
$x$, \ie, 
\begin{equation*}
    \abs{f(t, x) - f(t, x')} \le K \abs{x - x'}. 
\end{equation*}
If $F(x, y, u)$ is of the form 
\begin{equation*}
    F(x, y, u) = K(x, y)u, 
\end{equation*}
then it is called an linear integral equation. 
We are looking for a solution $x = x(t)$ to Equation 
(\ref{equ:metric_spaces:fixed_point_theorem_ode}) such that $x(t)$ is 
differentiable and $(t, x(t)) \in S$ with the retangular area $S = \left\{ 
(t, x) \in \bR^2: \abs{t - t_0} \le a, \abs{x - x_0} \le b \right\}$ for 
given $a, b > 0$. 
Then we have the following result: 
\begin{thm}[The Picard-Lindelof Theorem]
\index{theorem!Picard-Lindelof \~{}}
\index{Picard-Lindelof theorem}
The Equation (\ref{equ:metric_spaces:fixed_point_theorem_ode}) has a unique 
solution defined on the interval $[t_0 - h, t_0 + h]$ where $h \in (0, b]$ 
satisfying $h K < 1$ and $hM \le b$ where $M = \max_{(t, x) \in S} 
\abs{f(t, x)}$. 
\end{thm}
\begin{proof}
\textbf{Step 1.}
The solution of Equation (\ref{equ:metric_spaces:fixed_point_theorem_ode}) 
is a solution to the integral equation 
\begin{equation}
    \label{equ:metric_spaces:fixed_point_theorem_ode:integral}
    x(t) = x_0 + \int_{t_0}^{t} f(x(\tau), \tau) \diff \tau, \quad 
    t \in [t_0 - a, t_0 - b]
\end{equation}
and vice versa. 

\textbf{Step 2.}
Let $E$ be the subspace of $X = C[t_0 - h, t_0 + h]$ such that 
$d(x - x_0) \le b$ and consider the integral operator $A: E \to E$ 
defined as  
\begin{equation*}
    Ax = x_0 + \int_{t_0}^{t} f(\tau, x(\tau)) \diff \tau, \quad
    t \in [t_0 - h, t_0 + h]
\end{equation*}
where $x_0(t) = x_0$. 
We then show that $A(M) \subseteq M$. 
Indeed, letting $x \in M$, then 
\begin{equation*}
        \abs{\int_{t_0}^{t} f(\tau, x(\tau)) \diff \tau} 
        \le \abs{t - t_0} \max_{(t, x) \in S} f(x, t)
        \le h M \le b, 
\end{equation*}
and hence 
\begin{equation*}
    d(Ax, x) = \max_{t \in [t_0 - h, t_0 + h]} 
    \abs{\int_{t_0}^{t} f(\tau, x(\tau)) \diff \tau} \le b. 
\end{equation*}

\textbf{Step 3.}
We now prove that $A$ is a contractive operator. 
By the assumption of Lipschitz continuity, 
\begin{equation*}
    \begin{aligned}
        d(Ax_1, Ax_2) &= \max_{t \in [t_0 - h, t_0 + h]}
        \abs{\int_{t_0}^{t} f(\tau, x_1(\tau))
        - f(\tau, x_2(\tau)) \diff \tau} \\ 
        &\le h K d(x_1, x_2) = k d(x_1, x_2) 
    \end{aligned}
\end{equation*}
where $k = hK < 1$. 
Hence $A$ is a $k$-contractive operator and there is a unique solution to 
Equation (\ref{equ:metric_spaces:fixed_point_theorem_ode}). 
\end{proof}
\end{example}

\begin{example}[Freholm Integral Equations]
Consider the integral equation 
\begin{equation}
    \label{equ:metric_spaces:fixed_point_theorem_integral_equation:integral}
    u(x) = f(t) + \lambda \int_{a}^{b} F(x, y, u(y)) u(y) \diff y, \quad 
    x \in [a, b].
\end{equation}
along with the iteration method 
\begin{equation}
    \label{equ:metric_spaces:fixed_point_theorem_integral_equation:iteration}
    u_{n + 1} = f(t) + \lambda \int_{a}^{b} F(x, y, u_n(y)) \diff y, \quad 
    x \in [a, b], n = 1, 2, \dots
\end{equation}
where $u_0 \equiv 0$ and $-\infty < a < b < \infty$. 
\begin{thm}
Assume the following:
\begin{enumerate}
    \item The function $f: [a, b] \to \bR$ is continuous. 
    \item The function $F: [a, b] \times [a, b] \times \bR \to \bR$ and the 
    partial derivative $F_u: [a, b] \times [a, b] \times \bR \to 
    \bR$ are continuous. 
    \item There is a number $L$ such that $\abs{F_u(x, y, u)} \le L$ for all 
    $x, y \in [a, b]$ and $u \in \bR$. 
    \item Let the real number $\lambda$ be given such that $(b - a) 
    \abs{\lambda} L < 1$. 
    \item Set $X = C[a, b]$. 
\end{enumerate}
Then the sequence $\{u_n\}$ constructed by 
(\ref{equ:metric_spaces:fixed_point_theorem_integral_equation:iteration}) 
converges to the unique solution of 
(\ref{equ:metric_spaces:fixed_point_theorem_integral_equation:integral}).
\end{thm}
\begin{proof}
Define the operator 
\begin{equation*}
    A(u)(x) \coloneqq f(x) + \lambda \int_{a}^{b} F(x, y, u(y)) \diff y, \quad
    \text{for all } x \in [a, b]. 
\end{equation*}
Then the original equation 
(\ref{equ:metric_spaces:fixed_point_theorem_integral_equation:integral}) 
corresponds to the fixed-point problem $u = Ax$. 
By the classical mean value theorem, 
\begin{equation*}
    \abs{F(x, y, u) - F(x, y, v)} \le L \abs{u - v}, 
\end{equation*}
which implies 
\begin{equation*}
    \begin{aligned}
        d(Au, Av) &= \max_{a \le x \le b} \abs{(Au)(x) - (Av)(x)} \\
        &= \max_{a \le x \le b} 
        \abs{ \lambda \int_{a}^{b} F(x, y, u(y)) - F(x, y, v(y)) \diff y}\\
        &\le \abs{\lambda} (b - a)L \max_{a \le x \le b} \abs{u(x) - v(x)} 
        = k d(x, u). 
    \end{aligned}
\end{equation*}
where $k \coloneqq \abs{\lambda} (b - a) L < 1$. 
Hence $A: X \to X$ is a $k$-contractive operator ad has a unique solution. 
\end{proof}
\end{example}

\begin{example}[Volterra Integral Equations]
Consider the Equation 
\begin{equation}
    x(t) = f(t) + \lambda \int_{a}^{t} K(t, s) x(s) \diff s, \quad 
    a \le t \le b
\end{equation}
where $K$ is continous on $[a, b] \times [a, b]$. 
Letting $M = \max_{a \le s, t \le b} \abs{K(t, s)}$, then we have following: 
\begin{thm}
Let $X = C[a, b]$ and define 
\begin{equation}
    (Tx)(t) = f(t) + \lambda \int_{a}^{b} K(s, t) x(s) \diff s, \quad 
    a \le t \le b.
\end{equation} 
Then $T$ has a unique fixed-point in $X$. 
\end{thm}
\begin{proof}
We shall prove that there exists some $n > 0$ such that $T^n$ is a 
contractive mapping. 
This is proved by showing 
\begin{equation}
    \label{equ:metric_spaces:fixed_point_theorem:volterra:target}
    \abs{(T^nx_1)(t) - (T^n x_2)(t)} \le \abs{\lambda}^n M^n 
    \frac{(t - a)^n}{n!}d(x_1, x_2). 
\end{equation}
We prove the above equation by induction as follows. 
For $n = 1$, 
\begin{equation*}
    \begin{aligned}
        \abs{(Tx_1)(t) - Tx_2(t)} &= \abs{\lambda} 
            \abs{\int_{a}^{t} K(t, s) (x_1(s) - x_2(s)) \diff s} \\
        &\le \abs{\lambda} M (t - a) d(x_1, x_2). 
    \end{aligned}
\end{equation*}
Supposing Equation (\ref{equ:metric_spaces:fixed_point_theorem:volterra:target}) 
holds true for $n$, then for $n + 1$, 
\begin{equation*}
    \begin{aligned}
        \abs{(T^{n+1})(x_1) - (T^{n+1}) (x_2)} 
        &= \abs{T\left( T^n x_1 \right)(t) - T\left( T^n x_2  \right)(t)} \\
        &= \abs{\lambda} \abs{\int_{a}^{t} K(t, s) 
            \left( (T^n x_1)(t) - (T^n x_2)(t) \right) \diff s
        } \\ 
        &\le \abs{\lambda} M \int_{a}^{t} 
            \abs{(T^n x_1)(t) - (T^n x_2)(t)} \diff s \\ 
        &\le \abs{\lambda} M \int_{a}^{t} 
            \frac{\abs{\lambda}^n M^n}{n!} d(x_1, x_2) \diff s \\
        & = \frac{\abs{\lambda}^{n+1} 
            M^{n+1}}{(n+1)!} (t - a)^{n+1} d(x_1, x_2). 
    \end{aligned}
\end{equation*}
Therefore, 
\begin{equation*}
    d(T^nx_1, T^nx_2) \le \frac{\abs{\lambda}^n M^n}{n!} (b - a)^n 
        d(x_1, x_2) = k_n d(x_1, x_2)
\end{equation*}
where $k_n \coloneqq \frac{\abs{\lambda}^n M^n}{n!} (b - a)^n$. 
Since $k_n \to \infty$ as $n \to \infty$, $T^n$ is a contractive mapping 
for $n$ big enough. 
\end{proof}
\end{example}
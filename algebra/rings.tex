\chapter{Rings}
%%%%%%%%%%%%%%%%%%%%%%%%%%%%%%%%%%%%%%%%%%%%%%%%%%%%%%%
%%  Rings and Homomorphisms
%%%%%%%%%%%%%%%%%%%%%%%%%%%%%%%%%%%%%%%%%%%%%%%%%%%%%%%
\section{Rings and Homomorphisms}
\begin{defn}
A ring is a set $\fR$ together with two binary operations (usually denoted 
as addition and multiplication) such that 
\begin{enumerate}
    \item $\fR$ is an \emph{abelian group} under addition;
    \item $\fR$ is a semigroup under multiplication;
    \item $a (b + c) = ab = ac$ and $(a + b)c = ac + bc$ for any $a, b, c 
    \in \fR$ (left and right distributive laws).
\end{enumerate}

In addition, if $ab = ba$ for all $a, b \in \fR$, then $\fR$ is said to 
be commutative; if $\fR$ has an element $1_\fR$ such that 
\begin{equation*}
    1_\fR r = r 1_\fR = r \text{ for any } r \in \fR, 
\end{equation*}
then $\fR$ is said to be a ring with identity.
\end{defn}

\begin{defn}
\index{Left!\~{} zero divisor}
\index{Right!\~{} zero divisor}
\index{Zero divisor}
\index{Integral domain}
A nonzero element $a$ in a ring $\fR$ is called a \emph{left (\resp right) 
zero divisor} if $ab = 0$ (\resp $ba = 0$) for some nonzero element 
$b \in \fR$. 
An element of $\fR$ is a \emph{zero divisor} if it is both a left and right 
zero divisor. 

A commutative ring with identity $1_\fR \neq 0$ is called an \emph{integral 
domain} if it has no zero divisors. 
\end{defn}
Here is a direct and useful property of rings with no zero divisors. 
\begin{prop}
A ring $\fR$ has no zero divisors if and only if the left and right 
cancellation laws hold in $\fR$, that is, for any $a, b, c \in \fR$ with 
$a \neq 0$, 
\begin{equation}
    ab = ac \text{ or } ba = ca \implies a = c.
\end{equation} 
\end{prop}

\begin{defn}
\index{invertible}
\index{unit}
\index{left!\~{} invertible}
\index{right!\~{} invertible}
\index{left!\~{} inverse}
\index{right!\~{} inverse}
An element $a$ in a ring $\fR$ is said to be \emph{left (\resp right) 
invertible} if there exists $b \in \fR$ such that $ab = 1_\fR$ (\resp $ba = 
1_\fR$). 
The element $b$ of $\fR$ is said to be a \emph{left (\resp right) inverse} 
of $a$. 
An element $a \in \fR$ is \emph{invertible} or called a \emph{unit} if it 
is both left and right invertible. 
\end{defn}

\begin{defn}
\index{principal!\~{} ideal}
\index{principal!\~{} ideal ring}
\index{principal!\~{} ideal domain}
Let $X$ be a subset of a ring $\fR$. The intersection of all [left] ideals 
containing $X$ is called the [left] \emph{ideal generated by $X$}, denoted 
by $(X)$. The elements of $X$ are called the generators of the ideal $(X)$. 

An ideal $(a)$ generated by a single element $a \in \fR$ is called a 
\emph{principal ideal}. 
A \emph{principal ideal ring} is a ring in which every ideal is principal.  
A principal ideal ring which is an integral domian is called a 
\emph{principal ideal domain}. 
\end{defn}

%%%%%%%%%%%%%%%%%%%%%%%%%%%%%%%%%%%%%%%%%%%%%%%%%%%%%%%
%%  Ideals
%%%%%%%%%%%%%%%%%%%%%%%%%%%%%%%%%%%%%%%%%%%%%%%%%%%%%%%
\section{Ideals}
In the theory of rings, ideals play an analogous role to normal subgroups 
in the theory of groups. 

\begin{defn}
\index{subring}
\index{ideal}
Let $\fR$ be a ring and $\fS$ be a nonempty subset of $\fR$ that is closed 
under addition and multiplication of $\fR$. 
If $\fS$ is itself a ring under these operations, then $\fS$ is called a 
\emph{subring} of $\fR$. 

A subring $\fI$ of a ring $\fR$ is a \emph{left ideal} in $\fR$ if 
\begin{equation}
    r \in \fR, x \in \fI \implies rx \in I;
\end{equation}
$\fI$ is a \emph{right ideal} if 
\begin{equation}
    r \in \fR, x \in \fI \implies xr \in I.
\end{equation}
$\fI$ is an ideal if it is both a left and right ideal. 
\end{defn}
\begin{example}
    The center of a ring $\fR$ is the set $\fC = \{c \in \fR: cr = rc 
    \text{ for any } r \in \fR\}$. Then $\fC$ is a subring of $\fR$ but may 
    not be an ideal. Notice that $fC = \fR$ if and only if $\fR$ is a 
    commutative ring. 
\end{example}

There is a very useful characterization of ideals:
\begin{thm}
\label{thm:characterization_of_ideals}
A nonempty subset $\fI$ of of a ring $\fR$ is a left ideal if and only if 
for all $a, b \in \fI$ and $r \in \fR$, $a - b \in \fI$ and $ra \in \fI$. 
\end{thm}
\begin{proof}
It suffices to prove that $\fI$ is a subring of $\fR$ if and only if 
$a - b \in \fI$ and $ab \in \fI$ for any $a, b \in \fI, r \in \fR$. 
Let $a \in \fI$. We have $0 = a - a \in \fI$ and $-a = 0 - a \in \fI$. 
Thus $a + b = a - (-b) \in \fI$ for $a, b \in \fI$. 
Consequently, $\fI$ is a subring of $\fR$.
\end{proof}
\begin{thm}
\label{thm:principal_ideal}
Let $\fR$ be a ring, $a \in \fR$ and $X \subseteq \fR$. Then 
\begin{enumerate}
    \item \label{thm:principal_ideal:1}
    The principal ideal $(a)$ consists of all elements of the form 
    \begin{equation}
        ra + as + na + \sum_{i=1}^{m}r_ias_i \quad 
        r, s, r_i, s_i \in \fR; m \in \bN^\ast \text{ and } n \in \bZ.
    \end{equation}
    \item \label{thm:principal_ideal:2}
    If $\bR$ has an identity, then $(a) = \left\{ 
        \sum_{i=1}^m r_i a s_i : r_i, s_i \in \bR, m \in \bN^\ast 
    \right\}$.
    \item \label{thm:principal_ideal:3}
    If $a$ is in the center of $\fR$ (in particular, $\fR$ is a commutative
    ring), then $(a) = \{ra + na: r \in \fR, n \in \bZ\}$.
    \item \label{thm:principal_ideal:5}
    If $\fR$ has an identity and $a$ is in the center of $\fR$, then $\fR a 
    = (a) = a \fR$.
\end{enumerate}
\end{thm}
\begin{proof}[Proof of Theorem \ref{thm:principal_ideal}]
\ref{thm:principal_ideal:1} Put 
\begin{equation}
    \fI = \left\{
        ra + as + na + \sum_{i=1}^m r_i a s_i: r, s, r_i, s_i \in \fR, 
        m \in \bN^\ast
    \right\}.
\end{equation}
Obviously, $\fI$ is a subring of $\fR$ and $a \in \fI$. On the other hand, 
assume that $\fI'$ is a ideal containing $a$, then by definition of ideals, 
elements of the form $ra$, $as$, $ns$ are contained in $\fI'$ whence $ras 
\in \fI'$ for any $r, s \in \fI'$. Thereafter, 
\begin{equation}
    x = ra + as + na + \sum_{i=1}^m r_i a s_i: r, s, r_i, s_i \in \fR, 
    m \in \bN^\ast
\end{equation}
is an element of $\fI'$. Consequently, $\fI \subseteq \fI'$ and $\fI = (a)$.

\ref{thm:principal_ideal:2} \ref{thm:principal_ideal:5} Trivial.
\end{proof}

\begin{rmk}
If $\fR$ is a commutative ring and $a, b \in \fR$, then $(a)(b) \subseteq 
(ab)$.
\end{rmk}

Let $A_1, A_2, \ldots, A_n$ be a family of subsets of ring $\fR$. Denote 
by $A_1 + A_2 + \ldots + A_n$ the set 
\begin{equation*}
    \left\{
        \sum_{i=1}^n a_n: a_i \in A_i \text{ for all } i = 1, 2, \ldots, n
    \right\}.
\end{equation*}
Likewise, denote by $A_1 A_2 \cdots A_n$ the set 
\begin{equation*}
    \left\{
        a_1 a_2 \cdots a_n: a_i \in A_i \text{ for all } i = 1, 2, \ldots, n
    \right\}.
\end{equation*}

As we have stated before, ideals is a basic tool to study the structure of 
rings as normal subgroups in the theory of groups. 
More specifically, ideals can be used to define quotient rings. 
\begin{defn}

\end{defn}


\subsection{Prime Ideals and Maximal Ideals}

\begin{defn}
\index{Prime ideal}
\index(Ideal!Prime \~{})
An ideal $P$ of a ring $\fR$ is said to be \emph{prime} if $P \neq \fR$ 
and for any ideals $A, B$ of $\fR$ 
\begin{equation}
    AB \subseteq P \implies A \subseteq P \text{ or } B \subseteq P.
\end{equation}
\end{defn}

Here is a useful characterization of prime ideals.
\begin{thm}
\label{thm:prime_ideal_element}
If $P$ is an ideal in a ring $\fR$ such that $P \neq \fR$ and for any 
$a, b \in \fR$ 
\begin{equation}
    \label{equ:characterization_prime_ideal}
    ab \in P \implies a \in P \text{ or } b \in P, 
\end{equation}
then $P$ is prime. Conversely, if $\fR$ is commutative and $P$ is prime, 
then $P$ satisfies condition (\ref{equ:characterization_prime_ideal}).
\end{thm}

\begin{proof}[Proof of Theorem \ref{thm:prime_ideal_element}]
We first prove the former statement. 
If $P$ is not prime, then there exist ideals ideals $A, B$ in $\fR$ such 
that  $AB \subseteq P$ and $A \not \subseteq P$ and $B \not \subseteq P$. 
Thus there exist elements $a \in A $ and $b \in B$ such that $a \not \in 
P$ and $b \not \in P$. Clearly, $ab \in AB \subseteq P$. 
However, according to condition (\ref{equ:characterization_prime_ideal}), 
we deduce that $a \in P$ or $b \in P$, which is a contradiction. So $P$ is 
prime. 

Now we prove the second statement. Assume that elements $a, b \in fR$ 
satisfy $ab \in P$. Then $(a)(b) \subseteq (ab)$ since $\fR$ is commutative. 
By definition, $(ab) \subseteq P$, whence $(a)(b) \subseteq P$. Because $P$ is 
prime, we have $(a) \subseteq P$ or $(b) \subseteq P$, which implies $a \in b$ 
or $b \in P$.
\end{proof}

The following theorem illustrates the functionality of prime ideals in 
the study of rings. 
\begin{thm}
\label{thm:functionality_prime_ideals}
Let $\fR$ be a commutative ring with identity $1_\fR \neq 0$ and $P$ is an 
ideal in $\fR$. 
Then $P$ is a prime ideal in $\fR$ if and only if the quotient ring $\fR / 
P$ is an integral domain. 
\end{thm}

\begin{defn}
\index{Maximal ideal}
\index{Ideal!Maximal \~{}}
An ideal (\resp left ideal) $M$ in a ring $\fR$ is called \emph{maximal} 
provided that $M \neq \fR$ and for any ideal (\resp left ideal) $N$ in $\fR$
\begin{equation}
    M \subseteq N \subseteq \fR \implies N = M \text{ or } N = \fR.
\end{equation}
\end{defn}

\begin{rmk}
According to the definition above, $M$ is a maximal element of the set of 
all ideals except $\fR$. Sometimes we say an ideal $I$ is maximal with 
respect to a property, meaning that $I$ is a maximal element of the set of 
all ideals with the given property except $\fR$. 
\end{rmk}

\begin{thm}
\label{thm:rings:ideals:prime_maximal_ideals:connection_prime_maximal_ideals}
If $\fR$ is a commutative ring such that $\fR^2 = \fR$ (in particular, 
$\fR$ has identity), then every maximal ideal $M$ in $\fR$ is prime. 
\end{thm}

\begin{thm}
\label{thm:rings:existence_of_maximal_ideals}
In a nonzero ring $\fR$ with identity maximal [left] ideals always exist. 
In fact every [left] ideal $\fI \neq \fR$ is contained in a maximal [left] 
ideal. 
\end{thm}
%%%%%%%%%%%%%%%%%%%%%%%%%%%%%%%%%%%%%%%%%%%%%%%%%%%%%%%
%% Factorization in Commutative Rings
%%%%%%%%%%%%%%%%%%%%%%%%%%%%%%%%%%%%%%%%%%%%%%%%%%%%%%%
\section{Factorization in Commutative Rings}
\subsection{Divisors}
Let's extend the definitions of divisibility, greatest common 
divisor and prime in the ring of integers to arbitrary commutative 
rings. 

\begin{defn}
\index{factor}
\index{factor!proper \~{}}
\index{proper!\~{} factor}
\index{divide}
\index{associate}
A nonzero element $a$ of a commutative ring $R$ divides an element
$b \in R$, denoted by $a \mid b$ if there exists an element $x \in R$ such 
that $b = ax$. 
If $a$ divides $b$, then we say $a$ is a factor of $b$. 
If $a$ is a factor of $b$ while $b$ is not a factor of $a$, then $a$ is 
said to be a \emph{proper factor} of $a$. 
Elements $a$, $b$ are said to be associates if $a \mid b$ and $b \mid a$, 
denoted as $a \sim b$.
\end{defn}

\begin{thm}
\label{thm:basic_facts_on_divisibility}
Let $a$, $b$, $u$ be elements of a commutative ring $R$. Then 
\begin{enumerate}
    \item \label{thm:basic_facts_on_divisibility:1}
    $a \mid b$ if and only if $(b) \subseteq (a)$. 
    \item \label{thm:basic_facts_on_divisibility:2}
    $a$ and $b$ are associates if and only if $(a) = (b)$.
    
    \item \label{thm:basic_facts_on_divisibility:4}
    $u$ is a unit if and only if $(u) = R$.
    
\end{enumerate}
\end{thm}
\begin{proof}[Proof of Theorem \ref{thm:basic_facts_on_divisibility}]
\ref{thm:basic_facts_on_divisibility:1}
By Theorem \ref{thm:principal_ideal} \ref{thm:principal_ideal:5}, 
$(a) = \fR a = a \fR$ and $(b) = \fR b = b \fR$.

\ref{thm:basic_facts_on_divisibility:2} Obvious.

\ref{thm:basic_facts_on_divisibility:4}
$a$ is a unit $\iff$ $a^{-1}$ exists $\implies$ $r = r a^{-1} a 
\in \fR a = (a)$ for any $r \in \fR$. 
Conversely, $(a) = \fR \implies 1_\fR = ra$ for some $r \in \fR$.  
\end{proof}

\begin{defn}
\index{greatest common divisor}
\index{GCD}
Let $X$ be a subset of a commutative ring $\fR$. An element $d \in \fR$ 
is said to be a \emph{greatest common divisor (GCD)} of $X$ if 
\begin{enumerate}
    \item $d \mid x$ for all $x \in X$; 
    \item $c \mid x$ for all $x \in X \implies c \sim d$. 
\end{enumerate}
\end{defn}
Greatest common divisor does not always exist. 

\begin{defn}
Let $\fR$ be a commutative ring with identity. An element $c \in \fR$ is 
irreducible if 
\begin{enumerate}
    \item $c$ is a nonzero nonunit;
    \item $c = ab \implies a \text{ or } b \text{ is a unit}$, for any 
    $a, b \in \fR$.
\end{enumerate}
An element $p$ of $\fR$ is prime if 
\begin{enumerate}
    \item $p$ is a nonzero nonunit;
    \item $p \mid ab \implies p \mid a \text{ or } p \mid b$, for any 
    $a, b \in \fR$.
\end{enumerate}
\end{defn}

There is a close connection between prime (\resp irreducible) elements and 
prime (\resp maximal) principal ideals in $\fR$. 

\begin{thm}
\label{thm:connection_between_principal_ideals}
Let $p$ and $c$ be nonzero elements in an integral domain $\fR$. 
\begin{enumerate}
    \item \label{thm:connection_between_principal_ideals:1} 
    $p$ is prime if and only if $(p)$ is a nonzero prime ideal. 
    \item \label{thm:connection_between_principal_ideals:2}
    $c$ is maximal if and only if $(c)$ is maximal in the set $S$ 
    of all proper principal ideals of $\fR$.
    \item \label{thm:connection_between_principal_ideals:3}
    Every prime element of $\fR$ is irreducible. 
    \item \label{thm:connection_between_principal_ideals:4}
    If $\fR$ is a principal ideal domain, then $p$ is prime if and only if 
    $p$ is irreducible. 
    \item \label{thm:connection_between_principal_ideals:6}
    The only divisors of an irreducible element are its associates and units.
\end{enumerate}
\end{thm}

\begin{proof}[Proof of Theorem \ref{thm:connection_between_principal_ideals}]
\ref{thm:connection_between_principal_ideals:2} 
Assume that $c$ is an irreducible element of $\fR$. 
Then $(c)$ is a proper principal ideal in $\fR$ since $(c)$ is a nonunit. 
Let $a$ be an element in $\fR$ such that $(a)$ is a proper principal ideal 
in $\fR$ such that $(c) \subseteq (a)$. 
$a$ is a nonunit by Theorem \ref{thm:basic_facts_on_divisibility}
\ref{thm:basic_facts_on_divisibility:4}. 
By Theorem \ref{thm:principal_ideal} \ref{thm:principal_ideal:5}, $c = ra$ 
for some $r \in \fR$. Hence $r$ is a unit because of the irreducibility of 
$c$. Thus $a = r^{-1} c$ and $r' a = r' r^{-1} c \in (c)$ for any $r' \in 
\fR$, meaning that $(a) \subseteq (c)$. 
Conversely, if $(c)$ is maximal in $S$, then $c$ is a nonzero nonunit. 
Suppose that $c = ab$ and $a$ is a nonunit, which means that $(c) \subseteq 
(a)$ and $(a) \neq \fR$, whence $(a) = (c)$. Hence $a = rc$ for some $r 
\in \fR$. 
Consequently, we have $c = ab = rbc$. 
Since $\fR$ is an integral domain, $1_\fR = rb$, whence $b$ is a unit. 
\sidenote{Why?}

\ref{thm:connection_between_principal_ideals:3} 
$p$ is prime and $p = ab \implies p \mid a$ or $p \mid b$; say $p \mid a$. 
THus $pr = a$ for some $r \in \fR$, which implies $p = prb$. 
Since $\fR$ is an integral domain, $rb = 1_\fR$. 
Therefore,  $b$ is a unit and $p$ is irreducible. 

\ref{thm:connection_between_principal_ideals:4}
It suffices to prove that if $\fR$ is prime and $p$ is irreducible, then 
$p$ is prime. 
Before that, we prove that if $\fR$ is a principal ideal domain, and 
$(p)$ is a maximal ideal in the set of all proper principal ideals, then 
$(p)$ is a maximal ideal in $\fR$. 
Suppose that there exist a proper ideal $X$ in $\fR$ such that 
$(p) \subseteq X$. For any $x \in X$, $(x) \subseteq X$. 

\ref{thm:connection_between_principal_ideals:6}
Trivial. 
\end{proof}

\subsection{Unique Factorization Domains}
\begin{defn}
\index{unique factorization domain}
\index{UFD}
An integral domain $\fR$ is called a \emph{unique factorization domain 
(UFD)} provided that 
\begin{enumerate}
    \item every nonzero nonunit element $a \in \fR$ can be written 
    $c_1c_2\cdots c_n$ with $c_1, c_2, \ldots, c_n$ irreducible. 
    (Existence of factorization.)
    \item If $a \in \fR$ can be factorized as $a = c_1c_2\cdots c_n$ and 
    $a = d_1d_2 \cdots d_m$ with $c_i$ and $d_j$ irreducible, $i=1, 2, 
    \ldots, n$, $j=1, 2, \ldots, m$, then $n = m $ and for some permutation 
    $\sigma$ of $\{1, 2, \ldots, n\}$, $c_i \sim d_{\sigma(i)}$. 
    (Uniqueness of factorization.)
\end{enumerate}
\end{defn}

As mentioned in Theorem \ref{thm:connection_between_principal_ideals}, a 
prime element in a integral domain is irreducible. 
For unique factorization domain, the converse is true. 
Consequently, irreducible and prime elements coincide in UFD. 
\begin{prop}
\label{prop:rings:irreducible_prime_UFD}
Every irreducible element in a unique factorization domain is prime. 
\end{prop}
\begin{proof}
Suppose that $p$ is an irreducible element in a UFD $\fR$ and $c \mid ab$, 
for nonunits $a, b \in \fR$, which is equivalent to $pr = ab$ for some 
$r \in \fR$. 
By definition of UFD, $a$, $b$ and $r$ can be factorized as $a = c_1c_2 
\cdots c_n$, $b = d_1d_2 \cdots d_m$ and $r = q_1q_2 \cdots c_l$. 
Then we have, $pq_1q_2 \cdots q_l = c_1c_2 \cdots c_n d_1 d_2 \cdots d_m$. 
Hence, with uniqueness of factorization, $p$ associates with some $c_i$ or 
$d_j$, $i=1, 2, \ldots, n$ and $j=1, 2, \ldots, m$. Thus $p \mid a$ or 
$p \mid b$. 
\end{proof}

From Theorem \ref{thm:connection_between_principal_ideals} we know for 
principal ideal domain, prime and irreducible elements do coincide as the 
case for UFD. 
It seems plausible that every principal ideal domain is a UFD. 
In order to prove that this is indeed the case, we first prove the following 
lemma. 

\begin{lemma}
\label{lemma:rings:finite_chain_of_principal_ideal_domain}
If $\fR$ is a principal ideal ring and $(a_1) \subseteq (a_2) \subseteq \cdots$ 
is a chain of ideals in $\fR$, then for some positive integer $n$, $(a_i) = 
(a_{n})$ for all $i > n$. 
\end{lemma}
\begin{proof}
Let $\fA = \bigcup_{i=1}^\infty (a_i)$. We contend that $\fA$ is an ideal. 
Indeed, for any $a, b \in \fA$, $a \in (a_i), b \in (a_j)$ for some positive 
integers $i, j$. 
Assuming that $i \le j$, $a \in (a_j)$ and $a - b \in (a_j) \subseteq \fA$. 
Furthermore, for any $r \in \fR$ we have $ra \in \fR a \subseteq (a_i) \subseteq 
\fA$.

Since $\fR$ is a principal ideal domain, $\fI = (d)$ for some $d \in \fR$. 
Obviously, $d \in \fI$ thus $d$ belonging to some $(a_n)$, $n \in 
\bZ^\ast_+$. 
Hence, $(d) \subseteq (a_i)$ for all $i \ge n$. 
Therefore, $(a_i) = \fA$ for all $i \ge n$. 
\end{proof}

\begin{thm}
\label{thm:principal_domain_UFD}
Every principal ideal domain $\fR$ is a unique factorization domain. 
\end{thm}
\begin{proof}
Let $S$ be the set of all nonzero nonunit elements of $\fR$ that cannot 
be factored a finite product of irreducible elements. 
\underline{We shall first show that $S$ is empty. }
Suppose that $S$ is nonempty and $a \in S$. 
Then $(a) \subsetneqq \fR$ by Theorem \ref{thm:basic_facts_on_divisibility:4} 
and is contained in a maximal ideal $(c)$ by Theorem 
\ref{thm:rings:existence_of_maximal_ideals}. 
Hence, $c$ is a irreducible element of $\fR$ by Theorem 
\ref{thm:basic_facts_on_divisibility} \ref{thm:basic_facts_on_divisibility:2}. 
Therefore, $(a) \subseteq (c)$ implies that $c$ divides $a$, \ie, $a = cx$ for 
some $x \in \fR$. 
\underline{We contend that $x \in S$.} 
Indeed, if $x$ were a unit, then $a, c$ are associates and thus $a$ is 
irreducible, contradicting the hypothesis that $a \in S$. 
If $x$ were a nonunit and $x \not \in S$, then $x$ has a finite 
factorization, whence $a$ does, which is also a contradiction. 
Hence, $x \in S$. 
Furthermore, \underline{we claim that the ideal $(a) \subsetneqq (x)$. }
Indeed, since $x \mid a$, we have $(a) \subseteq (x)$. 
However, if $(a) = (x)$ were true, we deduce that $x = ay$ for some $y \in S$. 
Thus, $a = cx = cay$ and $cy = 1$ by cancellation law, which contradicts the 
irreducibility of $c$. 

With Axiom of Choice \sidenote{To be completed.}, we can choose a 
$x_a \in \fR$ for each $a \in S$ in the way given in the proceeding 
paragraph and denote the choice function by $f: S \to S, a \mapsto x_a$. 
By Recursion Theorem, we have a unique function $\phi: \bN \to S$ such that 
$\phi(0) = a$ and $\phi(n+1) = f(\phi(n))$. 
We can denote $\phi(n)$ by $a_n$, and then we obtain a sequence $a_0, a_1, 
\ldots$ of elements of $S$ such that 
\begin{equation}
    (a_0) \subsetneqq (a_1) \subsetneqq (a_2) \subsetneqq \cdots.
\end{equation}
But by Lemma \ref{lemma:rings:finite_chain_of_principal_ideal_domain}, 
this is not allowed. 
Therefore, $S$ must be empty, whence every nonzero nonunit element of $\fR$ 
can be factored as af finite product of irreducible elements. 

\underline{Finally, we need to prove the uniqueness of factorization. }
If $c_1c_2\cdots c_n = c = d_1d_2\cdots d_m$ where $c_i, d_j$ are 
irreducible, then $c_1$ divides $d_j$ for some $j=1, 2, \ldots, m$. 
Without loss of generality, assume $j = 1$. 
Since the factors of an irreducible element are either units or its 
associates (Theorem \ref{thm:connection_between_principal_ideals}
\ref{thm:connection_between_principal_ideals:6}), we have $c_1$ and $d_1$ 
are associates. 
The proof of uniqueness is now completed by a routine inductive argument. 
\end{proof}

Although greatest common divisor may not exist for some commutative ring 
$\fR$, in the case of uniqueness factorization domain, greatest common 
divisor does exists for any finite subset of $\fR$. 
\begin{thm}
\label{thm:rings:existence_of_GCD_in_UFD}
Let $\fR$ is a unique factorization domain. 
If $X = \{a_1, a_2, \cdots, a_n\} \subseteq \fR$, then there exits a greatest 
common divisor of $X$.
\end{thm}
\begin{proof}
$a_i$ can be factored as $a_i = c_1^{s_{i, 1}} c_2^{s_{i, 2}} \cdots 
c_m^{s_{i, m}}, i=1, 2, \ldots, n$. 
Set $t_j = \min \{s_{1, j}, s_{2, j}, \ldots, s_{n, j}\}$ for $j = 1, 2, 
\ldots, m$.
We can prove that $c = c_1^{t_1} c_2^{t_2} \cdots c_m^{t_m}$ is a greatest 
common divisor of $X$. 
It is trivial to prove that $c$ is a common divisor of $X$. 
Suppose that $d$ is a common divisor of $X$, and now we prove that $d$ is a 
divisor of $c$, which can be done by establishing the fact that any common 
divisor of $X$ is of the form $c_1^{k_1} c_2^{k_2} \cdots c_m^{k_m}$ with 
$k_j \le t_j, j=1, 2, \ldots, m$.
\end{proof}

Apart from the condition of uniqueness of factorization, we have more 
conditions that characterize unique factorization domain: 

\begin{thm}
Let $\fR$ be an integral domain. Then the following statements are 
equivalent: 
\begin{enumerate}
    \item $\fR$ is a uniqueness factorization domain. 
    \item $\fR$ satisfies the condition of existence of factorization 
    and the hypothesis of Proposition \ref{prop:rings:irreducible_prime_UFD}. 
    \item $\fR$ satisfies the condition of existence of factorization 
    and the hypothesis of Theorem \ref{thm:rings:existence_of_GCD_in_UFD}.
\end{enumerate}
\end{thm}

\subsection{Euclidean Domains}
\begin{defn}
\index{Euclidean ring}
\index{Euclidean domain}
Let $\fR$ be a commutative ring. $\fR$ is called a \emph{Euclidean ring} 
if there exists a function $\phi: \fR^\ast = \fR - \{0\} \to \bN$ such that:
\begin{enumerate}
    \item if $a, b \in \fR$ and $ab \neq 0$, then $\phi(a) \le \phi(ab)$;
    \item if $a, b \in \fR$ and $b \neq 0$, then there exist $q, r \in \fR$ 
    such that $a = qb + r$ with $r = 0$, or $r \neq 0$ and $\phi(r) \le 
    \phi(b)$. 
\end{enumerate}
A Euclidean ring is called a \emph{Euclidean domain} if it is an integral domain. 
\end{defn}

\begin{example}
The ring $\bZ$ of integers is a Euclidean domain with $\phi(x) = \abs{x}$. 
\end{example}

\begin{example}
The ring $\fF[x]$ of polynomials over a field $\fF$ is a Euclidean domain 
$\phi(f) = \deg f$ for $f \neq 0$. 
\end{example}

\begin{example}
\index{Gaussian intergers}
The ring $\bZ[\sqrt{-1}] = \{a + b\sqrt{-1}: a, b \in \bZ\}$, known as 
\emph{Gaussian integers} is a Euclidean domain with $\phi(a + b\sqrt{-1}) 
= a^2 + b^2$.
It is easy to verify that $\phi(\alpha \beta) = \phi(\alpha) \phi(\beta)$ 
for any $\alpha, \beta \in \bQ[\sqrt{-1}]$. 
Suppose $\beta \neq 0, \beta \in \bZ[\sqrt{-1}]$ and $\alpha \in 
\bZ[\sqrt{-1}]$. 
Then $\beta^{-1} \in \bQ[\sqrt{-1}]$ and assume that $\alpha \beta^{-1}$ 
\begin{equation*}
    \alpha \beta^{-1} = u + v \sqrt{-1} \in \bQ[\sqrt{-1}].
\end{equation*}
There exist $c, d \in \bZ$ such that $\abs{c - u} \le \frac{1}{2}$ and 
$\abs{d - v} \le \frac{1}{2}$. 
Setting $x = u - c$ and $y = v - d$, we have $\alpha = \beta \left( 
(x + c) + (y + d) \sqrt{-1}\right)$. 
Then we have $\alpha = \beta \theta + \gamma$ with 
\begin{equation}
    \theta = c + d\sqrt{-1} \text{ and } \gamma = \beta(x + y \sqrt{-1}) = 
    \alpha - \beta (c + d \sqrt{-1}) = \alpha - \beta \theta.
\end{equation}
If $\gamma \neq 0$, then $\pi(r) = \phi(\beta (x + y\sqrt{-1})) 
= \phi(\beta) \phi(x + y \sqrt{-1}) = \phi(\beta) \left( x^2 + y^2 \right) 
\le \frac{1}{2} \phi(\beta) \le \phi(\beta)$. 
\end{example}

\begin{thm}
Every Euclidean domain $\fR$ is a principal ideal domain. 
\end{thm}
\chapter{$L^p$ Spaces}
\label{chp:lp_spaces}
\section{$L^2$ Spaces}
\begin{defn}[$L^p$ space]
\index{$L^p$ space}
\index{space!$L^p$ \~{}}
Suppose $a$, $b \in \bR$, $p \in [1, \infty)$ and $-\infty < a < b < \infty$. 
Let $L^p([a, b])$ denote all the Lebesgue measurable functions 
$u: [a,b] \to \bR$ such that 
\begin{equation}
    \int _a ^b \abs{u(x)} ^p \diff x < \infty.
\end{equation}
When not leading to ambiguousness, we may omit the domain where the 
functions are defined, just writing $L^p$.
\end{defn}

Addition and scalar multiplication can be introduced as the ordinary 
pointwise addition and scalar multiplication of functions. 
Naturally, the domain of $L^p$ can be easily extend to $\bR ^n$, $n \ge 1$.

\begin{thm}
\label{thm: l2inl1}
Square-integrable functions are absolutely integrable, \ie, $L^2 \in L^1$.
\end{thm}
\begin{proof}
Proof is easily obtained by the inequality
\begin{equation}
    \abs{u} \le \frac{1 + u^2}{2}.
\end{equation}
\end{proof}
\begin{thm}
    $L^p$ space is a linear space.
\end{thm}
\begin{proof}
Suppose $u, v \in L_2$. From the basic inequality, we have 
\begin{equation}
    \abs{uv} \le \frac{u^2 + v^2}{2}.
\end{equation}
Along with theorem \ref{thm: l2inl1}, $uv$ is square-integrable.
\begin{equation}
    (u + v) ^2 = u ^2 + 2uv + v^2, 
\end{equation}
which means $u+v$ is a square-integrable function, \ie, $u + v \in L_2$.

Obviously, $\alpha u$ is square-integrable for any $alpha \in \bR$.
\end{proof}

By introducing inner product to $L_2$, we make $L_2$ a Hilbert space. 
Define
\begin{equation}
\label{equ:l2inp}
    \inp{u}{v} = \int _a ^b u(x)v(x) \diff x.
\end{equation}
By induction, norm on $L_2$ is given as 
\begin{equation}
    \norm{u} = \sqrt{\inp{u}{u}}.
\end{equation}

\begin{thm}
With inner product defined in \ref{equ:l2inp}, $L_2$ is a Hilbert space.
\end{thm}
\begin{proof}
Clearly, formula \ref{equ:l2inp} is well-defined inner product of $L_2$. 
Then the only thing left is to prove $L_2$ is complete. 
\end{proof}
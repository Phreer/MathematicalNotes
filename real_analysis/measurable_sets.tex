\chapter{Measurable Sets}
\section{Classes of Sets}
To establish measure on a set $X$, we should give a set of subsets of $X$ 
(also call a set of sets a class) that are measurable. 
For the class of measurable sets, we should impose some constraints on them. 

\begin{defn}
\index{$\pi$-system of sets}
A $\pi$-system on a set $X$ is a nonempty class $\mathcal{P}$ of subsets in 
$X$ that are closed under intersections, \ie, 
\begin{equation*}
    A, B \in \mathcal{P} \implies A \cap B \in \mathcal{P}.
\end{equation*}
\end{defn}

\begin{example}
The class $\mathcal{P}_\bR = \{(-\infty, a]: a \in \bR\}$ is a $\pi$-system. 
In fact, class that consists of all open intervals is a $\pi$-system. 
So do left open right closed intervals and left closed right open intervals.
\end{example}

\begin{defn}
\index{semiring of sets}
A \emph{semiring of sets} is a $\pi$-system $\mathcal{D}$ on a set $X$ 
such that for any $B \subseteq A \in \mathcal{D}$, there exists a finite 
disjoint subclass $\{C_k \in \mathcal{D}: k = 1, 2, \ldots, n\}$ of 
$\mathcal{D}$ such that 
\begin{equation*}
    A \backslash B = \bigcup_{k=1}^n C_k. 
\end{equation*}
\end{defn}

Since semirings are $\pi$-systems, for any $A, B$ in a semiring $\cD$, 
$A \backslash B = A \backslash (A \cap B) = \bigcup_{k=1}^n C_k$, 
$C_k \in \cD, k = 1, 2, \ldots, n$. 

\begin{example}
The class $\mathcal{D}_\bR = \{(a, b]: a, b \in \bR\}$ is a semiring on 
$\bR$.
\end{example}

\begin{example}
Let $X$ be a finite set. Then the class of all sets of single point is a 
semiring on $X$. 
\end{example}
\begin{defn}
\index{ring of sets}
A \emph{ring of sets} on a set $X$ is a nonempty class $\mathcal{R}$ of sets 
of $X$ that is closed under the union and difference of pairs of sets, \ie, 
\begin{equation*}
    A, B \in \mathcal{R} \implies A \cup B \in \mathcal{R}, 
    A \backslash B \in \mathcal{R}. 
\end{equation*}
\end{defn}

\begin{example}
The class $\mathcal{R}_\bR = \bigcup_{n=1}^\infty 
\left\{ \bigcup_{k=1}^n (a_k, b_k]: a_k, b_k \in \bR \right\}$ is a ring on 
$\bR$. 
\end{example}

\begin{prop}
\label{prop:measurable_sets:ring_semiring}
A ring is a semiring. 
\end{prop}
\begin{proof}
Let $A, B$ be elements of a ring $\mathcal{R}$. 
Then $A \backslash B \in \mathcal{R}$, $B \backslash A \in \mathcal{R}$ and 
$A \cap B = A \backslash (A \backslash B)$. 
\end{proof}
\begin{defn}
\index{algebra}
\index{field of sets}
An algebra (or Boolean algebra) over a set $X$ is a nonempty $\pi$-system 
$\mathcal{A}$ on $X$ that is closed under the complement, \ie, 
$E\ in \mathcal{A} \implies A^C \in \mathcal{A}$. 
\end{defn}

\begin{prop}
\label{prop:measurable_sets:ring_algebra}
A ring $\mathcal{R}$ on a set $X$ is an algebra if and only if 
$X \in \mathcal{R}$.
\end{prop}
\begin{proof}
Let $A$ and $B$ be elements of an algebra $\mathcal{A}$. 
Then $A \backslash B = A \cap B^C \in \mathcal{A}$ and $A \cup B = 
\left( A^C \cap B^C \right)^C \in \mathcal{A}$. 
Since $\mathcal{A}$ is nonempty, there is an element $E \in \mathcal{A}$, 
whence $X = E \cup E^C \in \mathcal{A}$. 

Obviously, let $A$ and $B$ be elements of a ring $\mathcal{R}$ containing 
$X$. 
Then $A^C = X \backslash A \in \mathcal{R}$. 
By Proposition \ref{prop:measurable_sets:ring_semiring} and definition of 
algebra, $\mathcal{R}$ is an algebra. 
\end{proof}

The classes defined above have the property of closedness under finite 
operations, which is insufficient for measurable sets. 
%%%%%%%%%%%%%%%%%%%%%%%%%%%%%%%%%%%%%%%%%%%%%%%%%%%%%%%
%%  Monotone class
%%%%%%%%%%%%%%%%%%%%%%%%%%%%%%%%%%%%%%%%%%%%%%%%%%%%%%%
\begin{defn}
\index{monotone class}
A nonempty class $\mathcal{M}$ of sets is \emph{monotone} if, for every 
monotone sequence $\{E_n\}$ of sets in $\mathcal{M}$, we have 
\begin{equation*}
    \lim_{n \to \infty} E_n \in \cM. 
\end{equation*}
\end{defn}

\begin{defn}
\index{$\lambda$-system}
A \emph{$\lambda$-system} on a set $X$ is a class $\cL$ of sets of $X$ 
such that 
\begin{enumerate}
    \item $X \in \cL$; 
    \item $A, B \in \cL \text{ and } B \subseteq A 
    \implies A \backslash B \in \cL$; 
    \item $\{A_n\}$ is an increasing sequence of sets $\cL$ of $\implies 
    \bigcup_{n=1}^\infty A_n \in \cL$. 
\end{enumerate}
\end{defn}

\begin{prop}
A $\lambda$-system is a monotone class. 
\end{prop}
\begin{proof}
It is trivial. 
\end{proof}

\begin{defn}
\index{$\sigma$-algebra}
A \emph{$\sigma$-system} on a set $X$ is a class of sets $\cF$ of $X$ 
such that 
\begin{enumerate}
    \item $X \in \cF$; 
    \item $A \in \cF \implies A^C \in \cF$; 
    \item $A_n \in \cF, n = 1, 2, \ldots 
    \implies \bigcup_{i=1}^\infty A_n \in \cF$. 
\end{enumerate}
\end{defn}
If $\{B_n\}$ is a sequence in a $\sigma$-algebra, then 
\begin{equation*}
    \bigcap_{n=1}^\infty B_n = \left( \bigcup_{n=1}^\infty B_n^C \right)^C 
    \in \cF. 
\end{equation*}

\begin{example}
Let $X$ be a set. The class $\{\varnothing, X\}$ is the minimal $sigma$-
algebra on $X$ while the power set of $X$ is the maximal $\sigma$-
algebra on $X$. 
\end{example}

\begin{prop}
A $\sigma$-algebra is a $\lambda$-system. 
\end{prop}
\begin{proof}
Let $A$ and $B$ belong to a $\sigma$-algebra $\cF$. 
Then $A \backslash B = A \cap B^C \in \cF$. 
\end{proof}

\begin{prop}
A $\sigma$-algebra is an algebra. 
\end{prop}
\begin{proof}
Trivial. 
\end{proof}

In conclusion, we have the following relations: 
\begin{quote}
$\pi$-systems $\supset$ semirings $\supset$ rings $\supset$ algebras 
$\supset$ $\sigma$-algebras. 

monotone classes $\supset$ $\lambda$-systems $\supset$ $\sigma$-algebras. 
\end{quote}

Amongst these classes, the most essential one is $\sigma$-algebra, which is 
used to establish measure. 
The elements of $\sigma$-algebra on which the $measure$ is defined are what 
we call measurable sets. 

\begin{prop}
\label{prop:measurable_sets:construction_of_sigma_algebra}
A class is a $\sigma$-algebra if 
\begin{enumerate}
    \item it's a monotone algebra; 
    \item it's both a $\pi$-system and a $\lambda$-system. 
\end{enumerate}
\end{prop}

%%%%%%%%%%%%%%%%%%%%%%%%%%%%%%%%%%%%%%%%%%%%%%%%%%%%%%%
%%  Sigma-ring
%%%%%%%%%%%%%%%%%%%%%%%%%%%%%%%%%%%%%%%%%%%%%%%%%%%%%%%
\begin{defn}
\index{$\sigma$-ring}
A \emph{$\sigma$-ring} on a set $X$ is a ring $\cS$ on $X$ that is closed under the 
formation of countable unions. 
\end{defn}
If $\cS$ is a $\sigma$-ring on $X$, there is an element in $\cS$ and thus 
$\varnothing \in \cS$. 
Furthermore, $X \in \cS$ and $A \in \cS \implies A^C \in \cS$. 
It is easy to verify that a $\sigma$-ring containing $X$ is a 
$\sigma$-algebra. 

\begin{prop}
A $\sigma$-ring is a monotone class; a monotone ring is a $\sigma$-ring. 
\end{prop}
\begin{proof}
The first statement is obvious. 
To prove the second assertion, we must show that a monotone ring is closed 
under the formation of countable union. 
If $\cM$ is a monotone ring and if $\{A_n\}$ is a sequence of sets of $\cM$, 
then the sequence 
\begin{equation*}
    B_n = \bigcup_{k=1}^n A_n
\end{equation*}
is a monotone sequence of sets whose limit is $\bigcup_{k=1}^{\infty} A_n$. 
Since $\cM$ is monotone, $\bigcup_{k=1}^\infty A_k = \lim_{n \to \infty} B_n 
\in \cM$. 
\end{proof}

%%%%%%%%%%%%%%%%%%%%%%%%%%%%%%%%%%%%%%%%%%%%%%%%%%%%%%%
%%  Section: generated rings and algebras
%%%%%%%%%%%%%%%%%%%%%%%%%%%%%%%%%%%%%%%%%%%%%%%%%%%%%%%
\section{Generated Rings and Algebras}
\begin{prop}
\label{prop:measurable_sets:existence_of_generated_rings}
Let $X$ be a set. 
If $\cE$ is a class of sets of $X$, then there is a unique ring 
$\cR$ (\resp, monotone class, $\lambda$-system, or $\sigma$-algebra) 
such that $\cR \supset \cE$ and if $\cR'$ is another ring (\resp, monotone 
class, $\lambda$-system, or $\sigma$-algebra) on $X$ containing $\cE$, 
then $\cR \subseteq \cR'$. 
\end{prop}
\begin{proof}
Since the class of all subsets of $X$ is a ring, the collection of rings 
containing $\cE$ is nonempty. 
It is easy to verify that the intersection any collection of rings is also 
a ring. 
Thus the the intersection of all rings containing $\cE$ is the ring with 
disired property. 
The proofs of other cases are similar. 
\end{proof}

\begin{defn}
\index{ring!generated by a class}
The ring (\resp, monotone class, $\lambda$-system, $\sigma$-algebra) is 
called the ring (\resp, monotone class, $\lambda$-system, $\sigma$-algebra) 
generated by $\cE$; it will be denoted by $\cR(\cE)$ (\resp, $\cM(\cE))$, 
$\cA(\cE)$). 
\end{defn}

The tree core theorems of this section are as follows. 

\begin{thm}
If $\cD$ is a semiring, then 
\begin{equation}
    \label{equ:measurable_sets:generated_ring_of_semiring}
    \cR(\cD) = \left\{ \bigcup_{k=1}^n U_k: 
    n \in \bN^\ast, 
    U_k \in \cD \text{ are disjoint} \right\}. 
\end{equation}
\end{thm}
\begin{proof}
Denote by $\cB$ the set given by Equation 
(\ref{equ:measurable_sets:generated_ring_of_semiring}).
Clearly, $\cB \subseteq \cR(\cD)$. 
Supposint that $A = \bigcup_{k=1}^{n_1} A_k \in \cD$ and 
$B = \bigcup_{k=1}^{n_2} B_k \in \cD$, then 
\begin{equation*}
    \begin{aligned}
        A \backslash B 
        &= \left( \bigcup_{k=1}^{n_1}A_k \right) \backslash
        \left( \bigcup_{l=1}^{n_2}B_k \right) 
        = \bigcup_{k=1}^n \left( 
        A_k \backslash \bigcup_{l=1}^{n_2} B_l\right) \\
        &= \bigcup_{k=1}^{n_1} \bigcap_{l=1}^{n_2} 
        \left( A_k \backslash B_l \right) 
        = \bigcup_{k=1}^{n_1} \bigcap_{l=1}^{n_2} 
        \left( A_k \backslash (A_k \cap B_l) \right) \\ 
        &= \bigcup_{k=1}^{n_1} \bigcap_{l=1}^{n_2} 
        \left( \bigcup_{i=1}^{n_{kl}} C_i^{k, l} \right)
        \in \cB
    \end{aligned}
\end{equation*}
where $\{C_i^{k, l} \in \cD : i = 1, 2, \ldots, n_{kl}\}$ are disjoint for 
each pair $(k, l)$. 

Then $A \cup B = A \cup (B \backslash A)$ since $A$ and $B \backslash A$ 
are disjoint and are elements of $\cB$ and thus can be written as unions 
of finite disjoint sets of $\cD$. 
Thus, $\cB$ is a ring. 

Now we have to show that $\cB$ is a minimal ring containing $\cD$. 
This is true because of closedness of the formation of finite unions. 
\end{proof}

\begin{thm}
\label{thm:measurable_sets:algebra_ma_fa}
If $\cA$ is a algebra on a set $X$, then $\cM(A) = \cF(A)$. 
\end{thm}
\begin{proof}
On the one hand, $\cM(\cA) \subseteq \cF(\cA)$, since $\cF(\cA)$ is monotone. 
On the other hand, to prove $\cF(\cA) \subseteq \cM(\cA)$, it suffices to 
prove that $\cM(\cA)$ is a $\sigma$-algebra, by Proposition 
\ref{prop:measurable_sets:construction_of_sigma_algebra}. 
However, to show this it suffices to show that $\cM(\cA)$ is a ring since 
$X \in \cA \subseteq \cM(\cA)$ by Proposition 
\ref{prop:measurable_sets:ring_algebra}. 

First, we show that for any $A \in \cA$ and $B \in \cM(\cA)$, 
\begin{equation}
    \label{equ:measurable_sets:algebra_ma_fa:1}
    A \cup B \in \cM(\cA) \text{ and } A \backslash B \in \cM(\cA). 
\end{equation}
Setting 
\begin{equation*}
    \cH_A = \{B \in \cM(\cA): B \cup A \in \cM(\cA), 
    A \backslash B \in \cM(\cA)\}, 
\end{equation*}
it is not difficult to verify that $\cH_A$ is monotone and $\cA \subseteq 
\cH_A$, whence $\cM(\cA) \subseteq \cH_A$ and Equation 
(\ref{equ:measurable_sets:algebra_ma_fa:1}) follows. 

Second, setting 
\begin{equation*}
    \cG_B = \{A \in \cM(\cA): A \cup B \in \cM(\cA), 
    A \backslash B \in \cM(\cA)\}, 
\end{equation*}
likewise, $\cG_B$ is monotone and by Equation 
(\ref{equ:measurable_sets:algebra_ma_fa:1}), $\cG_B \supset \cA$, 
whence $\cM(\cA) \subseteq \cG_B$. 
Therefore, for any $A, B \in \cM(\cA)$, 
\begin{equation*}
    A \cup B \in \cM(\cA) \text{ and } A \backslash B \in \cM(\cA). 
\end{equation*}
The conclusion is established. 
\end{proof}

In practice, an equivalent form of Theorem 
\ref{thm:measurable_sets:algebra_ma_fa} is frequently used. 
\begin{cor}
If $\cA$ is an algebra and $\cM$ is a monotone class, then 
\begin{equation*}
    \cA \subseteq \cM \implies \cF(\cA) \subseteq \cM. 
\end{equation*}
\end{cor}

\begin{thm}
If $\cP$ is a $\pi$-system on a set $X$, then $\cL(\cP) = \cF(\cP)$. 
\end{thm}
\begin{proof}
Since $\sigma$-algebras are $\pi$-systems, we have $\cF(\cP) \subseteq 
\cL(\cP)$. 
Now we prove $\cF(\cP) \supset \cL(\cP)$ and it suffices to prove that 
$\cL(\cP)$ is a $\pi$-system thus a $\sigma$-algebra. 

Putting $A \in \cP$ and $B \in \cL(\cP)$, let 
\begin{equation*}
    \cH_A = \{B \in \cL(\cP): B \cap A \in \cL(\cP)\}, 
\end{equation*}
which is a $\lambda$-system containing $\cP$, whence $\cL(\cP) \subseteq 
\cH_A$. 

Further, set 
\begin{equation*}
    \cG_B = \{A \in \cL(\cP): A \cap B \in \cL(\cP)\}, 
\end{equation*}
which is also a $\lambda$-system containing $\cP$, whence $\cL(\cP) \subseteq 
\cG_B$, \ie, for $A, B \in \cL(\cP)$ 
\begin{equation*}
    A \cap B \in \cL(\cP). 
\end{equation*}
The conclusion is established. 
\end{proof}

This theorem has an equivalent form: 
\begin{cor}
If $\cP$ is a $\pi$-system and $\cL$ is a $\lambda$-system, then 
\begin{equation*}
    \cP \subseteq \cL \implies \cF(\cP) \subseteq \cL. 
\end{equation*}
\end{cor}

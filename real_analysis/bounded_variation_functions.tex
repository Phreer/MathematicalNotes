\chapter{Functions of Bounded Variation}
\begin{defn}[total variation]
\index{total variation}
The \emph{total variation} of a continuous real-valued function $f$, 
defined on an interval $[a, b] \subseteq \bR$ is the quantity 
\begin{equation*}
    V_a^b(f) = \sup_{P \in \cP} \sum_{i=0}^{n_P - 1} 
        \abs{f(x_{i+1}) - f(x_i)}
\end{equation*}
where the supremum is taken over the set 
\begin{equation*}
    \cP = \left\{ P = \left\{ \left(x_0, x_1, \ldots, x_{n_P} 
        \right)\right\}: a = x_0 < x_1 < \cdots < x_{n_{P}} = b \right\}
\end{equation*}
of all partitions of the interval $[a, b]$. 
Denote by $V_P(f)$ the variation of $f$ over partion $P$, \ie, 
\begin{equation*}
    V_P(f) = \sum_{i=0}^{n_P-1} \abs{f(x_{i+1}) - f(x_i)}. 
\end{equation*}
\end{defn}

\begin{defn}[BV function]
\index{BV function}
\index{function of bounded variation}
\label{def:bounded_variation_function}
A function $f \in C[a, b]$ is of bounded variation (BV function) if its 
total variation is finite. 
\end{defn}